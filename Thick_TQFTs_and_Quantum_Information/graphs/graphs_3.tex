
\subsection{Constructs on Expression Graphs}

Consider expression graphs $G$ and $H$. We can take the disjoint unions of their
vertex and edge sets. It is clear that the edge orderings of the vertices of
$G$ and $H$ collectively provide an edge ordering for every vertex of
$G \amalg H$. We observe that $S(G \amalg H) = S(G) \amalg S(H)$ so that the
orderings of $S(G)$ and $S(H)$ provide an ordering of $S(G \amalg H)$, where the
vertices of $G$ come before those of $H$. The empty graph is an expression and
hence acts as a unit for the disjoint union operation. It is easy to see that
the associators and unitors for the coproduct in the category of sets are
expression isomorphisms.

We then proceed to define a gluing of expression graphs. For an expression
graph $G$, $S(G)$ without any edges is itself an expression graph.
The edgeless vertices are in both $S(G)$ and $T(G)$. Since $S(G)$ is also the
first level of $G$, $S(G)$ is ordered, by definition. We then observe that
$T(G)$ is the union of the last level and the set of edgeless vertices. Thus,
using a method similar to point \ref{alg:edgeless} in the algorithm given in
\S\ref{subsec:alg_graph_exp}, we have an induced ordering of $T(G)$, so that
$T(G)$ is also an expression graph without any edges.
There are obvious order-preserving expression homomorphisms
$S(G) \hto G \hot T(G)$ that are isomorphisms onto their images. We note
that if there is a an expression isomorphism $\psi : S(H) \to T(G)$ for
expression graphs $G$ and $H$, then $\psi$ is unique by \ref{cor:expiso_unique}.
This allows us to define the following notion of gluing.

\begin{defn}[Gluing of Expression Graphs]
Let $G$ and $H$ be expression graphs such that there is a unique expression
isomorphism
\[
  \psi_{G, H} : S(H) \to T(G)
\]
Then, we say that $G$ and $H$ are gluable at $S(H) \cong T(G)$. We define the
pushout of the following span in $\Set$ to be the gluing $H * G$ of graphs.
\[
  H \hot S(H) \to[\phi] T(G) \hto G
\]
\end{defn}

\begin{thm}
Gluings $H * G$ of expression graphs $G$ and $H$ at $S(H) \cong T(G)$ are
expression graphs.
\end{thm}
\begin{proof}
First we show that $S(H * G) = S(G)$. If $v \in S(H * G) \setminus S(G)$, then
$v$ has an incoming edge in $G$ and since gluing does not delete edges, $v$ has
an incoming edge in $H * G$, so that $v \not\in S(H * G)$. Thus,
$S(H * G) \subseteq S(G)$. If $v \in S(G)$, then $v$ has no incoming edges in
$G$. It is clear that gluing cannot introduce incoming edges to the source
vertices of $G$ so that $S(G) \subset S(H * G)$. Thus, $H * G$ also inherits the
ordering of its source vertices from $G$.

We observe that only the vertices in $T(G)$ receive new edges and these are all
outgoing while the vertices in $T(G)$ have no outgoing edges in $G$. Thus,
$H * G$ inherits edge orderings unambiguously from $G$ and $H$ collectively.
Therefore, $H * G$ is an expression graph.
\end{proof}

\begin{exm}\label{exm:expression_gluing}
Consider $G$ from example \ref{exm:egraph1} and $H$ as follows:
\[\begin{tikzpicture}[xscale=2,yscale=0.75]
\node at (3, 3) (v10) {$10$};
\node at (3, 1) (v11) {$11$};
\node at (4, 3) (v12) {$12$};
\node at (4, 1) (v13) {$13$};
\node at (5, 4) (v14) {$14$};
\node at (5, 2) (v15) {$15$};
\node at (5, 0) (v16) {$16$};
\midarrow[0.33]{v8}{v13}
\midarrow[0.33]{v9}{v12}
\midarrow{v8}{v14}
\midarrow{v12}{v14}
\midarrow{v12}{v15}
\midarrow{v13}{v15}
\midarrow{v13}{v16}
\end{tikzpicture}\]
We have the following diagram of $H * G$:
\[\begin{tikzpicture}[xscale=2,yscale=0.75]
\node at (0, 3) (v1) {$1$};
\node at (0, 1) (v2) {$2$};
\node at (1, 3) (v3) {$3$};
\node at (1, 1) (v4) {$4$};
\node at (2, 4) (v5) {$5$};
\node at (2, 2) (v6) {$6$};
\node at (2, 0) (v7) {$7$};
\node at (3, 3) (v8) {$8 \cong 10$};
\node at (3, 1) (v9) {$9 \cong 11$};
\node at (4, 3) (v12) {$12$};
\node at (4, 1) (v13) {$13$};
\node at (5, 4) (v14) {$14$};
\node at (5, 2) (v15) {$15$};
\node at (5, 0) (v16) {$16$};
\midarrow{v1}{v3}
\midarrow[0.33]{v1}{v4}
\midarrow[0.33]{v2}{v3}
\midarrow{v3}{v5}
\midarrow{v3}{v6}
\midarrow{v4}{v7}
\midarrow{v5}{v8}
\midarrow{v6}{v9}
\midarrow{v7}{v9}
\midarrow[0.33]{v8}{v13}
\midarrow[0.33]{v9}{v12}
\midarrow{v8}{v14}
\midarrow{v12}{v14}
\midarrow{v12}{v15}
\midarrow{v13}{v15}
\midarrow{v13}{v16}
\end{tikzpicture}\]
\end{exm}

It is then easy to verify that gluing of expression graphs is associative and
unital up to expression isomorphism much like the disjoint union. We can further
verify that the data of expression graphs defined so far form a monoidal double
category whose objects are finite, ordered sets, vertical $1$--morphisms are
unique order isomorphisms, horizontal $1$--morphisms $G : U \to V$ are
expression graphs $G$ with $S(G) \cong U$ and $T(G) \cong V$, and $2$--morphisms
are expression isomorphisms, with horizontal composition given by gluing and
monoidal product given by disjoint union.

We now observe some properties of the expression construction algorithm of
\S\ref{subsec:alg_graph_exp}.

\begin{exm}
We observe that, in general, the expression construction does not result in the
same graph when we apply it before gluing as opposed to after gluing. Let
$G$ and $H$ be as follows:
\[
\begin{tikzpicture}[yscale=0.5]
\node at (0, 3) (v1) {$1$};
\node at (0, 1) (v2) {$2$};
\node at (2, 3) (v3) {$3$};
\midarrow{v1}{v3}
\node at (1, -1) (G) {$G$};
\end{tikzpicture}
\qquad
\qquad
\begin{tikzpicture}[yscale=0.5]
\node at (0, 3) (v4) {$4$};
\node at (0, 1) (v5) {$5$};
\node at (2, 3) (v6) {$6$};
\node at (2, 2) (v7) {$7$};
\node at (2, 0) (v8) {$8$};
\midarrow{v4}{v6}
\midarrow{v5}{v7}
\midarrow{v5}{v8}
\node at (1, -1) (H) {$H$};
\end{tikzpicture}
\]
Applying the expression construction on $G$ and $H$ separately and then gluing
the results yields:
\[
\begin{tikzpicture}[yscale=0.5]
\node at (0, 3) (v1) {$1$};
\node at (0, 1) (v2) {$2$};
\node at (2, 3) (v4) {$3 \cong 4$};
\node at (2, 1) (v5) {$2 \cong 5$};
\node at (4, 3) (v6) {$6$};
\node at (4, 2) (v7) {$7$};
\node at (4, 0) (v8) {$8$};
\midarrow{v1}{v4}
\midarrow{v2}{v5}
\midarrow{v4}{v6}
\midarrow{v5}{v7}
\midarrow{v5}{v8}
\end{tikzpicture}
\]
Applying the expression construction on $H * G$ results in:
\[
\begin{tikzpicture}[yscale=0.5]
\node at (0, 3) (v1) {$1$};
\node at (2, 3) (v4) {$3 \cong 4$};
\node at (0, 1) (v5) {$2 \cong 5$};
\node at (4, 3) (v6) {$6$};
\node at (2, 2) (v7) {$7$};
\node at (2, 0) (v8) {$8$};
\node at (4, 2) (v77) {$7$};
\node at (4, 0) (v88) {$8$};
\midarrow{v1}{v4}
\midarrow{v4}{v6}
\midarrow{v5}{v7}
\midarrow{v5}{v8}
\midarrow{v7}{v77}
\midarrow{v8}{v88}
\end{tikzpicture}
\]
However, in this case, we observe that $\Exp{H * G} = \Exp{H} \circ \Exp{G}$
because the corresponding expressions in $\Cob_2$ are equal!
\end{exm}

In fact, we have the following results which show that all differences that can
arise between expression constructions before and after gluing can be reduced to
simple cases similar to the one above.

\begin{lem}
For expression graphs $G$ and $H$ with $n$ and $m$ levels respectively and
gluable at $S(H) \cong T(G)$, the number of levels in $H * G$ is $n + m - 1$.
\end{lem}
\begin{proof}
Let the level sets of $G$ be $V_1, \dots, V_n$ and
those of $H$, be $W_1, \dots, W_m$. By \ref{cor:lvltolvl}, there exists a path
$w_1, \dots, w_m$ with $w_i \in W_i$ in $H$, and also a path
$v_1, \dots, v_n = w_1$ with $v_i \in V_i$ in $G$. The concatenation of these
paths shows that $H * G$ has at least $n + m - 1$ levels.

Let $X_1, \dots, X_k$ be the level sets of $H * G$. For any vertex
$x \in X_k$, by \ref{cor:lvltolvl}, there must be a path $x_1, \dots, x_k = x$
in $H * G$ with $x_i \in X_i$. However, since the edges of $H * G$ are the edges
of $G$ or $H$, we must have that all $x_1, \dots, x_{j}$ are in
$G \setminus H$, $x_{j + 1} \in G \cap H = S(H) \cong T(G)$ and all
$x_{j + 2}, \dots, x_k$ are in $H \setminus G$, for some $j$. If this is
not the case, then we must have an edge from some vertex in $H$ to some vertex
in $G$, which is impossible. Therefore, $H * G$ has at most $n + m - 1$ levels.
\end{proof}

We observe that the first five steps of the algorithm from
\S\ref{subsec:alg_graph_exp} commute with gluing. One might think that the sixth
step onwards does not. However, we will see that it does. For this, we note the
following result about changes in level sets after gluing.

\begin{thm}
For expression graphs $G$ and $H$ gluable at $S(H) \cong T(G)$,
the level sets of $H * G$ are formed by taking the level sets of $G$ and those
of $H$, identifying the first level of $H$ with the last level of $G$ and moving
vertices between levels in the following pattern:
\begin{enmrt}
\li Move vertices in $S(H)$ which glue to edgeless vertices to the first level.
\li For each such vertex $v$ in $S(H)$ and each edge $(v, v')$ where $(v, v')$
is the first incoming edge of $v$, move $v'$ to the second level.
\li Repeat the previous step with $v'$ in place of $v$.
\end{enmrt}
\end{thm}
\begin{proof}
By the previous lemma, it makes sense to say that the levels of $H * G$ are
formed by taking the levels of
$G$ and $H$ and moving vertices between them. Let the levels of $G$ be $V_1,
\dots, V_n$ and those of $H$ be $W_1, \dots, W_m$. We will informally say that
$V_n = W_1$ and that the levels of $H * G$ are
$V_1, \dots, V_n $ (or $W_1$)$, \dots, W_m$ for the sake of simplifying the
language, although the movement of vertices makes the equality false. The cases
for movement of vertices in the gluing $H * G$ are as follows:
\begin{enmrt}
\li If $v \in V_i$ for $1 \leq i \leq n$ before gluing, then $v$ cannot move to
another $V_j$ or $W_k$.
We proceed by induction. For $i = 1$, we observe that $S(H * G) = S(G) = V_1$
and hence no $v \in V_1$ can move forward since it has no incoming edges. For
$i = r + 1$, if $v \in V_i$, then there exists an edge $(v', v) \in G$ with
$v' \in V_{r}$ and $v'$ does not move by induction. Hence, $v'$ cannot move to
$V_j$ for $j < i$ since $(v', v)$ would be an edge that does not go forward in
levels. Since all the incoming edges of $v \in V_i$ are from
$V_1, \dots, V_{i - 1}$, $v'$ cannot move to $V_j$ for some $j > i$ or $W_k$ for
some $k$, because this would result in $v$ not having edges from the immediate
previous level.

\li If $v \in S(H)$ before gluing, then $v$ can only move when $v$ glues to some
vertex $v_G \in G$ that has no incoming edges in $G$ and, in this case,
$v_G \in V_1$ in $H * G$ and $v$ moves to $V_1$ by gluing to $v_G$. This is true
because if $v_G$ has some incoming edges in $G$, then $v_G \in V_n$ and we are
in the previous case so that $v_G$ stays in its level and $v$ merely to glues to
$v_G$.

\li If $v \in W_k$ for some $1 < k \leq m$ before gluing, then there exists a
path $w_1, \dots, w_k = v$ with $w_i \in W_i$ in $H$. We can additionally assume
that for each $i$, $(w_i, w_{i + 1})$ is the first incoming edge of $w_{i + 1}$.
After gluing, if $v$ moves to some $W_{k'}$, then the first incoming edge
$v = w_k$ is from $w_{k - 1}$ so that $w_{k - 1}$ must have also moved to
$W_{k' - 1}$, by our definition of level ordering. Repeating this argument, we
see that $w_1$ must move forward in levels which is impossible by the previous
case. If, on the other hand, $v$ moves backwards, we must have each $w_i$ move
backwards as well because otherwise, we will have at least one edge not going
forward in levels. In particular, $w_1$ must move backwards and hence to $V_1$.
In this case, $w_2$ moves to $V_2$, $w_3$ to $V_3$ and so on until $w_k$ moves
to $V_k$ if $k \leq n$. If $k > n$, then $w_1, \dots, w_n$ move to
$V_1, \dots, V_n$ and $w_{n + 1}, \dots, w_{n + k - n} = w_k$ move to
$W_1, \dots, W_{k - n}$ respectively.
\end{enmrt}
\end{proof}

Whether we construct level sets before or after gluing, such movements must take
place and hence the sixth step where we construct the levels commutes with
gluing. We then observe that applying the seventh step before gluing results in
no edgeless vertices in both graphs so that there are no movement of vertices
between levels by the above theorem. Applying the seventh step after gluing
results in movement of vertices and then ``extending'' some vertices to the last
level. In either case, even though we do not obtain the same graph, the parts
that differ, do so only by prefixes and suffixes of edges between copies of the
same vertex. Similar reasoning establishes the same about the eigth step, so
that the resulting cobordisms differ by identity cobordisms at some places and
hence not at all, yielding the following theorem.

\begin{thm}
For expression graphs $G$ and $H$, gluable at $S(H) \cong T(G)$,
$\Exp{H * G} = \Exp{H} \circ \Exp{G}$.
\end{thm}

We can also define the following constructs unambiguously.

\begin{defn}[Expression Tensor Product]
Let $G$ and $H$ be expression graphs. We define:
\[
  \Exp{G} \tensor \Exp{H} := \Exp{G \amalg H}
\]
\end{defn}

\begin{defn}[Expression Substitution]
Given an expression graph $G$, we write
\[
  \Exp{G}[f] \text{ or } \Exp{G}[f(u, v)] \text{ or } \Exp{G}[(u, v)/f(u, v)]
\]
to denote the expression obtained by replacing each edge $(u, v) \in G$ in
$\Exp{G}$ with some string of symbols $f(u, v)$, depending on $(u, v)$.
\end{defn}

