
\subsection{Single Manifold TQFT}\label{subsec:sing_man_tqft}

Notice that the theory we developed so far transfers verbatim from the smooth
setting to the setting of complex bundles in that we can replace all instances
of $\R$ with $\C$, smooth bundles and $\R$--linear connections with smooth
manifolds, complex bundles and $\C$--linear connections on complex bundles
respectively. From this point onwards, when we say manifold, bundle, connection,
etc., we will mean these in either of the two settings. We will write $\K$ to
mean either of $\R$ or $\C$. With this convention, we consider the following
data for a monoidal double category:

\begin{enmrt}
\li Object category: objects are totally ordered, $2$--coloured, finite sets;
morphisms are order-preserving, colour-preserving (unique) bijections
$V \stackrel{!}{\longleftrightarrow} V'$

\li Morphism category: objects (horizontal $1$--morphisms) are transport graphs
$(G, \gamma)$, where for a fixed manifold $M$, a fixed bundle $\pi : E \to M$,
and a fixed connection $\nabla$ on $\pi$, $G$ is a pretransport graph with
$\gamma$ a geometric realization of $G$ in $M$; morphisms are tuples
\[
  (f_0, f_1, h) : (G_1, \gamma^1) \to (G_2, \gamma^2)
\]
where $(f_0, f_1)$ is an automorphism of the connection $\nabla$ and
$(f_0, h)$ is a transport isomorphism $(G_1, \gamma^1) \to (G_2, \gamma^2)$
\footnote{It is possible that this condition forces $(f_0, f_1) = (\id, \id)$.}

\li Source functor: $S : G \mapsto S(G)$; for a $2$--morphism $(f_0, f_1, h)$,
$S(f_0, f_1, h)$ is the unique order-preserving bijection
$S(\dom h) \to S(\codom h)$

\li Target functor: $T : G \mapsto T(G)$, defined similarly as $S$

\li Unit functor: $U : V \mapsto V$; $U : f \mapsto f$ -- each finite,
totally-ordered, $2$--coloured set is a transport graph with no edges and
order- and colour- preserving bijections between finite sets are unique
transport isomorphisms

\li Horizontal composition:
$(G_2, \gamma^2) * (G_1, \gamma^1) = (G_2 * G_1, \gamma^2 * \gamma^1)$

\li Horizontal associators: inherited from the categories of sets and manifolds

\li Horizontal unitors: inherited like associators

\li Monoidal product: disjoint union

\li Monoidal unit: empty set for object category, empty graph for morphism
category
\end{enmrt}

\begin{defn}[Double Category of Transport Graphs in a Manifold]
The above data defines the double category of transport graphs in $M$ and we
denote it as $\TG(M)$. We denote the object category as $\TG(M)_0$ and the
morphism category as $\TG(M)_1$.
\end{defn}

We then consider a double functor defined as follows.

\begin{defn}[Parallel Field Theory over a Manifold]\label{defn:sing_man_tqft}
For a finite, totally-ordered, $2$--coloured set
$V = \set{v_1, \dots, v_n}$ in $\TG(M)_0$, we set
\[
  F(V) := \bigotimes_{i = 1}^{n} c(v_i)
\]
where $c(v) = A$ if $v$ is blue and $c(v) = \K$ if $v$ is green,
for some $\K$--algebra $A$.

For every unique order- and colour-preserving bijection $f : V \to V'$ in
$\TG(M)_0$, we set $F(f) := \id_{F(V)} = \id_{F(V')}$.

For a transport graph $(G, \gamma)$ in $M$ -- an object in $\TG(M)_1$ -- and an
edge $(u, v) \in G$, we denote $\nabla^{\gamma_{u, v}}$ to be the linear map
$A \to A$ obtained by parallel transport along $\gamma_{u, v}$, with respect to
$\nabla$. Fixing some element $a_{u, v} \in A$, we then define:
\[
  F(u, v) := \begin{cases}
    \id_A
      & u = v \text{ is blue} \\
    \id_{\K} % 1 \mapsto \text{trace}\br{\nabla^{\gamma_{u, v}}(a_{u, v})}
      & u \text{ is green and } v \text{ is green} \\
    \nabla^{\gamma_{u, v}}
      & u \text{ is blue and } v \text{ is blue} \\
    1 \mapsto \nabla^{\gamma_{u, v}}(a_{u, v})
      & u \text{ is green and } v \text{ is blue} \\
    \text{trace} \circ \nabla^{\gamma_{u, v}}
      & u \text{ is blue and } v \text{ is green}
  \end{cases}
\]
We then obtain a linear map:
\[
  F(G, \gamma) := \Exp{G}[F(u, v)]
               :  F(S(G)) \to F(T(G))
\]
For a $2$--morphism
$(f_0, f_1, h) : (G_1, \gamma_1) \to (G_2, \gamma_2)$ in $\TG(M)_1$, we consider
the path $(r_t, s_t)$ in the isomorphism group of the connection $\nabla$ such
that $(r_0, s_0) = (\id_M, \id_E)$ and $(r_1, s_1) = (f_0, f_1)$. We then have a
smoothly varying family of functions $s_t\gamma : E(G) \to C^0(I, M)$
where $(s_t\gamma)_{u, v} = s_t \circ \gamma_{u, v}$. This yields a smoothly
varying family of linear maps, which we write as:
\[
  F(f_0, f_1, h) := \Exp{G}[s_t\gamma]
    : F(S(G)) \to F(T(G)), t \in [0, 1]
\]
We call $F$ a parallel field theory on $M$.
\end{defn}

\begin{rmk}\label{rmk:any_vect_space}
We note that so far we have considered $\cwedge$ and $\cvee$ to mean the product
and its dual for some algebra $A$. This is not strictly necessary. We could, in
principle, choose, for each instance of $\cwedge$ in an expression of a
transport graph, a distinct map $A \tensor A \to A$ when defining a parallel
field theory as above, as long as it does not disturb the monoidal double
functoriality of $F$. In fact, we could replace $A$ with an arbitrary vector
space $V$ and work with arbitrary linear maps $V \tensor V \to V$ and
$V \to V \tensor V$ in replacing $\cwedge$ and $\cvee$ in expressions arising
from transport graphs.
\end{rmk}

We note that the definition of $F$ does not specify the codomain. We will define
the codomain double category in the next section along with a notion of TQFTs
based on transport graphs in cobordisms equipped with connections.

We also note that there is some redundancy in this setup. The empty graph, the
single green vertex and paths with only green vertices are not the same
horizontal $1$--morphism but they are all morphisms
$\varnothing \to \varnothing$ and map to the identity morphism of $\K$ under
$F$. This differs from $\Cob_2$ and $2\Thick$ in that there are unique
morphisms $\varnothing \to \varnothing$ in these categories as well as their
corresponding double categories that map to the identity morphism of $\K$ under
a usual TQFT.

