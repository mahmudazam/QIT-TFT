
\subsection{Quantum Computing with Parallel Transport}

Take $A$ to be the complex matrix algebra $\M_2(\C)$ of $2 \times 2$ complex
matrices with the usual multiplication. Let
$\mathcal{U} = \set{u_1, \dots, u_n}$ be a set
of single qubit quantum gates -- unitary matrices -- in $\M_2(\C)$ such that
each $u_k$ is accessible from $\DThick$. Let the thick tangle with a bundle,
a connection and an admissible transport graph that realizes the accessibility
of the $u_k$ be $M_k$, for each $k \in \set{1, \dots, n}$.

We recall that, in the simplest terms, a quantum circuit is a sequence of
composeable complex linear unitary maps
$U_i : (\C^2)^{\tensor N} \to (\C^2)^{\tensor N}$, $i = 1, \dots, p$.
For simplicity, we assume that
each $U_i$ arises as a tensor product $\bigotimes_{j = 1}^{N} g_{i, j}$ for
gates $g_{i, j} \in \mathcal{U}$. Since each $g_{i, j}$ is accessible, we have
a $G_{i, j} \in \set[M_k]{k = 1, \dots, n}$ such that $F(G_{i, j}) = g_{i, j}$
for a parallel field theory $F$. Then, we have:
\[
 U := U_p \circ \cdots \circ U_1
  = \bigotimes_{j = 1}^{N} g_{p, j} \circ \cdots
    \circ \bigotimes_{j = 1}^{N} g_{1, j}
  = \bigotimes_{j = 1}^{N} (g_{p, j} \circ \cdots \circ g_{1, j})
\]
where $\circ$ is matrix multiplication (composition of linear maps). Consider
the following transport graph in the pair-of-pants:

\[\begin{tikzpicture}[scale=0.25]
\pants{1, 0}
\colvert{blue}{-1, 6}{s1}
\colvert{blue}{-1, 4}{s2}
\colvert{blue}{3, 5}{t}
\colvert{blue}{1, 0}{a}
\midarrow{s1}{t}
\midarrow{s2}{t}
\end{tikzpicture}\]
where each edge is geometrically realized as a constant path on a single point
-- shown as a blue dot -- in the pair-of-pants. This graph is admissible and
provides a binary operation on thick tangles with target $I$ in the obvious way.
We denote this operation as $\wedge$. It is then easy to see that
\[
  U = F\br{\coprod_{j = 1}^{N} G_{p, j} \wedge \cdots \wedge G_{1, j}}
\]
We should stress that $\wedge$ is not associative, even up to isomorphism, in
$\DThick$ but its image under $F$ is. The reason for failure of associativity is
that graphs do not have the smooth structure needed for associator isomorphisms
for $\wedge$.

This setup provides a very basic formalism for expressing quantum
circuits in the language of thick tangles equipped with bundles, connections and
transport graphs.
We note, however, that it is not clear what structure plays the role of
qubits or registers in this picture. One easy way to get around this is to
consider the following embedding of $\C^2$ into $\M_2(\C)$:
\[
  \bmat{a \\ b} \mapsto \bmat{a & 0 \\ b & 0}
\]
If we then require that the following matrices are accessible:
\[
  \bmat{1 & 0 \\ 0 & 0} \text{ and } \bmat{0 & 0 \\ 1 & 0}
\]
we can model classical inputs with thick tangles just like gates. Multiplying
inputs with circuits using the pair-of-pants then models the application of
the circuit to the input.

One issue with this approach is that there might be gates acting on more
than one qubit that are not elementary tensor products of single qubit gates.
For instance, the controlled not gate is one such example. In order to obtain
non-elementary tensors, we will require addition of vectors. To capture this
in the language of parallel field theories, we need a notion of addition for
cobordisms. We will return to this idea in the next subsection. For now, we
observe another approach to quantum computing using parallel field theories.

We now take the fibres of our bundles to be $\C^2$ -- the space where a single
qubit lives (recall \ref{rmk:any_vect_space}).
We view our previous collection of quantum gates
$g_{i, j} \in \M_2(\C)$ as a collection of linear maps
$g_{i, j} : \C^2 \to \C^2$ and assume that the $g_{i, j}$ are accessible from a
collection of $2$--dimensional thick
tangles $G'_{i, j} : I \to I$ (not $\varnothing \to I$, this time) such that
$F(G'_{i, j}) = g_{i, j}$. Then, the quantum circuit $U$ can be expressed as:
\[
  U = F\br{\coprod_{j = 1}^{N} G'_{p, j} * \cdots
           * \coprod_{j = 1}^{N} G'_{1, j}}
\]
recalling that $*$ is gluing of thick tangles. Let $G$ be the input to $F$ in
the above equation.

Notice that single qubit inputs are linear maps $\C \to \C^2$. Hence, we now
assume that linear maps
\[
  \ket{0} = z \mapsto z\bmat{1 \\ 0} \text{ and }
  \ket{1} = z \mapsto z \bmat{0 \\ 1}
\]
are accessible. That is, a quantum register expressed as thick tangles is a
disjoint union of thick tangles (of course, equipped with transport graphs)
$\mathbf{0}, \mathbf{1} : \varnothing \to I$ realizing the accessibility of
$\ket{0}$ and $\ket{1}$ respectively. In this case, we will have
$F(\mathbf{0}) = \ket{0}$ and $F(\mathbf{1}) = \ket{1}$.

It is then easy to see that an application of a quantum circuit to a quantum
register is given by composition of thick tangles. That is, let $R$ be a thick
tangle formed from the disjoint union of a sequence of $\mathbf{0}$ and
$\mathbf{1}$. This represents the input register. We can then glue $R$ on the
source end of $G$ to obtain a thick tangle $H$. Then,
$F(H) = F(G * R) = F(G) \circ F(R)$ is the result of giving the circuit
$F(G)$ the input from the register $F(R)$.

