
\subsection{Morphisms of Expression Graphs}

It is of interest to define morphisms of expression graphs. They will be
necessary to define interesting constructions on expression graphs later on.

\begin{defn}[Expression Homomorphism]
An expression graph homomorphism or expression homomorphism is a graph
homomorphism $f : G \to H$ between expression graphs such that $f(S(G))$ is
contained in some level set $L$ in $H$, $f$ preserves the ordering of $S(G)$ in
$L$ and $f$ preserves the edge orderings for each vertex.
\end{defn}

We then note some useful facts concerning level orderings and expression
isomorphisms. First, every expression graph $G$ has a level ordering as
described in the algorithm in \S\ref{subsec:alg_graph_exp} so that the vertex
set is a disjoint union of the levels. Then, we have a level function $l_G$ for
$G$ which assigns to each vertex $v$ the integer $n$ for which $v$ is in the
$n$--th level set.

\begin{lem}\label{thm:expiso_lvlpres}
Expression isomorphisms $f : G \to H$ are level preserving, i.e.
\[
  l_G(v) = l_H(f(v))
\]
\end{lem}
\begin{proof}
Let
$\set{V_i}_{i = 1}^{N}$ and
$\set{W_j}_{i = 1}^{M}$ be the level sets of $G$ and $H$
respectively. Let $v \in V_k$ for some $1 \leq k \leq N$.
By corollary \ref{cor:lvltolvl}, there is a path $v_1, \dots, v_k = v$ with each
$v_i \in V_i$. Then, there is a path $f(v_1), \dots, f(v_k) = f(v)$ in $H$ with
$l_H(f(v_{i})) < l_H(f(v_{i + 1}))$ (since edges only go forward in levels) and
$l_H(f(v_1)) \geq 1$. Hence, by extending this path in $H$ backwards to $W_1$,
we have that $l_H(f(v)) \geq l_G(v)$.
On the other hand, let $f(v) \in W_m$ so that there is
a path $w_1, \dots, w_m$ in $H$ with each $w_i \in W_i$. Using $f^{-1}$ on this
path and a similar argument as the one before, we can show that
$l_H(f(v)) \leq l_G(v)$.
\end{proof}

\begin{cor}
For any expression isomorphism $f : G \to H$, if $V_i$ is the $i$--th level set
in $G$, then $f(V_i)$ is the $i$--th level set in $H$.
\end{cor}
\begin{proof}
Let $W_i$ be the $i$--th level set of $H$. Then, by the previous lemma
$f(V_i) \subset W_i$. Again, using $f^{-1}$ in place of $f$ and $W_i$ in place
of $V_i$, we have
$f^{-1}(W_i) \subset V_i \implies W_i \subset f(V_i)$. Hence, $f(V_i) = W_i$.
\end{proof}

\begin{lem}
Expression isomorphisms $f : G \to H$ are order-preserving on levels, i.e.
\[
  u \leq v \iff f(u) \leq f(g)
\]
\end{lem}
\begin{proof}
Let $u, v$ be in the same level set in $G$ and $u \leq v$. We proceed by
induction on $l_G(u) = l_G(v)$. When $l_G(u) = l_G(v) = 1$, then
$f(u) \leq f(v)$, by definition. The other direction is obtained similarly with
$f^{-1}$ in place of $f$.

Let $l_G(u) = l_G(v) = k + 1$. By the construction of level sets and their
ordering given in the expression construction algorithm, there exist
$u', v' \in V(G)$ such that the following hold:
\begin{enmrt}
\li $l_G(u') = l_G(v') = k$
\li $u' \leq v'$
\li there are edges $(u', u)$ and $(v' v)$
\li there are no edges $(u'', u)$ or $(v'', v)$ with $u'' < u'$ or $v'' < u'$
\li if $u' = v'$, $(u', u) \leq (v', v)$
\end{enmrt}
By the previous corollary, $l_H(f(u')) = l_H(f(v')) = k$.
If there is an $x < f(u')$ or a $y < f(v')$ in $V(H)$ with edges
$(x, f(u'))$ or $(y, f(v'))$, then by induction, $f^{-1}(x) < u'$ or
$f^{-1}(y) < v'$ with some edge $(f^{-1}(x), u)$ or $(f^{-1}(y), v)$,
contradicting the conditions on $u'$ and $v'$. Thus, the edges $(f(u'), f(u))$
and $(f(v'), f(v))$ ensure that $f(u) \leq f(v)$. The other direction is again
obtained similarly by replacing $f$ with $f^{-1}$.
\end{proof}

\begin{cor}\label{cor:expiso_unique}
Expression isomorphisms $f : G \to H$ are unique.
\end{cor}
\begin{proof}
Let $g : G \to H$ be another expression isomorphism. Then both $f$ and $g$
restrict to order-preserving bijections on the finite level sets and hence must
agree on the level sets. Thus, $f$ and $g$ agree on $G$.
\end{proof}

In light of the last corollary, it is reasonable to consider expression graphs
up to expression isomorphisms from this point onwards. We will next define some
useful constructs on expression graphs that will facilitate our desired notion
of TQFTs.

