
\subsection{Gauge Transformations}

Suppose we have $\pi_E : E \to M$ and $\nabla$ as before as well as another
bundle $\pi_{E'} : E \to M$ with a bundle isomorphism $f = (u, v) : E \to E'$ --
a pair of maps $u : M \to M'$ and $v : E \to E'$ with $v$ linear on each fibre
of $E$, making the following diagram commute:
\[\begin{tikzcd}
E \ar[d, "\pi_E" left] \ar[r, "v" above] & E' \ar[d, "\pi_{E'}" right]\\
M \ar[r, "u" below] & M'
\end{tikzcd}\]

Let $s : M' \to E' \in \Gamma(E')$. Then we have a section
$\wh{f}(s) \in \Gamma(E)$ defined by
\[
  \wh{f}(s) = v^{-1} \circ s \circ u
\]
Noting that $u : M \to M'$ is a diffeomorphism, it is easy to verify $du$
is also a bundle isomorphism, from the definition of differentials. Furthermore,
there is a bundle isomorphism $d^*u : T^*M \to T^*M'$ corresponding to $du$,
defined, for each $g : T_xM \to \R \in T^*_xM$, by the composite
\[
  (d^*u)(g) := T_{u(x)}M' \to[(du)^{-1}] T_xM \to[g] \R \in T^*_{u(x)}M'
\]

Denoting $\tilde{f}(x \tensor g) := v(x) \tensor (d^*u)(g)$, we then define:
\[\begin{array}{ccccc}
f \diamond \nabla
&:& \Gamma(E') &\to    & \Gamma(E' \tensor T^*M') \\
&:& s &\mapsto& \tilde{f} \circ \nabla(\wh{f}(s)) \circ u^{-1} \\
&&& = &
  \tilde{f} \circ \nabla(v^{-1} \circ s \circ u) \circ u^{-1}
\end{array}\]
We wish to show that $f \diamond \nabla$ is a connection.
Let $c \in \K$. Then, for a section $s \in \Gamma(E')$, have:
\begin{align*}
   & (f \diamond \nabla)(c \cdot s) \\
  =& \tilde{f}\nabla(v^{-1}(c \cdot s)u)u^{-1} \\
  =& \tilde{f}\nabla(c \cdot v^{-1}su)u{-1}
      && \text{fibre-wise linearity of } v^{-1} \\
  =& \tilde{f} (c \cdot  \nabla(v^{-1}su)u{-1})
      && \text{linearity of } \nabla \\
  =& c \cdot \tilde{f}\nabla(v^{-1}su)u^{-1}
      && \text{fibre-wise linearity of } \tilde{f} \\
  =& c \cdot (f \diamond \nabla)(s)
\end{align*}
We also observe that, for sections $s_1, s_2 \in \Gamma(E')$, we have
\begin{align*}
   & (f \diamond \nabla)(s_1 + s_2)\\
  =& \tilde{f}\nabla(v^{-1}(s_1 + s_2)u)u^{-1} \\
  =& \tilde{f}\nabla(v^{-1}(s_1u + s_2u))u^{-1}
    && \text{definition of pointwise addition} \\
  =& \tilde{f}\nabla(v^{-1}s_1u + v^{-1}s_2u)u^{-1}
    && \text{fibre-wise linearity of } v^{-1} \\
  =& \tilde{f}(\nabla(v^{-1}s_1u)
    + \nabla(v^{-1}s_2u))u^{-1}
    && \text{linearity of } \nabla \\
  =& \tilde{f}(\nabla(v^{-1}s_1u)u^{-1}
    + \nabla(v^{-1}s_2u)u^{-1})
    && \text{definition of pointwise addition} \\
  =& \tilde{f}\nabla(v^{-1}s_1u)u^{-1}
    + \tilde{f}\nabla(v^{-1}s_2u)u^{-1}
    && \text{fibre-wise linearity of } \tilde{f} \\
  =& (f \diamond \nabla)(s_1) + (f \diamond \nabla)(s_2)
\end{align*}
Thus, $f \diamond \nabla$ is $\K$--linear. Now, for $r \in \Cinf(M', \K)$, we
again obersve:
\begin{align*}
   & (f \diamond \nabla)(r \cdot s) \\
  =& \tilde{f}\nabla(v^{-1}(r \cdot s)u)u^{-1} \\
  =& \tilde{f}\nabla(v^{-1}(ru \cdot su))u^{-1}
    && \text{pointwise multiplication} \\
  =& \tilde{f}\nabla(ru \cdot v^{-1}su)u^{-1}
    && \text{fibre-wise linearity of } v^{-1} \\
  =& \tilde{f}(ru \cdot \nabla(v^{-1}su) + (v^{-1}su) \tensor d(ru))u^{-1}
    && \text{Leibniz property} \\
  =& \tilde{f}(ru \cdot \nabla(v^{-1}su))u^{-1}
      + \tilde{f}((v^{-1}su) \tensor d(ru))u^{-1}
    && \text{distribute over $+$ as before} \\
  =& \tilde{f}(ruu^{-1} \cdot \nabla(v^{-1}su)u^{-1})
      + \tilde{f}((v^{-1}su) \tensor d(ru))u^{-1}
    && \text{pointwise multiplication} \\
  =& r \cdot \tilde{f}\nabla(v^{-1}su)u^{-1}
      + \tilde{f}((v^{-1}su) \tensor d(ru))u^{-1}
    && \text{fibre-wise linearity of } \tilde{f} \\
  =& r \cdot (f \diamond \nabla)(s)
      + \tilde{f}((v^{-1}su) \tensor d(ru))u^{-1} \\
  =& r \cdot (f \diamond \nabla)(s)
      + (v \tensor (d^*u))(v^{-1}su \tensor d(ru))u^{-1} \\
  =& r \cdot (f \diamond \nabla)(s)
      + (v \tensor (d^*u))(v^{-1}suu^{-1} \tensor d(ru)u^{-1})
    && \text{by definition \eqref{eqn:conn_tensor}} \\
  =& r \cdot (f \diamond \nabla)(s)
      + (v \tensor (d^*u))(v^{-1}s \tensor d(ru)u^{-1}) \\
  =& r \cdot (f \diamond \nabla)(s)
      + s \tensor (d^*u)d(ru)u^{-1}
    && \text{$\tensor$ for sections}
\end{align*}

It now suffices to show that $(d^*u)d(ru)u^{-1} = dr$. Pointwise, we have:
\[
  ((d^*u) \circ d(ru) \circ u^{-1})(x) = d(ru)|_{T_{u^{-1}(x)}M} \circ (du)^{-1}
\]
We then observe a useful property of the derivative
operator $d-$. If $a : L \to M$ and $b : M \to N$ are smooth maps, then it is
easy to verify, from the definition of the differential, that
\[
  d(b \circ a) = db \circ da
\]
Then, we observe that
\[
  d(ru)|_{T_{u^{-1}(x)}M} \circ (du)^{-1}
  = dr|_{T_xM'} \circ du|_{T_{u^{-1}(x)}M} \circ (du)^{-1}
  = dr|_{T_xM'}
  = dr(x)
\]
as required. This completes the proof of the following theorem.

\begin{thm}
Let $\pi : E \to M$, $\pi' : E' \to M'$ be bundles with a bundle isomorphism
$f = (u, v) : E \to E'$. If $\nabla$ is a connection on $\pi$, then
$f \diamond \nabla$ is a connection on $\pi'$.
\end{thm}

The following theorem shows that the operation $- \diamond -$ commutes with
composition of diffeomorphisms. This result ultimately provides a notion of
morphism of connections from bundle isomorphisms.
\begin{thm}
Let $\pi_i : E_i \to M_i$ be bundles
for $i \in \set{1, 2, 3}$ along
bundle isomorphisms
$f_{j, j + 1} = (u_{j, j+ 1}, v_{j, j + 1}) : \pi_{j} \to \pi_{j + 1}$
for $j \in \set{1, 2}$. Then, if $\nabla$ is a connection on $\pi_1$, we have:
\[
  (f_{2, 3}f_{1, 2}) \diamond \nabla
  = f_{2, 3} \diamond (f_{1, 2} \diamond \nabla)
\]
\end{thm}
\begin{proof}
By expanding expressions, we obtain:
\begin{align*}
((f_{2, 3}f_{1, 2}) \diamond \nabla)(s)
=& (v_{2,3}v_{1,2} \tensor d^*(u_{2,3}u_{1,2}))
   \nabla((v_{2, 3}v_{1, 2})^{-1}s(u_{2, 3}u_{1,3}))
   (u_{2,3}u_{1,2})^{-1}
\end{align*}
We observe that for any suitable maps $p, q, w$, we have:
\[
  d^*(pq)(w) = w \circ (d(pq))^{-1}
  = w \circ (dq)^{-1} \circ (dp)^{-1}
  = d^*p(w \circ (dq)^{-1})
  = d^*p(d^*q(w))
\]
so that the first expression becomes:
\begin{align*}
((f_{2, 3}f_{1, 2}) \diamond \nabla)(s)
=& (v_{2,3}v_{1,2} \tensor d^*u_{2,3}d^*u_{1,2})
   \nabla((v_{2, 3}v_{1, 2})^{-1}s(u_{2, 3}u_{1,3}))
   (u_{2,3}u_{1,2})^{-1} \\
=& (v_{2,3} \tensor d^*u_{2,3})\sbr{(v_{1,2} \tensor d^*u_{1,2})
   \nabla(v_{1, 2}^{-1}(v_{2, 3}^{-1}su_{2, 3})u_{1,2})
   u_{1,2}^{-1}}u_{2,3}^{-1} \\
=& (v_{2,3} \tensor d^*u_{2,3})
   (f_{1,2} \diamond \nabla)(v_{2, 3}^{-1}su_{2, 3})
   u_{2,3}^{-1} \\
=& (f_{2,3} \diamond (f_{1,2} \diamond \nabla))(s)
\end{align*}
as required.
\end{proof}

From this point, we will write $f\nabla$ as opposed to $f \diamond \nabla$,
as long as the action is clear from context. Now, Let $\pi_i : E_i \to M_i$ be
bundles equipped with connections
$\nabla_i$ for $i \in \set{1, 2, 3, 4}$. Let
$f_{i, i + 1} = (u_{i, i + 1}, v_{i, i + 1}) : \pi_{i} \to \pi_{i + 1}$
be bundle isomorphisms for $i \in \set{1, 2, 3}$. Then, the composite
$f_{1, 3} := f_{2, 3}f_{1, 2} = (u_{2, 3}u_{1, 2}, v_{2, 3}v_{1, 2})
=: (u_{1, 3}, v_{1, 3})$ is clearly a bundle isomorphism. We similarly define
composites $f_{i, j}$ for each $i < j \in \set{1, 2, 3, 4}$. Now, suppose
$\nabla_{i + 1} = f_{i, i + 1}\nabla_i$ for each $i \in \set{1, 2, 3}$. Then, we
immediately have, from the previous theorem:
\[
  f_{i, k}\nabla_i = f_{j, k}f_{i, j}\nabla_i = f_{j, k}\nabla_j = \nabla_k
\]
for each $i < j < k$ in $\set{1, 2, 3, 4}$. In particular,
\[
  \nabla_4
    = f_{3, 4}(f_{2, 3}f_{1, 2}\nabla_1)
    = (f_{3,4}f_{2,3}f_{1,2}) \nabla_1
    = (f_{3, 4}f_{2, 3})f_{1, 2}\nabla_1
\]

We then observe the action of identity bundle morphisms. The identity bundle
morphism on $\pi_1$ is the pair $\id_{\pi_1} = (\id_{E_1}, \id_{M_1})$. Then,
\[
  \id_{\pi_1}\nabla_1(s)
    = (\id \tensor \id) \nabla_1(\id \circ s \circ \id) \id
    = \nabla_1(s)
\]
so that $\id_{\pi_1}\nabla = \nabla$. We finally observe that
$f(f^{-1}\nabla) = (ff^{-1})\nabla = \id\nabla = \nabla$ for any connection
$\nabla$ and any compatible bundle morphism $f$.

These observations motivate the following definition.
\begin{defn}[Gauge Transformation]
Let $\pi_1$ and $\pi_2$ be bundles equipped with connections $\nabla_1$ and
$\nabla_2$ respectively. Then, a bundle isomorphism
$f = (u, v) : \pi_1 \to \pi_2$ satisfying $f\nabla_1 = \nabla_2$ is called an
isomorphism, or simply morphism, of connections or a gauge transformation.
\end{defn}

From the work above, we have established the following results.
\begin{thm}
There exists a groupoid whose objects are connections and
whose morphisms are gauge transformations or isomorphisms of connections.
\end{thm}
\begin{defn}[Category of Connections]
We will call the category of the above theorem the category or groupoid of
connections. We will denote this category $\Conn$.
\end{defn}

We will see that connections on bundles on compact manifolds with boundary, when
specialized slightly, assemble into a double categorical structure compatible
with that of the cobordism double category formed by the underlying manifolds
over which we take the bundles. Before we proceed to this result, we will
require a notion of cobordism double category for bundles, which we develop
next.

