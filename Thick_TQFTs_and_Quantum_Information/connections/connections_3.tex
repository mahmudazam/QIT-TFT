
\subsection{Monoidal Double Category of Connections}

For $i \in \set{1, 2, 3}$ and connections $\nabla_i$ on smooth bundles
$\pi_i : E_i \to M_i$, we define a function
\[
  \nabla_1 \amalg \nabla_2
    : \Gamma(E_1 \amalg E_2)
    \to \Gamma((E_1 \tensor T^*M_1) \amalg (E_2 \tensor T^*M_2))
\]
as follows, for $j \in \set{0, 1}$:
\[\begin{array}{crcl}
       & (\nabla_1 \amalg \nabla_2)(s_1 \amalg s_2)(x, j)
       & = & \nabla_{j + 1}(s_{j + 1})(x) \\
  \iff & (\nabla_1 \amalg \nabla_2)(s_1 \amalg s_2)
       & = & \nabla_1(s_1) \amalg \nabla_2(s_2)
\end{array}\]
It is easy to see that this function satisfies the connection identities
piecewise so that it satisfies these identities on its entire domain. Thus,
$\nabla_1 \amalg \nabla_2$ is a connection. Letting
$f = (u, v) = (\alpha_{M_1, M_2, M_3}, \alpha_{E_1, E_2, E_3})$, we now wish to
verify that
\begin{equation}\label{eqn:conn_assoc}
  f \diamond (\nabla_1 \amalg (\nabla_2 \amalg \nabla_3))
    = (\nabla_1 \amalg \nabla_2) \amalg \nabla_3
\end{equation}
For this, we will need to inspect the expression:
\begin{align*}
   & f \diamond (\nabla_1 \amalg (\nabla_2 \amalg \nabla_2))(
        (s_1 \amalg s_2) \amalg s_3
     ) \\
  =& \tilde{f}(\nabla_1 \amalg (\nabla_2 \amalg \nabla_2))(
      v^{-1}((s_1 \amalg s_2) \amalg s_3)u
     )u^{-1} \\
  =& \tilde{f}(\nabla_1 \amalg (\nabla_2 \amalg \nabla_2))(
      s_1 \amalg (s_2 \amalg s_3)
     )u^{-1} \\
  =& (v \tensor d^*u)(\nabla_1 \amalg (\nabla_2 \amalg \nabla_2))(
      s_1 \amalg (s_2 \amalg s_3)
     )u^{-1} \\
  =& (v \tensor d^*u)(
      \nabla_1(s_1) \amalg (\nabla_2(s_2) \amalg \nabla_3(s_3))
     )u^{-1}
\end{align*}
To reach our goal \eqref{eqn:conn_assoc}, we observe the following basic fact.
\begin{lem}
For tangent bundles $\pi_i : TM_i \to M_i$, $i \in \set{1, 2, 3}$, we have
$d\alpha_{M_1, M_2, M_3} = \alpha_{TM_1, TM_2, TM_3}$.
\end{lem}
\begin{proof}
We denote $\alpha_{TM} := \alpha_{TM_1, TM_2, TM_3}$,
$\alpha_{M} := \alpha_{M_1, M_2, M_3}$.
Let $R$ be a manifold, and $x$ and $y$, smooth maps making the following
diagram commute:
\[\begin{tikzpicture}[baseline=(a).base]
\node[scale=\diagscale] (a) at (0, 0){
\begin{tikzcd}[column sep=large, row sep=huge]
&
(TM_1 \amalg TM_2) \amalg TM_3
  \ar[d, "\pi_{(12)3}" description]
  \ar[rrd, "x" above right] & \\
TM_1 \amalg (TM_2 \amalg TM_3)
  \ar[ur, "\alpha_{TM}" above left]
  \ar[dr, "\pi_{1(23)}" below left] &
(M_1 \amalg M_2) \amalg M_3
  \ar[rr, "y\alpha_{M}^{-1}" description, dashed] & &
R \\ &
M_1 \amalg (M_2 \amalg M_3)
  \ar[u, "\alpha_{M}" description]
  \ar[urr, "y" below right] &
\end{tikzcd}
};
\end{tikzpicture}\]
The map $y\alpha_M^{-1}$ satisfies $(y\alpha_M^{-1})\alpha_M = y$ and
$(y\alpha_M^{-1})\pi_{(12)3} = y\pi_{1(23)}\alpha_{TM} = x$. Suppose another map
$r$ satisfies $r\alpha_M = y$. Then, $r = y\alpha_M^{-1}$ showing that
$y\alpha_M^{-1}$ is unique.

Therefore, $\alpha_{TM}$ makes the part of the diagram excluding $R$ a pushout
and hence must be $d\alpha_M$, by the uniqueness of the derivative.
\end{proof}

\begin{cor}
For tangent bundles $\pi_i$ as in the previous lemma, we have
\[
  (d^*\alpha_{M_1, M_2, M_3})(g) = g \circ \alpha_{TM_1, TM_2, TM_3}^{-1}
    = \br{\alpha^{-1}_{TM_1, TM_2, TM_3}}^*(g)
\]
\end{cor}

\begin{rmk}
In general, we denote precomposition by any map $\alpha$ as
$\alpha^*(g) = g \circ \alpha$.
\end{rmk}

The above corollary yields:
\begin{align*}
   & f \diamond (\nabla_1 \amalg (\nabla_2 \amalg \nabla_2))(
        (s_1 \amalg s_2) \amalg s_3
     ) \\
  =& (v \tensor d^*u)(
      \nabla_1(s_1) \amalg (\nabla_2(s_2) \amalg \nabla_3(s_3))
     )u^{-1} \\
  =& \br{\alpha_{M_1, M_2, M_3} \tensor \br{\alpha^{-1}_{TM_1, TM_2, TM_3}}^*}(
      \nabla_1(s_1) \amalg (\nabla_2(s_2) \amalg \nabla_3(s_3))
     )\alpha_{M_1, M_2, M_3}^{-1}
\end{align*}
where the last expression is easily seen to be
\[
  (\nabla_1(s) \amalg \nabla_2(s_2)) \amalg \nabla_3(s_3)
  = ((\nabla_1 \amalg \nabla_2) \amalg \nabla_3)((s_1 \amalg s_2) \amalg s_3)
\]
We can similarly show that the unitors in $\Man$ yield unitors for disjoint
unions of connections of bundles with equal fibres and equidimensional base
spaces. We have thus proved the following theorem.
\begin{thm}\label{thm:bundle_gluing}
For a vector space $V$ and a non-negative integer $d$, the subcategory of the
category of connections consisting of all connections on objects in
$\Bun_d^{V}$ and all morphisms of connections between them is a monoidal
category under disjoint union.
\end{thm}
\begin{defn}[Category of {$(V, d)$}--Connections]
We call the subcategory of the category of connections in the above theorem the
category of connections on $V$--fibred bundles on $d$--dimensional manifolds or
of $(V, d)$--connections. We denote this category $\Conn^V_d$.
\end{defn}

As we have a notion of gluing for (a subset of) complex bundles with a fixed
fibre, we will develop a notion of gluing for connections defined on these
bundles. It is well known that, for connections $\nabla$ and $\nabla'$ on a
smooth bundle $E \to M$, $\nabla - \nabla'$ is a smooth section of the bundle
$\Hom(E) \tensor T^*M \to M$, where $\Hom(E)$ is the internal $\Hom$-object in
the closed monoidal category of smooth bundles over $M$ that is isomorphic to
the bundle $E^* \tensor E$ \cite[Lemma 2.2]{Conn}. As a result, the set of all
connections on $E$ is an affine subspace of the $\Cinf(\R)$--module on $E$ is an
affine subspace of the $\Cinf(\R)$--module write each connection $\nabla$ on $E$
as a sum:
\[
  \nabla = \varphi + A_{\nabla}
\]
where $\varphi$ is a fixed element of $\Omega^1(\Hom(E))$ and $A_{\nabla}$
is an element of some subspace of $\Omega^1(\Hom(E))$ varying with $\nabla$. In
fact, every expression of this form is a connection. It is well known that
$\varphi$ can be taken as the exterior derivative operator $d$. In particular,
this allows us to define an action of $\Cinf(\R)$ on the set of connections as
follows:
\[
  r \cdot \nabla = d + r \cdot A_{\nabla} = d + A_{r \cdot \nabla}
\]
From this, a gluing operation for (a class of) connections is immediate. This
is made precise in the following theorem.

\begin{thm}
Let $E \to M$ and $E' \to M'$ be a horizontal $1$--morphism
(or bundle cobordism) in $\BBun^V_{\s{C}}$ for some cobordism category $\s{C}$,
such that the gluing $E' * E \to M' * M$. exists. Then, for any two connections
$\nabla$ and $\nabla'$ on $E$ and $E'$ respectively, there exists a connection
$\nabla' * \nabla$ that has the same output as $\nabla$ on local sections of $E$
defined away from its trivialized boundary collar and to $\nabla'$ for local
sections of $E'$ defined away from its trivialized boundary collar.
\end{thm}
\begin{proof}
Let the bump functions on $M$ and $M'$ used to trivialize $E$ and $E'$
respectively near boundaries, as defined in the proof of theorem
\ref{thm:bundle_gluing}, be $f$ and $f'$ respectively. Let
\[
  \nabla = d + A_{\nabla} \text{ and } \nabla' = d + A_{\nabla'}.
\]
We define:
\[
  \wh{\nabla} := f \cdot \nabla = d + f \cdot A_{\nabla}
  \text{ and } \wh{\nabla'} := f' \cdot \nabla' = d + f' \cdot A_{\nabla'}
\]
We can then extend the domain of $f$ to $M' * M$ by defining it to be zero on
$M'$. Similarly, we define $f'$ to be zero on $M$. This allows us to extend the
domain of $f \cdot A_{\nabla}$ and $f' \cdot A_{\nabla'}$ to $M' * M$ in the
same way. We can thus define:
\[
  \nabla' * \nabla := \wh{\nabla'} * \wh{\nabla}
    := d + f \cdot A_{\nabla} + f' \cdot A_{\nabla'}
\]
It is immediate that $\nabla' * \nabla$ satisfies the conditions of being a
connection pointwise. Hence, $\nabla' * \nabla$ is a connection on $E' * E$. We
also observe that where $f$ is $1$, $\nabla' * \nabla$ is equal to $\nabla$ and
likewise with $f', \nabla'$.
\end{proof}

\begin{defn}[Gluable Connection]
Connections of the form $\wh{\nabla}$ as in the previous theorem are called
gluable. The set of gluable connections on bundle cobordisms in
$\BBun^V_{\s{C}}$ is denoted $\Conn^V_{\s{C}}$.
\end{defn}

\begin{cor}
Gluing of connections in double category of bundle cobordisms is associative and
unital upto gauge isomorphisms.
\end{cor}
\begin{proof}
For associativity, it suffices to observe that extending domains of bump
functions by zeros is associative and that associators from $\BBun^V_{\s{C}}$
are associator gauge transformations -- the proof of the latter claim is similar
to that of the associativity of the disjoint union of connections.

For unitality, we first observe that on boundary collars $f \cdot \nabla$ is
equal to $d$ for all connections $\nabla = d + A_{\nabla}$. The exterior
derivative operators, that are themselves connections, on the gluing
units of $\BBun^V_{\s{C}}$ -- that is, the cylinders on boundaries -- suffices
as the gluing units for connections. The unitors carry over like associators.
\end{proof}

Finally, one verifies the following fundamental fact.

\begin{thm}
For any cobordism double category $\s{C}$, $\BBun^V_{\s{C}}$ can be promoted to
a monoidal double category of connections by taking pairs $(E, \nabla)$ for each
bundle cobordism $E$ in $\BBun^V_{\s{C}}$ and a gluable connection $\nabla$ on
$E$ as the horizontal $1$--morphisms. Vertical $1$--morphisms and $2$--morphisms
are taken to be gauge transformations. The rest of the structure is modified in
the obvious ways.
\end{thm}

\begin{defn}
We denote the monoidal double category of connections in the last theorem as
$\CConn^V_{\s{C}}$.
\end{defn}

At this point, one might think that we can immediately define a notion of TQFT
by choosing suitable paths for a representative for each cobordism class,
taking the linear maps obtained by parallel transport along these paths and
combine in a suitable way. However, the problem remains that two gauge
transformations need not take a collection of paths in the domain to the same
collection of paths in the codomain. We will build the machinery to handle this
matter in the next section.

