
\documentclass[./Thick_TQFTs_and_Quantum_Information.tex]{subfiles}

\begin{document}

\section{Parallel Transport}

Consider the monoidal double (or bi-) category of thick tangles, $\DThick$. We
give a concrete method to define a functor of monoidal double (or bi-)
categories from $\DThick$ to some suitable monoidal double (or bi-) category of
complex vector spaces. Note that we do not yet define the codomain of this
under-construction functor. We will define the codomain so as to make this
functor sensible.

The object function $F_0$ of a functor is given by assigning to each disjoint
union $X = I^{\amalg n}$ the $n$--th tensor power $F_0(X) = A^{\tensor n}$ of
some fixed algebra $A$ -- we may be specific enough to say that the bracketing
pattern of the image is the same as the preimage. It is easy to see that this
assignment is well-defined.

Next, we consider the horizontal $1$--morphisms or the cobordisms. Cobordisms
$Y : I^{\amalg n} \to I^{\amalg m}$ in $\DThick$, for positive $m$ and $n$, are
surfaces with boundary $I^{\amalg n} \amalg I^{\amalg m}$ and are hence
determined up to diffeomorphism by genus. We first consider some ``canonical''
genus $k$ cobordism $Y : I \to I$ where the holes are circles with centers along
a straight line from one boundary interval to another. This cobordism then
decomposes into a $k$--fold composition of $M * W : I \to I$ with itself where
$M$ is the ``canonical'' pair-of-pants and $W$ is the canonical
co-pair-of-pants. In associating a linear map ``functorially'' to $Y$, it
suffices to associate linear maps to $M$ -- we can then associate a linear map
to $W$ by duality and get a linear map for $Y$ by composition.

For associating a linear map to $M$, we take the trivial bundle
$\mathcal{O} : M \times A \to M$ with a connection $\nabla$. Then, we choose two
paths: one from the mid-point of each in-boundary interval to the mid-point of
the out-boudnary interval. By parallel transport along each curve, we get two
linear maps $l_1, l_2 : A \to A$. We then have a linear map
$l : A \tensor A \to A$ given by $l(x \tensor y) = l_1(x)l_2(y)$ where the
product in the right is the algebra product in $A$. Then, the linear map
associated to $W$ is the conjugate transpose $l^{\dagger}$.

We then obtain a linear map associated to $Y$: the $k$--fold composite
$(ll^{\dagger})^k$. We set $F_1(Y) := (ll^{\dagger})^k$. Now, consider a
cobordism $Z : I^{\amalg n} \to I^{\amalg m}$ of genus $k$ such that
$Z = RYQ$, where $Q$ is a cobordism $I^{\amalg n} \to I$ formed in some
``canonical'' way by a gluing of pairs of pants and cylinders and $R$ is a
cobordism $I \to I^{\amalg m}$ formed again in a ``canonical'' way from
co-pairs pants and cylinders. Of course, we associate a linear map to the
cylinder by another parallel transport along a curve between the mid-points of
its two boundaries. Then, again by composition, we get a linear map
$F_1 : A^{\tensor n} \to A^{\tensor m}$.

This gives an assignment $F_1(Z)$ for a representative $Z$ of each
diffeomorphism class of cobordisms in $\DThick$. For the assignment of a linear
map to every cobordism in $\DThick$, we take the following approach. Let $Z'$
be an arbitrary cobordism in the class of some $Z$ for which $F_1(Z)$ has been
defined as above. Then we pick a boundary preserving diffeomorphism
$f \in \Aut_{\Man}(Z)$ such that $f(Z) = Z'$\footnote{Of course, we take the
underlying topological space for each manifold in a diffeomorphism class to be
the same}. Let $\set{\gamma_i}$ be the family of curves which along which
parallel transport gave us the linear maps $F_1(Z)$. Then,
$\set{f \circ \gamma_i}$ is a family of curves in $Z'$ with the following nice
propeties.
\begin{thm}
If $\gamma_i$ and $\gamma_j$ share an end-point, then so do $f\gamma_i$ and
$f\gamma_j$.
\end{thm}
\begin{proof}
{\color{blue!55!black} TODO}
\end{proof}

\begin{thm}
The parallel transport ``assembly'' developed above is respected.
{\color{blue!55!black} TODO: Need to make this precise. Roughly, we need to
remember each $\gamma_i$, how it glues to the others and when to multiply or
take tensor products.}
\end{thm}

We now turn our attention to the case when $m$ or $n$ is zero. We can treat the
case $n = 0$ and obtain the other case by duality. Let
$Z : \varnothing \to I^{\amalg m}$ with genus $k$. Then $Z$ decomposes as
$Y * Z'$ where $Z'$ is a genus zero cobordism $\varnothing \to I$ which we will
call an ``atomic element'' cobordism. We call $Z$ an element cobordism and hence
$Z'$ is deemed atomic because we will not decompose it further. $Z'$ has a
boundary preserving diffeomorphism $f : Z \to Z'$ for some atomic
element $Z''$ deemed ``canonical''. We associate a linear map $a : \C \to A$ to
$Z''$ as follows. Take a loop $\gamma$ in $Z''$ on the mid-point of its only
boundary interval and obtain an element $a \in A$, or equivalently linear map
$a : \C \to A$ by parallel transport of some fixed element $a_0 \in A$. We set
$F_1(Z'') = a$ and get $F_1(Z')$ by parallel transport of $a_0$ along $f\gamma$.
$F_1(Z)$ is then obtained by composition.

This completes the definition of an object function $F_1$ for the
morphism category of $\DThick$. Now, we turn our attention to $2$--morphisms --
boundary preserving diffeomorphisms between cobordisms. Let $f : Z \to Z'$ be
one such diffeomorphism. We must assign to $f$ some object that functions as
roughly a ``morphism of morphisms of vector spaces''. We have a few ways ways
of achieving this from which we will ultimately have to pick one to formulate
the codomain double (or bi-) category -- $F_1(f) : F_1(Z) \to F_1(Z')$ could be:
\begin{enumerate}[(i)]

\item a function
\[
\Hom{\dom F_1(Z), \codom F_1(Z)} \to \Hom{\dom F_1(Z), \codom F_1(Z)}
\]
such that $F_1(f)(F_1(Z)) = F_1(Z')$.

\item a homotopy class of a path in $\Hom{\dom F_1(Z), \codom F_1(Z)}$ from
$F_1(Z)$ to $F_1(Z')$

\item a pair of mappings $f_d : \dom F_1(Z) \to \dom F_1(Z')$ and
$f_c : \codom F_1(Z) \to \codom F_1(Z')$ such that the following square commutes:
\begin{equation}\label{diag:eqcob}
\begin{tikzpicture}[baseline=(a).base]
\node[scale=\diagscale] (a) at (0, 0){
\begin{tikzcd}
\dom F_1(Z) \arrow[r, "F_1(Z)" above] \arrow[d, "f_d" left] &
\codom F_1(Z) \arrow[d, "f_c" right] \\
\dom F_1(Z') \arrow[r, "F_1(Z')" below] & \codom F_1(Z')
\end{tikzcd}
};
\end{tikzpicture}
\end{equation}
\end{enumerate}

\end{document}

