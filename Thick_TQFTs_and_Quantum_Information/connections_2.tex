
\subsection{Bundle Cobordisms}\label{subsec:bund_cob}

Consider smooth bundles $\pi_1 : E_1 \to M_1$ and $\pi_2 : E_2 \to M_2$. We will
consider the coproduct or disjoint union of these bundles in the category of
manifolds. There exists a smooth map
$\pi_1 \amalg \pi_2 : E_1 \amalg E_2 \to M_1 \amalg M_2$ which we will give the
structure of a vector bundle as follows. For this, we additionally assume that
the fibres of $E_1$ and $E_2$ are the same vector space. Let
$U = U_1 \amalg U_2, V = V_1 \amalg V_2$ be open sets in $M_1 \amalg M_2$ with
$U_i, V_i \subset M_i$ open for $i \in \set{1, 2}$, and consider
$(U_1 \amalg U_2) \cap (V_1 \amalg V_2) = (U_1 \cap V_1) \amalg (U_2 \cap V_2)$.
We have a transition function $G_{U_1, V_1}$ on $U_1 \cap V_1$ from the bundle
$\pi_1$ and one $H_{U_2, V_2}$ on $U_2 \cap V_2$ from $\pi_2$. We define a
function $(G \amalg H)_{U, V} : U \cap V \to \GL_n(\C)$ piecewise, as
follows:
\[
  (G \amalg H)_{U, V}(x) := \begin{cases}
    G_{U_1, V_1}(x), & x \in U_1 \cap V_1 \subset M_1 \\
    H_{U_2, V_2}(x), & x \in U_2 \cap V_2 \subset M_2
  \end{cases}
\]
which is smooth since it is a disjoint union of smooth functions. Therefore,
\[
  G \amalg H := \set[(G \amalg H)_{U, V}]
                    {U, V \subset M_1 \amalg M_2 \text{ are open}}
\]
is a vector bundle structure on $\pi_1 \amalg \pi_2$. A section of
$E_1 \amalg E_2$ is a smooth map
\[
  s : M_1 \amalg M_2 \to E_1 \amalg E_2
\]
satisfying $(\pi_1 \amalg \pi_2)s = \id_{M_1 \amalg M_2}$. We note that this
guarantees that the $s$ must be of the form $s_1 \amalg s_2$ where $s_i$ is a
section of $E_i$, $i \in \set{1, 2}$.

Similarly, $TM_1 \amalg TM_2 \to M_1 \amalg M_2$ is a vector bundle when $M_1$
and $M_2$ have the same dimension, and we can take this to be the definition of
the tangent bundle $T(M_1 \amalg M_2)$ on $M_1 \amalg M_2$. Now, let
$\pi_3 : E_3 \to M_3$ be another bundle where all the $E_i$ have the same fibres
and all the $M_i$ are equidimensional.

We can pick a convention for disjoint unions of sets as follows:
\[
  A \amalg B = (A \times \set{0}) \cup (B \times \set{1})
\]
Under this convention,
\[
  E_1 \amalg (E_2 \amalg E_3)
    = \set[(x_1, 0)]{x_1 \in E_1}
      \cup \set[((x_2, 0), 1)]{x_2 \in E_2}
      \cup \set[((x_3, 1), 1)]{x_3 \in E_3}
\]
and
\[
  (E_1 \amalg E_2) \amalg E_3
    = \set[((x_1, 0), 0)]{x_1 \in E_1}
      \cup \set[((x_2, 1), 0)]{x_2 \in E_2}
      \cup \set[(x_3, 1)]{x_3 \in E_3}
\]
We have similar descriptions for the two distinct parenthesizations for
$M_1 \amalg M_2 \amalg M_3$. Now, the map
\[
  \alpha_{E_1, E_2, E_3} : E_1 \amalg (E_2 \amalg E_3)
                           \to (E_1 \amalg E_2) \amalg E_3
\]
defined by
\[
  (x_1, 0) \mapsto ((x_1, 0), 0),
  ((x_2, 0), 1) \mapsto ((x_2, 1), 0),
  ((x_3, 1), 1) \mapsto (x_3, 1)
\]
is easily seen to be bijective and fibre-preserving. Smoothness and naturality
in the subscripts follow from those of associators in $\Man$. We can make a
similar argument for similarly defined unitors $\rho_E$ and $\lambda_E$. We thus
have the following theorem.
\begin{thm}
The subcategory of the category of bundles consisting of bundles with base
spaces of a fixed dimension $d$ and total spaces with isomorphic fibres is
monoidal under the disjoint union of manifolds.
\end{thm}
\begin{defn}[Category of {$(V, d)$--bundles}]
The subcategory of the category of bundles in the above theorem is called the
category of $V$--fibred bundles on $d$--dimensional manifolds or of
$(V, d)$--bundles and is denoted $\Bun^V_d$.
\end{defn}

We will now develop a notion of gluing complex bundles on compact manifolds with
boundary along with connections on these bundles. To accomplish this, we will
first show the following:
\begin{lem}\label{thm:bundle_gluing}
Let $M$ be a smooth compact manifold such that $\partial M$ has
a collar $C_0$ whose connected components are each contractible. For any complex
vector bundle $\pi : E \to M$ with fibre $P$,
there exists a complex bundle $\wh{\pi} : \wh{E} \to M$ which restricts to
the trivial bundle on a collar $C \subset C_0$ of $\partial M$ and to $E$
on $M \setminus C_0$.
\end{lem}
\begin{proof}
Let $U$ and $V$ be any two open sets of $M$ over which $E$ trivializes and
$G_{U, V} : U \cap V \to \Aut(P)$, the assignment of transition functions to
their intersection.
By the smooth collar theorem, there exists a nieghbourhood $C_0$ of $\partial M$
diffeomorphic to the cylinder $\partial M \times I$ on $\partial M$,
with $\partial M$ identified with $\partial M \times \set{1}$. By hypothesis, we
can take $C_0$ to have contractible components such that $E|_{C_0}$ is
isomorphic to the trivial bundle. Therefore, $G_{U, V}|_{C_0}$ is smoothly
homotopic to the constant map $U \cap V \cap C_0 \to \Aut(P) : x \mapsto \id_P$.
Let $H_{U, V} : I \times U \cap V \cap C_0 \to \Aut(P)$ be this homotopy so that
$H_{U, V}(1, -) = G_{U, V}|_{C_0}$ and $H_{U, V}(0, x) = \id_P$ for all
$x \in U \cap V \cap C_0$.

We can then cut $C_0$ into pieces
$C' \cong \partial M \times \sbr{0, \frac{1}{2}}$ and
$C \cong \partial M \times \sbr{\frac{1}{2}, 1}$ that are each diffeomorphic to
$\partial M \times I$. There exists a smooth bump function $f : M \to \R$ such
that $f$ is $1$ on $M \setminus C_0$, decreases to $0$ on $C'$ as we move
towards $\partial M \times \set{\frac{1}{2}}$ and vanishes on $C$:
\[
  f(x) = \begin{cases}
    1 & x \in M \setminus C_0 \\
    \frac{1}{2}(1 - \text{erf}(at + b))
      & x = (x', t) \in C', x' \in \partial M, t \in \sbr{0, \frac{1}{2}} \\
    0 & x \in C
  \end{cases}
\]
where $a$ and $b$ are appropriately chosen constants.

We then take the bundle $\wh{E} \to M$ with the same trivializations as $E$
and transition functions
\[
  K_{U, V}(x) = \begin{cases}
    H_{U, V}(f(x), x) & x \in C_0 \\
    G_{U, V}(x)       & x \in M \setminus C_0
  \end{cases}
\]
We observe that away from the collar $C_0$, the bundle is the same as $E$ and
inside $C$, it is trivial, as required.
\end{proof}

It is straightforward to verify that for any cospan $M \ot[f] X \to[g] N$ and
any finite dimensional vector space $V$ seen as an object in $\Man$, the
following holds:
\[
  V \times (M \amalg_X N) \cong (V \times M) \amalg_{X \times V} (V \times N)
\]
such that the isomorphism is fibre-preserving and linear on fibres. Hence,
trivial bundles always glue at boundaries to give trivial bundles. This
observation yields a gluing operation $- * -$ for the following collection of
complex bundles:
\[
  \set[\wh{E}]{E \text{ is a complex bundle with fibre } V}
\]
since the bundles $\wh{E}$ are trivial near their boundaries. We observe that
gluing fibres at the boundaries is associative up to diffeomorphism by the same
argument for the associativity of gluing manifolds along boundaries. It is also
not hard to verify that the associator diffeomorphisms are fibre-preserving and
linear on the fibres. Furthermore, given a bundle $\wh{E} \to M$ where
$\partial M = W_0 \amalg W_1$, we take the trivial bundles
$W_0 \times I \times V \to W_0 \times I$ and
$W_1 \times I \times V \to W_1 \times I$, and observe that they act as gluing
identities for $\wh{E}$ on either side by a simple reparametrization. This
establishes a notion of cobordism of bundles. That is,
\begin{thm}
Given a double category of cobordisms $\s{C}$ (e.g. $\CCob_d$ or $\DThick$)
and a complex vector space $V$, we have a double category $\BBun^V_{\s{C}}$
consisting of the following data:
\begin{enmrt}
\li Object category: objects are trivial $V$--bundles on the objects of $\s{C}$
and morphisms are bundle isomorphisms
\li Morphism category: objects are complex bundles $\wh{E} \to M$, for $M$ in
the morphism category of $\s{C}$ and complex bundles $E \to M$; morphisms are
bundle isomorphisms
\li Source functor: each bundle $\wh{E} \to M$ is sent to the trivial bundle on
the source of $M$; action on morphisms is by restriction to appropriate boundary
components
\li Target functor: defined analogously as the source functor
\li Unit functors: each bundle $\wh{E} \to M$ is sent to the trivial bundle on
the cylinder on the appropriate boundary components
\li Horizontal composition: gluing corresponding fibres at common boundary
\li Horizontal composition associators: inherited from the category of manifolds
\li Horizontal composition unitors: inherited like the associators
\li Monoidal product: disjoint union
\li Monoidal unit(s): empty bundle(s)
\end{enmrt}
\end{thm}

We also notice that the above constructions apply to smooth (real) bundles as
long as the transition functions at the points in some collar of the boundary
can be connected to the identity function by paths in the automorphism group of
the fibre. This is possible if these transition functions all have positive
determinant. \TODO{Justify this, if needed.} One might guess that the
structure on the category of bundles developed here transfers over to the
category of connections on the bundles involved. We next show that this is
indeed the case.

