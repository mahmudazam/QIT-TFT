
\documentclass[./Thick_TQFTs_and_Quantum_Information.tex]{subfiles}

\begin{document}

\section{Category of Connections}

For a vector bundle $\pi_E : E \to M$, we write $\Gamma(E)$ to denote the set of
smooth sections of the bundle -- we will not be using the sheaf structure unless
necessary. For $\K = \R$ when $E$ is a smooth bundle with a real vector space as
fibres, or $\K = \C$ when $E$ is a complex bundle, we recall that a connection
\cite{Conn} on $E$ is a $\K$--linear map
\[
  \nabla : \Gamma(E) \to \Gamma(E \tensor T^*M)
\]
such that for all $s \in \Gamma(E)$ and $r \in \Cinf(M, \R)$, the following
Leibniz property is satisfied:
\[
  \nabla(r \cdot s) = r \cdot \nabla(s) + s \tensor dr
\]
\begin{rmk}
If $r : M \to \R$ is a smooth function, then the derivative of $r$ is a map
$dr : TM \to T\R$ such that $(r, dr)$ is a bundle morphism making the following
a pushout diagram in the category of manifolds:
\[\begin{tikzcd}
TM \ar[d, "\pi_{TM}" left] \ar[r, "dr" above] & T\R \ar[d, "\pi_{T\R}" right]\\
M \ar[r, "r" below] & \R
\end{tikzcd}\]
By the definition of bundle morphism, $dr$ is linear on fibres so that for each
$x \in M$, $dr$ restricts to a linear map $T_xM \to \R$ and these maps vary
smoothly with $x$, so that $dr$ is a section of $T^*M$.
\end{rmk}
We will now see that isomorphisms of vector bundles have an action on the
connections on these bundles, leading to a notion of morphism for connections --
a first step in developing a double category of
``parallel transport machinery''.

\subsection{Bundle Isomorphisms Transform Connections}

Suppose we have $\pi_E : E \to M$ and $\nabla$ as before as well as another
bundle $\pi_{E'} : E \to M$ with a bundle isomorphism $f = (u, v) : E \to E'$ --
a pair of maps $u : M \to M'$ and $v : E \to E'$ with $v$ linear on each fibres
of $E$, making the following diagram commute:
\[\begin{tikzcd}
E \ar[d, "\pi_E" left] \ar[r, "v" above] & E' \ar[d, "\pi_{E'}" right]\\
M \ar[r, "u" below] & M'
\end{tikzcd}\]

Let $s : M' \to E' \in \Gamma(E')$. Then we have a section
$\wh{f}(s) \in \Gamma(E)$ defined by
\[
  \wh{f}(s) = v^{-1} \circ s \circ u
\]
Noting that $u : M \to M'$ is a diffeomorphism, it is easy to verify that $du$
is also a bundle isomorphism, using the pushout property. By duality, there is
a bundle isomorphism $d^*u : T^*M' \to T^*M$ corresponding to $du$. Denoting
$\tilde{f} := v \tensor (d^*u)^{-1}$, we then define:
\[\begin{array}{ccccc}
f \diamond \nabla
&:& \Gamma(E') &\to    & \Gamma(E' \tensor T^*M') \\
&:& s &\mapsto& \tilde{f} \circ \nabla(\wh{f}(s)) \circ u^{-1} \\
&&& = &
  \tilde{f} \circ \nabla(v^{-1} \circ s \circ u) \circ u^{-1}
\end{array}\]
We wish to show that $f \diamond \nabla$ is a connection.

Let $c \in \K$. Then, for a section $s \in \Gamma(E')$, have:
\begin{align*}
  (f \diamond \nabla)(c \cdot s)
    &= \tilde{f}\nabla(v^{-1}(c \cdot s)u)u^{-1} \\
    &= \tilde{f}\nabla(c \cdot v^{-1}su)u{-1}
      && \text{fibre-wise linearity of } v^{-1} \\
    &= \tilde{f} (c \cdot  \nabla(v^{-1}su)u{-1})
      && \text{linearity of } \nabla \\
    &= c \cdot \tilde{f}\nabla(v^{-1}su)u^{-1}
      && \text{fibre-wise linearity of } \tilde{f} \\
    &= c \cdot (f \diamond \nabla)(s)
\end{align*}
We also observe that, for sections $s_1, s_2 \in \Gamma(E')$, we have
\begin{align*}
   & (f \diamond \nabla)(s_1 + s_2)\\
  =& \tilde{f}\nabla(v^{-1}(s_1 + s_2)u)u^{-1} \\
  =& \tilde{f}\nabla(v^{-1}(s_1u + s_2u))u^{-1}
    && \text{by the definition of pointwise addition} \\
  =& \tilde{f}\nabla(v^{-1}s_1u + v^{-1}s_2u))u^{-1}
    && \text{by the fibre-wise linearity of } v^{-1} \\
  =& \tilde{f}(\nabla(v^{-1}s_1u)
    + \nabla(v^{-1}s_2u))u^{-1}
    && \text{by the linearity of } \nabla \\
  =& \tilde{f}(\nabla(v^{-1}s_1u)u^{-1}
    + \nabla(v^{-1}s_2u)u^{-1})
    && \text{by the definition of pointwise addition} \\
  =& \tilde{f}\nabla(v^{-1}s_1u)u^{-1}
    + \tilde{f}\nabla(v^{-1}s_2u)u^{-1}
    && \text{by the fibre-wise linearity of } \tilde{f} \\
  =& (f \diamond \nabla)(s_1) + (f \diamond \nabla)(s_2)
\end{align*}
Thus, $f \diamond \nabla$ is $K$--linear. Now, for $r \in \Cinf(M', \R)$, we
again obersve:
\begin{align*}
   & (f \diamond \nabla)(r \cdot s) \\
  =& \tilde{f}\nabla(v^{-1}(r \cdot s)u)u^{-1} \\
  =& \tilde{f}\nabla(v^{-1}(ru \cdot su))u^{-1}
    && \text{pointwise multiplication} \\
  =& \tilde{f}\nabla(ru \cdot v^{-1}su)u^{-1}
    && \text{$K$--linearity of } v^{-1} \\
  =& \tilde{f}(ru \cdot \nabla(v^{-1}su) + (v^{-1}su) \tensor d(ru))u^{-1} \\
  =& \tilde{f}(ru \cdot \nabla(v^{-1}su))u^{-1}
      + \tilde{f}((v^{-1}su) \tensor d(ru))u^{-1}
    && \text{distribute over $+$ as before} \\
  =& \tilde{f}(ruu^{-1} \cdot \nabla(v^{-1}su)u^{-1})
      + \tilde{f}((v^{-1}su) \tensor d(ru))u^{-1}
    && \text{pointwise multiplication} \\
  =& r \cdot \tilde{f}\nabla(v^{-1}su)u^{-1}
      + \tilde{f}((v^{-1}su) \tensor d(ru))u^{-1}
    && \text{fibre-wise linearity of } \tilde{f} \\
  =& r \cdot (f \diamond \nabla)(s)
      + \tilde{f}((v^{-1}su) \tensor d(ru))u^{-1} \\
  =& r \cdot (f \diamond \nabla)(s)
      + (v \tensor (d^*u)^{-1})(v^{-1}su \tensor d(ru))u^{-1} \\
  =& r \cdot (f \diamond \nabla)(s)
      + s \tensor (d^*u)^{-1}d(ru)u^{-1}
\end{align*}

It now suffices to show that $(d^*u)^{-1}d(ru)u^{-1} = dr$ -- this is a
straightforward diagram chase. \TODO{Actually, check it!}


Now, let $X, Y \in \Gamma(TM')$, $s \in \Gamma(E')$, $r \in \Cinf(M', \R)$.
We recall some basic identities that will be of use in showing that
$f \diamond \nabla$ is a connection:
\begin{align*}
(X + Y)u &= Xu + Yu && \text{by definition of pointwise addition} \\
(r \cdot X)u &= (ru) \cdot Xu && \text{by definition of pointwise scaling} \\
v(X + Y) &= vX + vY && \text{fibre-wise linearity of $v$} \\
v(r \cdot X) &= r \cdot vX && \text{fibre-wise linearity of $v$}
\end{align*}
We note that these identities hold for arbitrary bundle morphisms $(u, v)$,
smooth real-valued maps $r$ and sections $X, Y$, as long as the operations are
well-defined.

We now proceed to check that $f \diamond \nabla$ is a connection. To this end,
we first check additivity in the $\Gamma(TM')$ coordinate.
\begin{align*}
(f \diamond \nabla)((X + Y) \tensor s)
=& v\nabla(du^{-1}(X + Y)u \tensor v^{-1}su)u^{-1} \\
=& v\nabla((du^{-1}Xu + du^{-1}Yu) \tensor v^{-1}su)u^{-1} \\
=& v(\nabla(du^{-1}Xu \tensor v^{-1}su)
 + \nabla(du^{-1}Yu \tensor v^{-1}su))u^{-1} \\
=& v\nabla(du^{-1}Xu \tensor v^{-1}su)u^{-1}
 + v\nabla(du^{-1}Yu \tensor v^{-1}su))u^{-1} \\
=& (f \diamond \nabla)(X \tensor s) + (f \diamond \nabla)(Y \tensor s)
\end{align*}
Additivity in the $\Gamma(E')$ coordinate is similar. So, we check scalar
multiplicativity in the $\Gamma(TM')$ coordinate:
\begin{align*}
(f \diamond \nabla)(r \cdot X \tensor s)
=& v\nabla(du^{-1}(r \cdot X)u \tensor v^{-1}su)u^{-1} \\
=& v\nabla(du^{-1}(ru \cdot Xu) \tensor v^{-1}su)u^{-1} \\
=& v\nabla((ru \cdot (du^{-1}Xu)) \tensor v^{-1}su)u^{-1} \\
=& v (ru \cdot \nabla(du^{-1}Xu \tensor v^{-1}su))u^{-1} \\
=& v (ruu^{-1} \cdot \nabla(du^{-1}Xu \tensor v^{-1}su)u^{-1}) \\
=& r \cdot (v\nabla(du^{-1}Xu \tensor v^{-1}su)u^{-1}) \\
=& r \cdot ((f \diamond \nabla)(X \tensor s))
\end{align*}

We finally check the Liebnitz rule.
\begin{align*}
(f \diamond \nabla)(X \tensor (r \cdot s))
=& v\nabla(du^{-1}Xu \tensor v^{-1}(r \cdot s)u)u^{-1} \\
=& v(\nabla(du^{-1}Xu \tensor v^{-1}(ru \cdot su))u^{-1} \\
=& v\nabla(du^{-1}Xu \tensor ru \cdot (v^{-1}su))u^{-1} \\
=& v(
      d(ru)(du^{-1}Xu) \cdot v^{-1}su
      + ru \cdot \nabla(du^{-1}Xu \tensor v^{-1}su)
    )u^{-1} \\
=& v(d(ru)(du^{-1}Xu) \cdot v^{-1}su)u^{-1}
 + v(ru \cdot \nabla(du^{-1}Xu \tensor v^{-1}su))u^{-1}
\end{align*}
It now suffices to show that the left summand is
\[
  dr(X) \cdot s
\]
and the right summand is
\[
  r \cdot (f \diamond \nabla)(X \tensor s)
\]
For the right summand, we observe:
\begin{align*}
 & v(ru \cdot \nabla(du^{-1}Xu \tensor v^{-1}su))u^{-1} \\
=& v(ruu^{-1} \cdot \nabla(du^{-1}Xu \tensor v^{-1}su)u^{-1}) \\
=& v(r \cdot \nabla(du^{-1}Xu \tensor v^{-1}su)u^{-1}) \\
=&r \cdot (v\nabla(du^{-1}Xu \tensor v^{-1}su)u^{-1}) \\
=& r \cdot (f \diamond \nabla)(X \tensor s)
\end{align*}
For the left summand, we first observe a useful property of the derivative
operator $d-$. If $g : L \to M$ and $h : M \to N$ are smooth maps, then by the
pasting of pushout diagrams to give pushout diagrams, we have
\[
  d(h \circ g) = dh \circ dg
\]
Then, we have:
\begin{align*}
v(d(ru)(du^{-1}Xu) \cdot v^{-1}su)u^{-1}
=& v(\pi_{T\R}d(ru)(du^{-1}Xu) \cdot v^{-1}su)u^{-1}\\
=& v(\pi_{T\R}d(ruu^{-1})Xu \cdot v^{-1}su)u^{-1}\\
=& v(\pi_{T\R}drXu \cdot v^{-1}su)u^{-1}\\
=& v((\pi_{T\R}drX \cdot v^{-1}s)u)u^{-1}\\
=& v(v^{-1}(\pi_{T\R}drX \cdot s)u)u^{-1}\\
=& dr(X) \cdot s
\end{align*}
as required. This completes the proof of the following theorem.
\begin{thm}
Let $\pi : E \to M$, $\pi' : E' \to M'$ be bundles with a bundle isomorphism
$f = (u, v) : E \to E'$. If $\nabla$ is a connection on $\pi$, then
$f \diamond \nabla$ is a connection on $\pi'$.
\end{thm}
\begin{defn}
We call $f \diamond \nabla$ the shift of $\nabla$ along $f$ and say that $f$
takes $\nabla$ to $f \diamond \nabla$. From this point, we write $f\nabla$ as
opposed to $f \diamond \nabla$, whenever there is no confusion.
\end{defn}

\subsection{Category of Connections}\label{subsec:connctg}

Let $\pi_i : E_i \to M_i$ be a bundles equipped with a connections $\nabla_i$
for $i \in \set{1, 2, 3, 4}$. Let $f_{i, i + 1} = (u_{i, i + 1}, v_{i, i + 1})$
be bundle isomorphisms for $i \in \set{1, 2, 3}$. Then, the composite
$f_{1, 3} := f_{2, 3}f_{1, 2} = (u_{2, 3}u_{1, 2}, v_{2, 3}v_{1, 2})
=: (u_{1, 3}, v_{1, 3})$ is clearly a bundle isomorphism. We similarly define
composites $f_{i, j}$ for each $i < j \in \set{1, 2, 3, 4}$. Now, suppose
$\nabla_{i + 1} = f_{i, i + 1}\nabla_i$ for each $i \in \set{1, 2, 3}$. Then, we
immediately have
\[
  f_{i, k}\nabla_i = f_{j, k}f_{i, j}\nabla_i = f_{j, k}\nabla_j = \nabla_k
\]
for each $i < j < k$ in $\set{1, 2, 3, 4}$. In particular,
\[
  f_{3, 4}(f_{2, 3}f_{1, 2}\nabla_1) = \nabla_4
    = (f_{3, 4}f_{2, 3})f_{1, 2}\nabla_1
\]

We then observe the action of identity bundle morphisms. The identity bundle
morphism on $\pi_1$ is the pair $\id_{\pi_1} = (\id_{E_1}, \id_{M_1})$. Then,
\begin{align*}
   \id_{\pi_1}\nabla_1(X \tensor s)
&= \id_{E_1}\nabla_1(d(\id_{M_1})^{-1}X\id_{M_1}
                     \tensor \id_{E_1}^{-1}s\id_{M_1})\id_{M_1}^{-1} \\
&= \nabla_1(\id_{TM_1}^{-1}X \tensor s) \\
&= \nabla_1(X \tensor s)
\end{align*}
so that $\id_{\pi_1}\nabla = \nabla$. We finally observe that
$ff^{-1}\nabla = \id\nabla = \nabla$ for any connection $\nabla$ and any
compatible diffeomorphism $f$.

These observations motivate the following definition.
\begin{defn}
Let $\pi_1$ and $\pi_2$ be bundles equipped with connections $\nabla_1$ and
$\nabla_2$ respectively. Then, a bundle isomorphism
$f = (u, v) : \pi_1 \to \pi_2$ satisfying $f\nabla_1 = \nabla_2$ is called an
isomorphism, or simply morphism, of connections.
\end{defn}

From the work above, we have established the following results.
\begin{thm}
There exists a groupoid, denoted $\Conn$, whose objects are connections and
whose morphisms are isomorphisms of connections.
\end{thm}
\begin{defn}[Category of Connections]
We will call the category of the above function the category or groupoid of
connections. We will denote this category $\Conn$.
\end{defn}

We now try to establish a monoidal structure for a subcategory of the category
of connections. Let $\nabla_1$ and $\nabla_2$ be two connections with underlying
bundles $\pi_1 : E_1 \to M_1$ and $\pi_2 : E_2 \to M_2$ respectively. We will
consider the coproduct or disjoint union of these bundles in the category of
manifolds. There exists a smooth map
$\pi_1 \amalg \pi_2 : E_1 \amalg E_2 \to M_1 \amalg M_2$ which we will give the
structure of a vector bundle as follows. For this, we additionally assume that
the fibres of $E_1$ and $E_2$ are the same vector space. Let
$U = U_1 \amalg U_2, V = V_1 \amalg V_2$ be open sets in $M_1 \amalg M_2$ with
$U_i, V_i \subset M_i$ for $i \in \set{1, 2}$, and consider
$(U_1 \amalg U_2) \cap (V_1 \amalg V_2) = (U_1 \cap V_1) \amalg (U_2 \cap V_2)$.
We have a transition function $G_{U_1, V_1}$ on $U_1 \cap V_1$ from the bundle
$\pi_1$ and one $H_{U_2, V_2}$ on $U_2 \cap V_2$ from $\pi_2$. We define a
function $(G \amalg H)_{U, V} : U \cap V \to \GL_n(\C)$ piecewise, as
follows:
\[
  (G \amalg H)_{U, V}(x) := \begin{cases}
    G_{U_1, V_1}(x), & x \in U_1 \cap V_1 \subset M_1 \\
    H_{U_2, V_2}(x), & x \in U_2 \cap V_2 \subset M_2
  \end{cases}
\]
which is smooth since it is a disjoint union of smooth functions. Therefore,
\[
  G \amalg H := \set[(G \amalg H)_{U, V}]
                    {U, V \subset M_1 \amalg M_2 \text{ are open}}
\]
is a vector bundle structure on $\pi_1 \amalg \pi_2$. A section of
$E_1 \amalg E_2$ is a smooth map
\[
  s : M_1 \amalg M_2 \to E_1 \amalg E_2
\]
satisfying $(\pi_1 \amalg \pi_2)s = \id_{M_1 \amalg M_2}$. We note that this
guarantees that the $s = s_1 \amalg s_2$ where $s_i$ is a section of
$E_i$, $i \in \set{1, 2}$.

Similarly, $TM_1 \amalg TM_2 \to M_1 \amalg M_2$ is a vector bundle when $M_1$
and $M_2$ have the same dimension, and we can take this to be the definition of
the tangent bundle $T(M_1 \amalg M_2)$ on $M_1 \amalg M_2$. Now, let
$\pi_3 : E_3 \to M_3$ be another bundle where all the $E_i$ have the same fibres
and all the $M_i$ are equidimensional.

We can pick a convention for disjoint unions of sets as follows:
\[
  A \amalg B = (A \times \set{0}) \cup (B \times \set{1})
\]
Under this convention,
\[
  E_1 \amalg (E_2 \amalg E_3)
    = \set[(x_1, 0)]{x_1 \in E_1}
      \cup \set[((x_2, 0), 1)]{x_2 \in E_2}
      \cup \set[((x_3, 1), 1)]{x_3 \in E_3}
\]
and
\[
  (E_1 \amalg E_2) \amalg E_3
    = \set[((x_1, 0), 0)]{x_1 \in E_1}
      \cup \set[((x_2, 1), 0)]{x_2 \in E_2}
      \cup \set[(x_3, 1)]{x_3 \in E_3}
\]
We have similar descriptions for the two distinct parenthesizations for
$M_1 \amalg M_2 \amalg M_3$. Now, the map
\[
  \alpha_{E_1, E_2, E_3} : E_1 \amalg (E_2 \amalg E_3)
                           \to (E_1 \amalg E_2) \amalg E_3
\]
defined by
\[
  (x_1, 0) \mapsto ((x_1, 0), 0),
  ((x_2, 0), 1) \mapsto ((x_2, 1), 0),
  ((x_3, 1), 1) \mapsto (x_3, 1)
\]
is easily seen to be bijective and fibre-preserving. Smoothness and naturality
in the subscripts follow from those of associators in $\Man$. We can make a
similar argument for similarly defined unitors $\rho_E$ and $\lambda_E$. We thus
have the following theorem.
\begin{thm}
The subcategory of the category of bundles consisting of bundles with base
spaces of a fixed dimension $d$ and total spaces with equal fibres is monoidal
under the disjoint union of manifolds.
\end{thm}
\begin{defn}[Category of {$(V, d)$--bundles}]
The subcategory of the category of bundles in the above theorem is called the
category of $V$--fibred bundles on $d$--dimensional manifolds or of
$(V, d)$--bundles and is denoted $\Bun^V_d$.
\end{defn}

We now define a function
\[
  \nabla_1 \amalg \nabla_2
    : \Gamma(T(M_1 \amalg M_2) \tensor E_1 \amalg E_2)
    \to \Gamma(E_1 \amalg E_2)
\]
as follows, for $i \in \set{0, 1}$:
\[
  (\nabla_1 \amalg \nabla_2)((X_1 \amalg X_2) \tensor (s_1 \amalg s_2))(x, i)
    = \nabla_{i + 1}(X_{i + 1} \tensor s_{i + 1})(x)
\]
It is easy to see that this function satisfies the connection identities
piecewise so that it satisfies these identities on its entire domain. Thus,
$\nabla_1 \amalg \nabla_2$ is a connection. We then consider a connection
$\nabla_3$ on $\pi_3$. Letting
$f = (u, v) = (\alpha_{M_1, M_2, M_3}, \alpha_{E_1, E_2, E_3})$, we now wish to
verify that
\[
  f(\nabla_1 \amalg (\nabla_2 \amalg \nabla_2))
    = (\nabla_1 \amalg \nabla_2) \amalg \nabla_3
\]
For this, we will need to inspect the expression:
\begin{align*}
  & f(\nabla_1 \amalg (\nabla_2 \amalg \nabla_2))(
      (X_1 \amalg X_2) \amalg X_3
      \tensor (s_1 \amalg s_2) \amalg s_3
      ) \\
  =& v(\nabla_1 \amalg (\nabla_2 \amalg \nabla_3))\br{
    du^{-1}((X_1 \amalg X_2) \amalg X_3)u
    \tensor v^{-1}((s_1 \amalg s_2) \amalg s_3)u
  }u^{-1} \\
  =& v(\nabla_1 \amalg (\nabla_2 \amalg \nabla_3))\br{
    du^{-1}((X_1 \amalg X_2) \amalg X_3)u
    \tensor (s_1 \amalg (s_2 \amalg s_3))
  }u^{-1}
\end{align*}
We then observe the following basic fact.
\begin{lem}
For tangent bundles $\pi : TM_i \to M_i$, $i \in \set{1, 2, 3}$, we have
$d\alpha_{M_1, M_2, M_3} = \alpha_{TM_1, TM_2, TM_3}$.
\end{lem}
\begin{proof}
We denote $\alpha_{TM} := \alpha_{TM_1, TM_2, TM_3}$,
$\alpha_{M} := \alpha_{M_1, M_2, M_3}$.
Let $R$ be a manifold, and $x$ and $y$, smooth maps making the following
diagram commute:
\[\begin{tikzpicture}[baseline=(a).base]
\node[scale=\diagscale] (a) at (0, 0){
\begin{tikzcd}[column sep=large, row sep=huge]
&
(TM_1 \amalg TM_2) \amalg TM_3
  \ar[d, "\pi_{(12)3}" description]
  \ar[rrd, "x" above right] & \\
TM_1 \amalg (TM_2 \amalg TM_3)
  \ar[ur, "\alpha_{TM}" above left]
  \ar[dr, "\pi_{1(23)}" below left] &
(M_1 \amalg M_2) \amalg M_3
  \ar[rr, "y\alpha_{M}^{-1}" description, dashed] & &
R \\ &
M_1 \amalg (M_2 \amalg M_3)
  \ar[u, "\alpha_{M}" description]
  \ar[urr, "y" below right] &
\end{tikzcd}
};
\end{tikzpicture}\]
The map $y\alpha_M^{-1}$ satisfies $(y\alpha_M^{-1})\alpha_M = y$ and
$(y\alpha_M^{-1})\pi_{(12)3} = y\pi_{1(23)}\alpha_{TM} = x$. Suppose another map
$r$ satisfies $r\alpha_M = y$. Then, $r = y\alpha_M^{-1}$ showing that
$y\alpha_M^{-1}$ is unique.

Therefore, $\alpha_{TM}$ makes the part of the diagram excluding $R$ a pushout
and hence must be $d\alpha_M$, by the uniqueness of the derivative.
\end{proof}

The above theorem yields:
\begin{align*}
  & f(\nabla_1 \amalg (\nabla_2 \amalg \nabla_2))(
      (X_1 \amalg X_2) \amalg X_3
      \tensor (s_1 \amalg s_2) \amalg s_3
      ) \\
  =& v(\nabla_1 \amalg (\nabla_2 \amalg \nabla_3))\br{
    du^{-1}((X_1 \amalg X_2) \amalg X_3)u
    \tensor (s_1 \amalg (s_2 \amalg s_3))
  }u^{-1}\\
  =& v(\nabla_1 \amalg (\nabla_2 \amalg \nabla_3))\br{
    (X_1 \amalg (X_2 \amalg X_3))
    \tensor (s_1 \amalg (s_2 \amalg s_3))
  }u^{-1} \\
  =& \alpha_{TM_1, TM_2, TM_3}(\nabla_1 \amalg (\nabla_2 \amalg \nabla_3))\br{
    (X_1 \amalg (X_2 \amalg X_3))
    \tensor (s_1 \amalg (s_2 \amalg s_3))
  }\alpha_{M_1, M_2, M_3}^{-1}
\end{align*}
where the last expression is easily seen to be
\[
  ((\nabla_1 \amalg \nabla_2) \amalg \nabla_3)(
    (X_1 \amalg X_2) \amalg X_3 \tensor (s_1 \amalg s_2) \amalg s_3
  )
\]
We can similarly show that the unitors in $\Man$ yield unitors for disjoint
unions of connections of bundles with equal fibres and equidimensional base
spaces. We have thus proved the following theorem.
\begin{thm}\label{thm:bundle_gluing}
For a vector space $V$ and a non-negative integer $d$, the subcategory of the
category of connections consisting of all connections on objects in
$\Bun_d^{V}$ and all morphisms of connections between them is a monoidal
category under disjoint union.
\end{thm}
\begin{defn}[Category of {$(V, d)$}--Connections]
We call the subcategory of the category of connections in the above theorem the
category of connections on $V$--fibred bundles on $d$--dimensional manifolds or
of $(V, d)$--connections. We denote this category $\Conn^V_d$.
\end{defn}

\TODO{We need a notion of gluing connections.}

\subsection{Gluing Complex Bundles}

We will now develop a notion of gluing complex bundles on compact manifolds with
boundary along with connections on these bundles. To accomplish this, we will
first show the following:
\begin{thm}\label{thm:gluing}
Given any smooth compact manifold $M$ with boundary and a complex bundle
$\pi : E \to M$ with fibre $V \cong \C^n$, there exists a complex bundle
$\wh{\pi} : \wh{E} \to M$ which restricts to the trivial bundle on a collar $C$
of $\partial M$ and to $E$ on $M \setminus C_0$, for some open set $C_0$
strictly containing $C$.
\end{thm}
\begin{proof}
Let $U$ and $V$ any two open sets of $M$ over which $E$ trivializes and
$G_{U, V} : U \cap V \to \Aut(V)$, the assignment of transition functions to
their intersection. For each $x \in U \cap V$, $\Aut(V)$ being isomorphic of
$\GL(n, \C)$ and hence path-connected, there is a path
$\gamma_{U, V, x} : I \to \Aut(V)$ such that $\gamma_{x}(0) = G_{U, V}(x)$ and
$\gamma_{U, V, x}(1) = \id_V$. We can choose these paths in such a way that,
over the same trivializations, the transition functions
$\set{\gamma_{U, V, x}(t)}_{U, V, x}$ yield a complex bundle
$\pi_t : E(t) \to M$ for every $t \in I$.
\TODO{Show that this last claim does indeed hold.}

By the smooth collar theorem, there exists a nieghbourhood $C_0$ of $\partial M$
diffeomorphic to the cylinder $\partial M \times I$ on $\partial M$,
with $\partial M$ identified with $\partial M \times \set{1}$.
We can then cut $C_0$ into pieces
$C' \cong \partial M \times \sbr{0, \frac{1}{2}}$ and
$C \cong \partial M \times \sbr{\frac{1}{2}, 1}$ that are each diffeomorphic to
$\partial M \times I$. There exists a bump function $f : M \to \R$ such that
$f$ is $1$ on $M \setminus C_0$, decreasing on $C'$ and vanishes on $C$:
\[
  f(x) = \begin{cases}
    1 & x \in M \setminus C_0 \\
    \frac{1}{2}(1 - \text{erf}(at + b))
      & x = (x', t) \in C', x' \in \partial M, t \in \sbr{0, \frac{1}{2}} \\
    0 & x \in C
  \end{cases}
\]
where $a$ and $b$ are appropriately chosen constants.
\TODO{Cite the existence of such a bump function}.

Taking the same trivializations as $E$ and transition functions
$\set{\gamma_{U, V, x}(f(x))}_{U, V, x}$ yields the required bundle.
\TODO{Show that these transition functions do indeed form a bundle -- check
smoothness of transition function assignments to base-points and such.}
\end{proof}

It is straightforward to verify that for any cospan $M \ot[f] X \to[g] N$ and
any object $V$ in $\Man$, the following holds:
\[
  V \times (M \amalg_X N) \cong (V \times M) \amalg_{X \times V} (V \times N)
\]
such that the isomorphism is fibre-preserving and linear on fibres. Hence,
trivial bundles always glue at boundaries to give trivial bundles. This
observation yields a gluing operation $- * -$ for the following collection of
complex bundles:
\[
  \set[\wh{E}]{E \text{ is a complex bundle with fibre } V}
\]
since the bundles $\wh{E}$ are trivial near their boundaries. We observe that
gluing fibres at the boundaries is a pushout in the category of manifolds and
hence the operation is associative up to diffeomorphism. It is also not hard to
verify that the associator diffeomorphisms are fibre-preserving and linear on
the fibres. Furthermore, given a bundle $\wh{E} \to M$ where
$\partial M = W_0 \amalg W_1$, we take the trivial bundles
$W_0 \times I \times V \to W_0 \times I$ and
$W_1 \times I \times V \to W_1 \times I$, and observe that they act as gluing
identities for $\wh{E}$ on either side by a simple reparametrization. This
establishes a notion of cobordism of manifolds. That is,
\begin{thm}
Given a double category of cobordisms $\s{C}$ (e.g. $\Cob_d$ or $2\Thick$) and a
complex vector space $V$, we have a double category $\BBun^V_{\s{C}}$ consisting
of the following data:
\begin{enmrt}
\li Object category: objects are trivial $V$--bundles on the objects of $\s{C}$
and morphisms are bundle isomorphisms
\li Morphism category: objects are complex bundles $\wh{E} \to M$, for $M$ in
the morphism category of $\s{C}$ and complex bundles $E \to M$; morphisms are
bundle isomorphisms
\li Source functor: each bundle $\wh{E} \to M$ is sent to the trivial bundle on
the source of $M$; action of morphisms is by restriction to appropriate boundary
components
\li Target functor: defined analogously as the source functor
\li Unit functors: each bundle $\wh{E} \to M$ is sent to the trivial bundle on
the cylinder on the appropriate boundary components
\li Horizontal composition: gluing corresponding fibres at common boundary
\li Horizontal composition associators: inherited from the category of manifolds
\li Horizontal composition unitors: inherited like the associators
\li Monoidal product: disjoint union
\li Monoidal unit(s): empty bundle(s)
\end{enmrt}
\end{thm}

We also notice that the above constructions apply to smooth (real) bundles as
long as the transition functions at the points in some collar of the boundary
can be connected to the identity function by paths in the automorphism group of
the fibre. This is possible if these transition functions all have positive
determinant. \TODO{Justify this, if needed.}

\subsection{Gluing Connections on Complex Bundles}

As we now have a notion of gluing for (a subset of) complex bundles with a fixed
fibre, we will now use a similar idea to glue connections defined on these
bundles. Let $\pi : E \to M$ be a (smooth, complex or holomorphic) bundle. It is
well known that sheaves of global sections of smooth, complex or holomorphic
bundles over $M$ are $\Cinf(M, \R)$--modules and that the category of
modules over any ring is closed. Hence, it is immediate that the collection of
all $\Cinf(M, \R)$--linear maps $\Gamma(TM \tensor E) \to \Gamma(E)$ is
itself a $\Cinf(M, \R)$--module -- we will call it $\s{M}(E)$. We
know that a connection $\Gamma(TM \tensor E) \to \Gamma(E)$ is an $\R$--bilinear
map but it is not $\Cinf(M, \R)$--bilinear. However, we can show that it is
$\Cinf(M, \R)$--linear.

\begin{lem}
Let $\nabla : \Gamma(TM \tensor E) \to \Gamma(E)$ be a connection on
a smooth, complex or holomorphic bundle $E \to M$. Let $X, Y \in \Gamma(TM),
s, t \in \Gamma(E), r \in \Cinf(M, \R)$. Then the following hold:
\begin{align}
  \nabla((X \tensor s) + (Y \tensor t))
    &= \nabla(X \tensor s) + \nabla(Y \tensor t) \\
  \nabla(r \cdot (X \tensor s)) &= r \cdot \nabla(X \tensor s) \\
  \nabla(0) &= 0
\end{align}
so that $\nabla$ is $\Cinf(M, \R)$--linear.
\end{lem}
\begin{proof}
By definition, $\nabla$ is an $\R$-- or $\C$--bilinear map
$\Gamma(E) \times \Gamma(TM) \to \Gamma(E)$, or equivalently, an $\R$-- or
$\C$--linear map $\Gamma(TM \tensor E) \to \Gamma(E)$. From this, equations
1 and 3 are immediate. For equation 2, we recall the definition of the action
$r \cdot (X \tensor s)$:
\[
  (r \cdot (X \tensor s))(x)
    = r(x) \cdot (X(x) \tensor s(x))
    = (r(x) \cdot X(x)) \tensor s(x)
\]
where the last equality follows from the $\R$--bilinearity of $\tensor$. We can
rewrite the right side as:
\[
  (r(x) \cdot X(x)) \tensor s(x)
    = (r \cdot X)(x) \tensor s(x)
    = ((r \cdot X) \tensor s)(x)
\]
so that
\[
  r \cdot (X \tensor s) = (r \cdot X) \tensor s
\]
Equation 3 now follows from the definition of a connection.
\end{proof}

It is well known that the set of differences of connections on a bundle from a
submodule of the $\R$-- or $\C$--module of linear maps
$\Gamma(TM \tensor E) \to \Gamma(E)$. We will show that the ground ring here
can again be replaced with $\Cinf(M, \R)$ so that connections form an
affine sumodule of $\s{M}(E)$.

\begin{lem}
Let $\nabla_i, i \in \set{1, 2, 3, 4}$ be connections
$\Gamma(TM \tensor E) \to \Gamma(E)$. Then, the following hold:
\begin{enmrt}
\li $(\nabla_1 - \nabla_2) + (\nabla_3 - \nabla_4)$ is a difference of
connections.
\li For all $f \in \Cinf(M, \R)$, $f \cdot (\nabla_1 - \nabla_2)$ is a
difference of connections.
\li The linear map sending all inputs to zero is a difference of connections.
\end{enmrt}
In particular, the set
\[
  V = \set[\nabla_{\alpha} - \nabla_{\beta}]
          {\nabla_{\alpha}, \nabla_{\beta} \text{ are connections on } E}
\]
is a submodule of $\s{M}(E)$.
\end{lem}
\begin{proof}
Since $V$ is a linear subspace of the vector space of all $\R$--linear
maps $\Gamma(TM \tensor E) \to \Gamma(E)$, (i) and (iii) are immediate.
\end{proof}

\end{document}

