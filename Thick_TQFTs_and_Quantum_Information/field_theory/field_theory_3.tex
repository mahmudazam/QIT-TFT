
\subsection{Parallel Field Theory}

We are now equipped with all the machinery to define our desired notion of
topological quantum field theory.

\begin{defn}[Parallel Field Theory]
Let $A$ be a $\K$--algebra for $\K = \R$ or $\C$. Then, we define the data of a
monoidal double functor
\[
  F : \TG\br{\CConn^V_{\DThick}} \to \FFVect_{\K}
\]
Each horizontal $1$--morphism in the domain is identical to one in the
double category of transport graphs in a manifold. $F$ is thus defined on
horizontal $1$--morphisms identically to the functor in definition
\ref{defn:sing_man_tqft}.

We observe that each object can be reduced to a source or target of a
pretransport graph since, by definition the copies of $I$ and $\varnothing$ are
matched up with blue and green vertices respectively. This allows us to define
$F$ on objects identically to \ref{defn:sing_man_tqft} again. It is then easy to
see that the action of $F$ on vertical $1$--morphisms and $2$--morphisms can be
adapted similarly.

A monoidal double functor $F$ defined in this way is called a parallel field
theory (on $2$--dimensional thick tangles or $\DThick$).
\end{defn}

\begin{rmk}
The ending parenthetical remark in the previous definition suggests that we can
easily consider such field theories over other cobordism categories but we will
not pursue this idea for now.
\end{rmk}

We have not yet discussed if enough useful elements of the algebra $A$ can be
accessed with a field theory of this form. After all, this was the original
issue with $1$--categorical TQFTs. We will not treat this issue in full in this
paper but we will note that the machinery we have developed so far puts no
serious restrictions on the bundles or connections we can choose over our
manifolds. What we mean by this is that given an arbitray bundle with a
connection, we can make it gluable by only modifying it in a small collar of the
boundary. The bundle and connection behave as usual over rest of the base
manifold. We hope that this will provide enough structure to ensure that enough
useful algebra elements become accessible with a parallel field theory.
Nevertheless, we will make this problem precise for future work.
One of the main questions to be answered here the following:

\begin{qstn}\label{qstn:elem_from_pt}
Given a manifold $M$ and some fixed element $a$ of a complex algebra
$A^{\tensor n}$, are there
\begin{enmrt}
\li an $A$--fibred complex bundle $E \to M$,
\li a complex linear connection $\nabla$ on $E$,
\li a transport graph $G$ in $M$ with $S(G)$ consisting of only green vertices,
$T(G)$ having $n$ blue vertices and with the paths in the geometric realization
of $G$ away from some small neighbourhood of $\partial M$,
\end{enmrt}
such that $a$ is obtained as a linear map $\C \to A^{\tensor n}$ in
the process described in definition \ref{defn:sing_man_tqft} for horizontal
$1$--morphisms?
\end{qstn}

\begin{qstn}\label{qstn:map_from_pt}
Given a manifold $M$ and some linear map $f : V \to V$, are ther
\begin{enmrt}
\li a $V$--fibred complex bundle $E \to M$
\li a complex linear connection $\nabla$ on $E$,
\li a transport graph $G$ in $M$ with $S(G)$ and $T(G)$ both consisting of only
blue vertices, with its paths away from some neighbourhood of $\partial M$, as
before,
\end{enmrt}
such that $f$ is obtained in the process described in definition
\ref{defn:sing_man_tqft} for horizontal $1$--morphisms?
\end{qstn}

\TODO{Improve this statement.}

If we can answer this question in the affirmative for a considerable
collection of elements $a \in A$, then a parallel field theory gives us a
concrete way to perform computations involving the multiplication of $A$ and
automorphisms of $A$ and tensor products of these maps, using the geometry of
manifolds. Hence, we make the following definition:


\begin{defn}
If the answers to question \ref{qstn:elem_from_pt} (or \ref{qstn:map_from_pt})
is yes for some manifold $M$, then we say that $a$ (or $f$) is accessible from
$M$.
\end{defn}

If an element $a$ is accessible from a manifold $M$, it is in the image of a
parallel field theory on $M$.
If an element $a$ is accessible from a thick tangle
$M : \varnothing \to I^{\amalg n}$ where the chosen transport graph is
admissible, then $a$ is in the image of a parallel field theory on $\DThick$.

\begin{defn}
In the latter case, we say that $a$ is accessible from $\DThick$.
\end{defn}

After this, it is easy to see
that we can compute with accessible elements using the machinery of a parallel
field theory. This picture will become clearer as we treat quantum information
and computing in the next subsection.

