
\subsection{Parallel Field Theory}

We are now equipped with all the machinery to define our desired notion of
quantum field theory.

\begin{defn}[Parallel Field Theory]
Let $A$ be a $\K$--algebra for $\K = \R$ or $\C$. Then, we define the data of a
monoidal double functor
\[
  F : \TG\br{\CConn^V_{\DThick}} \to \FFVect_{\K}
\]
The object and morphism functions are defined identically as in definition
\ref{defn:sing_man_tqft} -- there is no ambiguity with boundary components
because they match the sources and targets of the transport graphs. Such a
monoidal double functor is called a parallel field theory (over thick tangles).
\end{defn}

\begin{rmk}
The ending parenthetical remark in the previous definition suggests that we can
easily consider such field theories over other cobordism categories but we will
not pursue this idea for now.
\end{rmk}

We have not yet discussed if enough useful elements of the algebra $A$ can be
accessed with a field theory of this form. After all, this was the original
issue with $1$--categorical TQFTs. We will not treat this issue in full in this
paper but we will note that the machinery we have developed so far puts no
serious restrictions on the bundles or connections we can choose over our
manifolds. What we mean by this is that given an arbitray bundle with a
connection, we can make it gluable by only modifying it in a small collar of the
boundary. The bundle and connection behave as usual over rest of the base
manifold. We hope that this will provide enough structure to ensure that enough
useful algebra elements become accessible with a parallel field theory.
Nevertheless, we will make this problem precise for future work.
The main question to be answered here is this:
\begin{displayquote}
Given a manifold $M$, a $V$--fibred
complex bundle $E \to M$, a complex linear connection $\nabla$ on $E$, and an
element $v \in V$, \textit{is there a path $\gamma$ in $M$ and a fixed
$s_v \in V$, depending on $v$, such that $v$ is obtained by parallel transport
of $s_v$ along $\gamma$}?
\end{displayquote}

\TODO{Improve this statement.}

\TODO{Connect with hyperbolic band theory paper, perhaps}

