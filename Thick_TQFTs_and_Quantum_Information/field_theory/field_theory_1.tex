
\subsection{Double Category of Finite Dimensional Vector Spaces}

We consider as the object or $0$--morphism category of the codomain double
category the monoidal category of finite-dimensional, real (or complex) vector
spaces $\FVect_{\K}$ for $\K = \R$ or $\C$. We then notice a categorification
of the collection of morphisms of this category as follows.

We consider the collection of all morphisms of $\FVect_{\K}$,
\[
  \s{L} := \set[f]{f \in \Hom_{\FVect_{\K}}(U, V), U, V \in \Ob \FVect_{\K}}
\]
as the object collection of the horizontal $1$--morphism category of our double
category. For each such map, we choose some $p \times q$ matrix representation
$[a_{ij}]$ which yields a mapping $\iota : \s{L} \to \M_{N}(\C)$, where
$\M_{\N}(\C)$ is the set of infinite complex matrices indexed by $\N \times \N$,
defined by:
\[
  \iota(a)_{ij} = \begin{cases}
    a_{ij} & 1 \leq i \leq p, 1 \leq j \leq q \\
    0      & \text{otherwise}
  \end{cases}
\]

This provides a notion of $2$--morphism for our double category under
construction. We notice that the Banach space $\s{B}$ of bounded operators on
the Hilbert space space $\ell^2(\N)$ can be seen as a subset of $\M_{\N}$, such
that $\s{L} \subset \s{B} \subset \M_{\N}$. It is well-known that $\s{B}$, being
a Banach space, is a simply-connected topological space. Now, for $a$ and $a'$
in $\s{L}$, we define a $2$--morphism to be a homomotpy class of paths $\alpha$
from $\iota(a)$ to $\iota(')$ in $\s{B}$, which we denote as
$\alpha : a \To a'$. By the fact the fundamental groupoid
$\Pi_1(\s{B})$ is a category, our $2$--morphisms have a strictly associative and
unital composition.  We note that the fundamental groupoid $\Pi_1(\s{B})$ is not
our morphism category -- it only supplies morphisms for $\s{L}$. An instance of
why the distinction is important is that a single morphism in $\Pi_1(\s{B})$
might represent morphisms between two different pairs of objects in $\s{L}$,
depending on the chosen matrix representations.

It is worthwhile observing the action of composition and tensor products of
linear maps on $2$--morphisms. We first notice that iota can be defined so that
it is multiplicative in two ways:
\[
  \iota(b \circ a) := \iota(b) \cdot \iota(a)
\]
where the right-hand-side product is the matrix product in $\M_{\N}(\C)$, and
\[
  \iota(a \tensor b) := \iota(a) \tensor \iota(b)
\]
where the $\tensor$ on the right is given by the Kronecker product. Now,
consider pairs of homotopic paths
$\alpha_1, \alpha_2 : a \To a'$ and
$\beta_1, \beta_2 : b \To b'$, where $b, a$ and $b', a'$ are
composeable pairs of linear maps. We consider the pointwise composites:
\begin{equation}\label{eqn:pointwise_comp}
  (\beta_i \circ \alpha_i)(t) := \beta_i(t) \circ \alpha_i(t), t \in [0, 1],
    i \in \set{1, 2}
\end{equation}
Now, the $\beta_i \circ \alpha_i$ are clearly paths
$b \circ a \To b' \circ a'$ in $\s{B}$-- we wish to show that
they are homotopic, making the operation well-defined on homomotopy classes of
paths. It suffices to observe that $\s{B}$ is simply connected, so that there is
exactly one class of homotopic paths between two points in $\Pi_1(\s{B})$.

Consider again elements
$a : U \to V, a' : U' \to V', b : X \to Y, b' : X' \to Y'$ of $\s{L}$. We define
\[
  n_x := \dim \dom x, m_x := \dim \codom x, x \in \set{a, a', b, b'}
\]
and
\[
  N_{x} = N_{x'} := \max\set{n_x, n_{x'}},
  M_{x} = M_{x'} := \max\set{m_x, m_{x'}}, x \in \set{a, b}
\]
We then have matrix representations of each $x \in \set{a, a', b, b'}$:
\[
  \iota'(x) := [\iota(x)_{ij}] \in \M_{M_x \times N_x}(\C)
\]
Now, $\M_{M_x \times N_x}(\C)$, being simply connected, has a path
$\gamma_x$ from $\iota'(x)$ to $\iota'(x')$ for each $x \in \set{a, b}$. In
fact, using the inclusion $\M_{M_x \times N_x}(\C) \hto \s{B}$ induced by
$\iota$, each $\gamma_x$ yields a path $\wh{\gamma_x}$ in $\s{B}$ from
$\iota(x)$ to $\iota(x')$, with its image contained in $\s{L}$. Since $\s{B}$ is
simply connected, every path in $\s{B}$ from $\iota(x)$ to $\iota(x')$ is
homotopic to $\wh{\gamma_x}$. Furthermore,
$\wh{\gamma_a}(t) \tensor \wh{\gamma_b}(t)$ is also a path from
$\iota(a \tensor a') = \iota(a) \tensor \iota(a')$ to
$\iota(b \tensor b') = \iota(b) \tensor \iota(b')$ in $\s{B}$, to which all
other paths with the same endpoints are homotopic. This shows that
\[
  (\wh{\gamma_a} \tensor \wh{\gamma_b})(t) :=
    \wh{\gamma_a}(t) \tensor \wh{\gamma_b}(t)
\]
is well-defined on homotopy classes of paths in $\Pi_1(\s{B})$. For
associativity of $\tensor$, we observe that
\[
  \iota(a \tensor (b \tensor c))
    = \iota(a) \tensor \iota(b) \tensor \iota(c)
    = \iota((a \tensor b) \tensor c)
\]
so that the constant path on $\iota(a) \tensor \iota(b) \tensor \iota(c)$
functions as the associator
\[
  a \tensor (b \tensor c) \to (a \tensor b) \tensor c
\]
For unitality of $\tensor$, we take the matrix $1_{\tensor} \in \s{B}$ whose
$(i, j)$ entry is $1$ if $i = j = 1$ and is $0$ otherwise, and then we observe:
\[
  \iota(a \tensor \id_{\K}) = \iota(a) \tensor 1_{\tensor} = \iota(a)
    = \iota(\id_{\K} \tensor a)
\]
so that the constant path on $\iota(a)$ functions as a left and right unitor.

We then take horizontal composition to be given by composition of linear maps
which is strictly associative and unital, with coherence following from that in
the category of vector spaces. The source and target functors are obvious -- we
send each linear map to its domain and codomain respectively. The unit functor
is also obvisous -- we send each object to its identity linear map.
The axioms of a monoidal double category given in the unpacked version of
definition 2.9 in \cite[5]{SymMonBicat} are also easily verified.

\TODO{Add an appendix entry for this verification.}

We compile these results into the following definition:
\begin{defn}[Double Category of Finite Dimensional Vector Spaces]
The following data form a monoidal double category:
\begin{enmrt}
\li Object category: $\FVect_{\K}$
\li Morphism category: $\s{L}$
\li Source functor: $\dom : (f : X \to Y) \mapsto X$
\li Target functor: $\codom : (f : X \to Y) \mapsto Y$
\li Unit functor: $V \mapsto \id_V$
\li Horizontal composition: $(g, f) \mapsto g \circ f$
\li Horizontal composition associator: constant path on
$\iota(a \circ b \circ c)$
\li Horizontal composition unitor: constant path on $\iota(\id_V)$, for a vector
space $V$
\li Monoidal product: $\tensor$ in appropriate contexts defined above
\li Monoidal unit: $\K$ for the object category and $\id_{\K}$ for the morphism
category
\li Monoidal associators: constant path on
$\iota(a) \tensor \iota(b) \tensor \iota(c)$ as an associator
\[
  a \tensor (b \tensor c) \to (a \tensor b) \tensor c
\]
\li Monoidal unitor: constant path on $\iota(\id_{\K}) = 1_{\tensor}$
\end{enmrt}
This is called the monoidal double category of finite dimensional $\K$--vector
spaces and is denoted $\FFVect_{\K}$.
\end{defn}

We finally note that if the choice of matrix representations poses foundational
problems, we can easily switch to the skeleton of $\FVect_{\K}$ consisting of
the spaces $\K^n$ for all $n \in \N$.

%\begin{enmrt}
%
%\li The object category is easily seen to be a monoidal category -- it is the
%monoidal category generated by a single object $A$ in a monoidal category (of
%vector spaces) and associators and unitors between its monoidal products.
%
%The morphism category consists of linear maps of the form:
%\[
%  a : A^{\tensor n} \to A^{\tensor m} \text{ and }
%  b : A^{\tensor n'} \to A^{\tensor m'}
%\]
%with some parenthesizing pattern for each domain or codomain tensor power. The
%tensor product $a \tensor b$ gives the object function of the monoidal product
%functor. The pointwise tensor product \ref{eqn:pointwise_tensor} is the morphism
%function of the monoidal product functor.
%
%\li The monoidal unit of the object category is the base field, say $\C$, and
%the unit horizontal $1$--morphism is also the monoidal unit for horizontal
%$1$--morphisms:
%\[
%  a \tensor 1_{\C} = a = 1_{\C} \tensor a
%\]
%\end{enmrt}

