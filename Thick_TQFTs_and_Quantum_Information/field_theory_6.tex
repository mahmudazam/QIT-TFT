
\subsection{Quantum Computing Revisited}

Given a parallel field theory as defined so far, we can extend the domain of
$F$ to the structure $\TG^+\br{\CConn^V_{\DThick}}$ in an obvious way to obtain
a ``functor'' $F^+$ as follows. $F^+$ is identical to $F$ on the object category
-- there are no issues here since the object category was not modified in
constructing $\TG^+\br{\CConn^V_{\DThick}}$. A horizontal $1$--morphism in this
``double category'', however, is of the form
\[
  M = M_1 + \cdots + M_k
\]
for horizontal $1$--morphisms $M_i, i \in \set{1, \dots, k}$ in
$\TG\br{\CConn^V_{\DThick}}$. $F^+(M)$ is defined to be
\[
  F^+(M) := F(M_1) + \cdots + F(M_k)
\]
where the addition on the right is not well-defined as is. To define this we
observe that each $F^+(M_i)$ is a linear map
$F(I)^{\tensor n_i} \to F(I)^{\tensor m_i}$. Then, we choose a sensible
embedding of each $F(I)^{\tensor n_i}$ in $F(I)^{\tensor \max_i n_i}$ and of
each $F(I)^{\tensor m_i}$ in $F(I)^{\tensor \max_i m_i}$. After this, the
addition is taken within the space of linear maps
$F(I)^{\tensor \max_i n_i} \to F(I)^{\tensor \max_i m_i}$. We also note that if
$M = \varnothing$, then we define:
\[
  F(M)(x) := 0, \forall x
\]

\begin{defn}
An additive parallel field theory is an ``additive monoidal double functor''
\[
  F : \TG^+(\CConn^V_{\DThick}) \to \FFVect_{\C}
\]
defined using the above construction.
Accessibility is defined similarly for additive parallel field theories.
\end{defn}

Given a collection of $1$--qubit gates, we can take sums of tensor products of
these gates to obtain multi-qubit gates that are not elementary tensors.  Thus,
if we can solve the accessibility problem for $1$--qubit gates in the first
sense of parallel field theories, we can express sums of tensor products of
these gates using multitangles. Thus, we have a concrete way to express both
quantum registers and circuits using structures in $\TG^+(\CConn^V_{\DThick})$
which finally yield usual linear algebraic quantum registers and circuits under
additive parallel field theories. In fact, both approaches to quantum computing
discussed before can be adapted to this framework. It is then also easy to adapt
this notion back to the single manifold case of \ref{subsec:sing_man_tqft}.

