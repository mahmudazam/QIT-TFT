
\documentclass[./Thick_TQFTs_and_Quantum_Information.tex]{subfiles}

\begin{document}

\section{A Double Categorical Approach}

It is well known that taking $d$--dimensional manifolds without boundary as
objects, diffeomorphisms between them as vertical $1$--morphisms, $(d +
1)$--dimensional cobordisms between them as horizontal $1$--morphisms and
boundary preserving diffeomorphisms between cobordisms as $2$--morphisms yields
a fibrant monoidal double category $\Cob_{d + 1}$ under the disjoint union of
manifolds \cite{SymMonBicat}. Shulman gives a trifunctor $\mathcal{H}$ from the
tricategory of fibrant double categories to the tricategory of bicategories that
takes $\Cob_{d + 1}$ to a monoidal bicategory -- in fact, Shulman proves that
$\mathcal{H}$ takes any fibrant monoidal double category to a bicategory.

On the other hand, the monoidal category of thick tangles $2\Thick$ as defined
in \cite{NonCommTQFT} is a decategorification of a monoidal bicategory of thick
tangles $\PTT$ defined in \cite{NonSemiSimp}. Taking inspiration from this
situation, we assume that there is a fibrant monoidal double category
$\DThick$ which, under $\mathcal{H}$, yields $\PTT$. The structures defining
$\DThick$ are the ones analogous to $\Cob_{2}$:
\begin{itemize}
\setlength{\itemsep}{0pt}
\item objects are diffeomorphism classes of disjoint unions of the interval
$I = [0, 1]$\footnote{The justification for equating disjoint unions of the
interval up to diffeomorphism is that the monoidal product on the objects of
$\PTT$ (and $2\Thick$) is strict given that the objects are taken to be the
integers $n$ as opposed to $n$--fold disjoint unions $I^{\amalg n}$ with
different bracketings.}
\item vertical $1$--morphisms are only the identity morphisms
\item horizontal $1$--morphisms $I^{\amalg n} \to I^{\amalg m}$ are surfaces
with boundary $I^{\amalg n} \amalg I^{\amalg m}$ along with an embedding $d$
into $\R \times I$ such that $d^{-1}(\R \times \set{0}) = I^{\amalg n}$ and
$d^{-1}(\R \times \set{1}) = I^{\amalg m}$\footnote{There are finer details here
which will be unimportant for our purposes.}
\item $2$--morphisms are diffeomorphisms between cobordisms
(horizontal $1$--morphisms) that preserve the boundary
\end{itemize}
We then attempt to use this notion to concretely define the data of a monoidal
double functor from $\DThick$ to a suitable monoidal double category of complex
vector spaces that yields enough unitary linear transformations in the image to
facilitate quantum computing.

We define the object function $F_0$ of such a functor by assigning to each
disjoint union $X = I^{\amalg n}$ the $n$--th tensor power $F_0(X) = A^{\tensor
n}$ of some fixed algebra $A$ -- we may be specific enough to pick a consistent
bracketing pattern for $A^{\tensor n}$. It is easy to see that this assignment
is well-defined. Since the vertical $1$--morphisms are only identities, the
object category of $\DThick$ is discrete and, hence, the vertical $1$--morphism
function is the unique, obvious one: $F_0(\id_X) = \id_{F_0(X)}$.

Next, we consider the horizontal $1$--morphisms or the cobordisms
$Z : I^{\amalg n} \to I^{\amalg m}$, which are determined up to diffeomorphism
by their genus. For positive $m$ and $n$, we first consider some ``canonical''
genus $k$ cobordism $Z : I \to I$ where the holes are circles with centers along
a straight line from one boundary interval to another. This cobordism then
decomposes into a $k$--fold composition of $M * W : I \to I$ with itself, where
$M$ is the ``canonical'' pair-of-pants and $W$ is the ``canonical''
co-pair-of-pants. An example is shown below:

\[\begin{tikzpicture}[scale=0.25]
\idcobup{1, 0}
\pants{1, 4}
\pants{5, 2}
\copants{9, 2}
\pants{13, 2}
\copants{17, 2}
\pants{21, 2}
\copants{25, 2}
\end{tikzpicture}\]
In associating a linear map ``functorially'' to $Z$, it
suffices to associate linear maps to $M$ -- we can then associate a linear map
to $W$ by duality and get a linear map for $Z$ by composition.

For associating a linear map to $M$, we take a complex bundle
$\pi : E \to M$ with fibre $A$ along with a connection $\nabla$. Then, we choose
two paths: one from the mid-point of each in-boundary interval to the mid-point
of the out-boudnary interval. By parallel transport along each curve, we get two
linear maps $l_1, l_2 : A \to A$. We then have a linear map
$l : A \tensor A \to A$ given by $l(x \tensor y) = l_1(x)l_2(y)$ where the
product in the right is the algebra product in $A$. Then, the linear map
associated to $W$ is the conjugate transpose $l^{\dagger}$.
We then obtain a linear map associated to $Z$: the $k$--fold composite
$(ll^{\dagger})^k$. We set $F_1(Z) := (ll^{\dagger})^k$.

Now, consider a cobordism $Y : I^{\amalg n} \to I^{\amalg m}$ of genus $k$ such
that $Y = RZQ$, where $Q$ is a cobordism $I^{\amalg n} \to I$ formed in some
``canonical'' way by a gluing of pairs of pants and cylinders (``identity''
cobordisms) and $R$ is a cobordism $I \to I^{\amalg m}$ formed again in a
``canonical'' way from co-pairs of pants and cylinders. Of course, we associate
a linear map to the cylinder by another parallel transport along a curve between
the mid-points of its two boundaries. Then, again by composition, we get a
linear map $F_1(Y) : A^{\tensor n} \to A^{\tensor m}$.

This gives an assignment $F_1(Y)$ for a representative $Y$ of each
diffeomorphism class of cobordisms in $\DThick$. For the assignment of a linear
map to every cobordism in $\DThick$, we take the following approach. Let $Y'$ be
an arbitrary cobordism in the class of some $Y$ for which $F_1(Y)$ has been
defined as above. Then we pick a boundary preserving diffeomorphism $f \in
\Aut_{\Man}(Y)$ such that $f(Y) = Y'$\footnote{Of course, we take the underlying
topological space for each manifold in a diffeomorphism class to be the same.}.
Let $\set{\gamma_i}$ be the family of curves which along which parallel
transport gave us the linear maps $F_1(Y)$. Then, $\set{f \circ \gamma_i}$ is a
family of curves in $Y'$ which yield linear maps by parallel transport under a
connection $f\nabla$, to be made precise later. Combining these using the same
``pattern'' or ``expression'' of algebra multiplications, tensor products and
compositions as we had for $Y$, we can obtain a linear map $F_1(Y')$. Continuing
the example, we have:

\[\begin{tikzpicture}[scale=0.25]
\pidcobup{1, 0}
\ppants{1, 4}
\ppants{5, 2}
\pcopants{9, 2}
\ppants{13, 2}
\pcopants{17, 2}
\ppants{21, 2}
\pcopants{25, 2}
\end{tikzpicture}\]

We now turn our attention to the case when $m$ or $n$ is zero. We can treat the
case $n = 0$ and obtain the other case by duality. Let
$Y : \varnothing \to I^{\amalg m}$ with genus $k$. Then $Y$ decomposes as
$R * Z * Z'$ where $Z$ and $R$ are as before and $Z'$ is a genus zero cobordism
$\varnothing \to I$. We call cobordisms $\varnothing \to I$ elements and we call
elements of genus zero, atomic elements because they will not decompose into any
simpler structures. $Z'$ has a boundary preserving diffeomorphism
$f : Z'' \to Z'$ for some atomic element $Z''$ deemed ``canonical''. We
associate a linear map $a : \C \to A$ to $Z''$ as follows.  Take a loop $\gamma$
in $Z''$ on the mid-point of its only boundary interval and obtain an element
$a \in A$, or equivalently a linear map $a : \C \to A$ by parallel transport of
some fixed element $a_0 \in A$. We set $F_1(Z'') = a$ and get $F_1(Z')$ by
parallel transport of $a_0$ along $f\gamma$. $F_1(Y)$ is then obtained by
composition.

This completes the definition of an object function $F_1$ for the morphism
category of $\DThick$. Now, we turn our attention to $2$--morphisms -- boundary
preserving diffeomorphisms between cobordisms. Let $f : Y \to Y'$ be one such
diffeomorphism. We must assign to $f$ some object that functions as roughly a
``morphism of morphisms of vector spaces''. One natural choice is homotopy
classes of paths between linear functions in spaces of linear functions under
some suitable norm. In this case, we can take the following approach. There
exists a path $\psi$ in the diffeomorphism group $\Diff(Y)$ of $Y$ from $\id_Y$
to $f$ such that we can ``move'' the parallel transport machinery ``along''
$\psi$ -- that is, taking connections $\psi(t)\nabla$ and paths
$\set{\psi(t)\gamma_i}$, for $t \in [0, 1]$ -- to get linear maps for each
$\psi(t)(Y)$, which constitute a path in the space $\Hom(\dom F_1(Y), \codom
F_1(Y))$. Note that, implicit in this is the assumption that
$\psi(1)\nabla = f\nabla$ and $\set{\psi(1)\gamma_i} = \set{f\gamma_i}$ yield
the linear map $F_1(Y')$ by parallel transport, which need not hold in general.

The issue is that the $2$--morphisms are not guaranteed to ``preserve parallel
transport''. In order for this picture to make sense, we need to consider
$2$--morphisms that do this. Another way of viewing the scenario is that we are
dealing with cobordisms equipped with parallel transport machinery and these
should be the objects of our morphism category.

\end{document}

