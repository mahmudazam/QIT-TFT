
\documentclass[./Thick_TQFTs_and_Quantum_Information.tex]{subfiles}

\begin{document}

\section{Monoidal Bicategory of Cobordisms}

\subsection{The Issue with Thick Tangles}

Recall the definition of a cobordism category given in \cite{Mahmud2021}.
Following this pattern of definition, we define a category $\Thick_d$ whose
objects are smooth, orientable, compact $d$--manifolds with boundary. For any
two such objects $X, Y$, let $M$ be a $(d + 1)$--manifold with boundary $U
\amalg V$ and a pair of inclusions $X \monic[a] M$, $Y \monic[n] M$ such that
$\im a = U$, $\im b = V$ and both $a$ and $b$ are diffeomorphisms onto their
images. Let $M'$ be another such manifold with $\partial M' = U' \amalg V'$ and
inclusions $X \monic[a'] M', Y \monic[b'] M'$ satisfying the same conditions:
$\im a' = U'$, $\im b = V'$ and $a', b'$ are diffeomorphisms onto their images.
Then we say $M \eqcob M'$ if and only if there exists a diffeomorphism
$f : M \longleftrightarrow M' : f^{-1}$ satisfying:
\begin{equation}\label{diag:eqcob}
  \begin{tikzcd}
    & X \arrow[dl, "a" above left] \arrow[dr, "a'" above right] & \\
    M \arrow[rr, "f" above] &
    & M' \\
    & Y \arrow[ul, "b" below left] \arrow[ur, "b'" below right] &
  \end{tikzcd}
\end{equation}

It is straightforward to check that $\eqcob$ is an equivalence relation. Then,
if $M \in \Thick_d(X, Y)$ and $N \in \Thick_d(Y, Z)$ with boundary inclusions
$X \monic[a_M] M, Y \monic[b_M] M, Y \monic[a_N] N, Z \monic[b_N] N$, their
composite is the class of the manifold obtained by gluing $M$ and $N$ along $Y$
-- that is, $N \circ M \in \Thick_d(X, Z)$ is the pushout $M \amalg_Y N$:
\begin{equation}\label{diag:gluingdef}
  \begin{tikzcd}
    Y \arrow[r, "b_M" above] \arrow[d, "a_N" left] & M \arrow[d, "p_M" right] \\
    N \arrow[r, "p_N" below] & M \amalg_Y N := N \circ M
  \end{tikzcd}
\end{equation}
with boundary inclusions
$a_{N \circ M} = p_Ma_M : X \monic N \circ M$ and
$b_{N \circ M} = p_Nb_N : Z \monic N \circ M$. It is also straightforward to
check that $(\amalg, \varnothing)$ is a monoidal structure on $\Thick_d$.

$\Thick_2$, then, is the category of planar cobordisms or thick tangles, which
was shown in \cite{NonCommTQFT} to be the monoidal category freely generated by
by a Frobenius monoid -- the interval $I$ along with the pair-of-pants cobordism
$\fn{M}{I \otimes I}{I}$, the cap $I$ $\fn{E}{\varnothing}{I}$ and their duals,
$\fn{W}{I}{I \otimes I}$ and $\fn{C}{I}{\varnothing}$. Monoidal functors
$F : \Thick_2 \to \Vect_{\K}$ that map $I$ to a matrix algebra $\M_n$
and the pair-of-pants to matrix multiplication force all elements
$F(X) : \K \to \M_n$, for $\fn{X}{\varnothing}{I}$, to be of the form
$1 \mapsto n^{2k}I_n$, where $I_n$ is the identity matrix in $\M_n$ and $k$ is
the genus of $X$. This is because smooth manifolds with the same genus are
diffeomorphic and holes decompose as gluings $M \circ W$, as we have seen.

We would, therefore, like to relax the diffeomorphism equivalence of cobordisms.
However, when we do this, gluing is no longer associative, but it is associative
up to diffeomorphism.  This motivates us to look towards framing $\Thick_2$ as a
bicategory where $2$--morphisms $f : M \To M'$ between cobordisms $\fn{M,
M'}{X}{Y}$ are smooth maps $f : M \to M'$ making diagram \eqref{diag:eqcob}
commute. We formalize this in the next subsections.

\subsection{The Base Bicategory}

Our first observation is that we use almost none of the details of the manifold
structures on objects in $\Thick_2$ in defining its morphisms and composition.
What we use are boundary inclusions and diagrams involving boundary inclusions.
This motivates us to first abstract cobordism categories to purely categorical
constructions.

\begin{defn}[Pre-cobordism Category]
Let $\s{C}$ be a category with binary coproducts and binary pushouts.
Furthermore, let $\Ob \s{C} = O \amalg M$\footnote{This is meant to be an
abstraction of the subcategory of $\Man$ with only $d$-- and
$(d + 1)$--manifolds.} with $\s{O}$ and $\s{M}$, the full
sub-categories generated by $C_1$ and $C_2$, respectively. In addition, let
$\s{O}$ and $\s{M}$ be closed under binary coproducts and each
containing an initial object, $\varnothing_{\s{O}}$ and $\varnothing_{\s{M}}$
respectively. Then, we say that $\s{C}$ is a pre-cobordism category with object
category $\s{O}$ and morphism category $\s{M}$.
\end{defn}

\begin{exm}
As intended, the sub-category of $\Man$ generated by the $d$-- and
$(d + 1)$--manifolds is a pre-cobordism category. This will be our main example.
\end{exm}

\begin{rmk}
Note that in this definition, we have not specified the existence of "cylinders
on" objects in $\s{O}$ residing in $\s{M}$. We will address this
as we define our desired bicategory.
\end{rmk}

\begin{rmk}
We have not required $\varnothing_{\s{O}}$ and $\varnothing_{\s{M}}$ be initial
in $\s{C}$, although for our main example involving manifolds, they will be.
\end{rmk}

\begin{rmk}
We observe that each of $\s{O}$ and $\s{M}$ are cocartesian and hence have
monoidal structures $(\amalg, \varnothing_{\s{O}})$ and
$(\amalg, \varnothing_{\s{M}})$, respectively.
\end{rmk}

\begin{rmk}
We could generalize pre-cobordism categories to have
$\Ob \s{C} = \coprod_{n \in \N} C_N$ so that $\Man$ itself becomes an example
but we avoid this for simplicity.
\end{rmk}

This is enough structure\footnote{\dots we believe \dots} to define a reasonable
monoidal bicategory of cobordisms. Let $\s{C}$ be a pre-cobordism category with
object category $\s{O}$ and morphism category $\s{M}$. We define the data
of this bicategory, denoted $\s{C}^*$, as follows:

\begin{enumerate}[(i)]

\item We define a set $\Ob \s{C}^* := \Ob \s{O}$. We call the elements
of $\Ob \s{C}^*$ objects or $0$--morphisms or $0$--cells.

\item For each pair of objects $X, Y \in \Ob \s{C}^*$, a $1$--morphism or
$1$--cell $(M, a, b) : X \to Y$ is an object $M \in \s{M}$ along with
monomorphisms $a : X \monic M$ and $b : Y \monic M$ in $\s{C}$; we also write
$\fn{M}{X}{Y}$, when $a$ and $b$ are not important. We denote the set of
$1$--morphisms $X \to Y$ as $\s{C}^*_1(X, Y)$ or $\s{C}^*_{X, Y}$ or
$\s{C}^*(X, Y)$.

\item For each pair of objects $X, Y \in \Ob \s{C}^*$ and each pair of
$1$--morphisms $(M, a, b)$ and $(M', a', b')$ in $\s{C}^*_{X, Y}$, a
$2$--morphism $f : (M, a, b) \To (M', a', b')$ is a morphism $\fn{f}{M}{M'}$ in
$\s{C}$ making diagram \eqref{diag:eqcob} commute -- we include the diagram here
for completeness:
\[
  \begin{tikzcd}
    & X \arrow[dl, "a" above left] \arrow[dr, "a'" above right] & \\
    M \arrow[rr, "f" above] &
    & M' \\
    & Y \arrow[ul, "b" below left] \arrow[ur, "b'" below right] &
  \end{tikzcd}
\]
We denote the set of
$2$--morphisms $M \To M'$ as $\s{C}^*_{X, Y}(M, M')$ or $\s{C}^*_2(M, M')$ or
$\s{C}^*(M, M')$ or $\s{C}^*_{M, M'}$.

\item For objects $X, Y, Z \in \Ob \s{C}^*$ and $1$--morphisms
$M \in \s{C}^*_{X, Y}$ and $N \in \s{C}^*_{Y, Z}$, we define horizontal
composition as follows:
\begin{align*}
  - \circ_{X, Y, Z} -
    &: \s{C}^*_{Y, Z} \times \s{C}^*_{X, Y}
    &  \to
    &&  \s{C}^*_{X, Z} \\
    &: ((N, a_N, b_N), (M, a_M, b_M))
    &  \mapsto
    &&  (M \amalg_{Y} N, p_Ma_M, p_Nb_N)
\end{align*}
where $p_M$ and $p_N$ are pushed-out maps as in \eqref{diag:gluingdef}.

\item For objects $X, Y \in \Ob \s{C}^*$ and $2$--morphisms
$f : P \To Q$, $g : Q \To R$ in $\s{C}^*_{X, Y}$, we define vertical
composition as the comoposition in $\s{C}$:
\[
  (g, f) \mapsto g \circ f
\]

\end{enumerate}

It is easy to see that vertical composition is well-defined and strictly
associative. For associativity of horizontal composition up to
$1$--isomorphisms, we prove the following lemma:
\begin{lem}
Let $W, X, Y, Z$ be objects of $\s{C}^*$ with $1$--morphisms, $L : W \to X$,
$M : X \to Y$ and $N : Y \to Z$. Then, there is a $2$--isomorphism
\[
  \alpha_{L, M, N}
  : N \circ (M \circ L) = (L \amalg_X M) \amalg_Y N
  \To L \amalg_X (M \amalg_Y N) = (N \circ M) \circ L
\]
natural in $L, M, N$.
\end{lem}
\begin{proof}
We let $a_R, b_R$ be the boundary inclusions and $p_R, q_R$ for suitable
choices of $R$. For instance, $a_M$ is the boundary inclusion $X \monic M$,
$b_M$ is the boundary inclusion $Y \monic M$ while $p_{M \circ L}$ is the
pushout inclusion $L \monic M \circ L$ and $q_{M \circ L}$ is the pushout
inclusion $M \monic M \circ L$. As shown in \cite{Mahmud2021}, we get the
following pasting of pushout diagrams:
\[\begin{tikzcd}[column sep=large, row sep=large]
  & X \arrow[r, "b_{L}" above] \arrow[d, "a_{M}" left]
  & L \arrow[d, green!55!black, "p_{M \circ L}" right]
      \arrow[rdd, bend left, "p_{(N \circ M) \circ L}"] \\
  Y \arrow[r, "b_{M}" above] \arrow[d, "a_{N}" left]
  & M \arrow[r, "q_{M \circ L}" below] \arrow[d, "p_{N \circ M}" right]
  & M \circ L \arrow[dd, green!55!black, near end, "r_{N \circ (M \circ L)}"]
      \arrow[rd, dashed, red, "\alpha" description]\\
  N \arrow[r, red, "q_{N \circ M}" below]
    \arrow[rrd, bend right, "q_{N \circ (M \circ L)}"]
  & N \circ M
      \arrow[rr, near start, crossing over, red, "q_{(N \circ M) \circ L}"]
      \arrow[rd, dashed, green!55!black, "\beta" description]
  &
  & (N \circ M) \circ L
      \arrow[dl, dashed, shift right=1ex, "\phi" description] \\
  & & N \circ (M \circ L)
      \arrow[ur, dashed, shift right=1ex, "\psi" description] &
\end{tikzcd}\]
where $\alpha$ and $\beta$ are smooth maps obtained by the pushout properties
of $M \circ L$ and $N \circ M$ respectively. The green paths in the diagram
yield a smooth map $\fn{\phi}{(N \circ M) \circ L}{N \circ (M \circ L)}$ by the
pushout property of $(N \circ M) \circ L$. Similarly, the red paths yield a
smooth map $\fn{\psi}{N \circ (M \circ L)}{(N \circ M) \circ L}$. The maps
$\psi$ and $\phi$ are then shown to be $2$--morphisms as well as inverse to
each other in \cite{Mahmud2021}.

\begin{small}
\[\begin{tikzcd}[column sep=large, row sep=large]
  &   &   & X \\
  &   & M &   \\
  & Y &   &   \\
N &   &   &   \\
\end{tikzcd}\]
\end{small}

\end{proof}

\end{document}

