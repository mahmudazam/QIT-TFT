
\documentclass[./Thick_TQFTs_and_Quantum_Information.tex]{subfiles}

\begin{document}

\section{Category of Connections}

We now make precise the notion of a category of ``parallel transport machinery''
as mentioned in the previous section. For a vector bundle $\pi_E : E \to M$, we
write $\Gamma(E)$ to denote the set\footnote{We are not using any sheaf theory,
yet.} of sections of the bundle. We recall that a connection on $E$ is a
function
\[
  \nabla : \Gamma(TM \tensor E) \to \Gamma(E)
\]
satisfying some algebraic properties that are not important at the moment.
Suppose we have another bundle $\pi_{E'} : E \to M$ with a bundle morphism
$f = (u, v) : E \to E'$ -- a pair of maps  $u : M \to M'$ and $v : E \to E'$
with $v$ linear on each fibres of $E$, making the following diagram commute:
\[\begin{tikzpicture}[baseline=(a).base]
\node[scale=\diagscale] (a) at (0, 0){
\begin{tikzcd}
E \ar[d, "\pi_E" left] \ar[r, "v" above] & E' \ar[d, "\pi_{E'}" right]\\
M \ar[r, "u" below] & M'
\end{tikzcd}
};
\end{tikzpicture}\]
The derivative of $u$ induces a bundle morphism
$(du \tensor v, u) : TM \tensor E \to TM' \tensor E'$, where
$(du \tensor v)(x \tensor y) = du(x) \tensor v(x)$ for $x \in TM$ and
$y \in E$. If, in addition, $f$ has an inverse $f^{-1} = (u^{-1}, v^{-1})$,
we have an induced mapping
$\wh{f} : \Gamma(TM' \tensor E') \to \Gamma(TM \tensor E)$ given, for each $s
\in \Gamma(TM' \tensor E')$, by the composite:
\[\begin{tikzpicture}[baseline=(a).base]
\node[scale=\diagscale] (a) at (0, 0){
\begin{tikzcd}[column sep=55, row sep=large]
TM \tensor E &
TM' \tensor E' \ar[l, "d(u^{-1}) \tensor v^{-1}" above]\\
M \ar[u, "\wh{f}(s)" left] \ar[r, "u" below]
& M' \ar[u, "s" right]
\end{tikzcd}
};
\end{tikzpicture}\]
Note that $d(u^{-1}) = (du)^{-1}$ so we will unambiguously write $du^{-1}$ from
this point. Now, $\nabla(\wh{f}(s))$ is a section of $E$ while
$(f \odot \nabla)(s) := v\nabla(\wh{f}(s))u^{-1}$ is a section of $E'$:
\[\begin{tikzpicture}[baseline=(a).base]
\node[scale=\diagscale] (a) at (0, 0){
\begin{tikzcd}[column sep=large, row sep=large]
E \ar[r, "v" above] &
E'\\
M \ar[u, "\nabla(\wh{f}(s))" left] &
M' \ar[u, "(f \odot \nabla)(s)" right] \ar[l, "u^{-1}" below]
\end{tikzcd}
};
\end{tikzpicture}\]
This yields a function:
\[\begin{array}{ccccl}
f \odot \nabla
&:& \Gamma(TM' \tensor E') &\to& \Gamma(E) \\
&:& X \tensor s &\mapsto&
    v\nabla((du^{-1} \tensor v^{-1})(X \tensor s)u)u^{-1} \\
& &             & = & v\nabla(du^{-1}Xu \tensor v^{-1}su)u^{-1}
\end{array}\]
In the last equality, we used the fact that
\[
  (X \tensor s)u = Xu \tensor su
\]
which follows from the definition of the tensor product of bundles. Using this
construction, we will define our desired double category.

\subsection{Bundle Isomorphisms Transform Connections}

We wish to show that $f \odot \nabla$ is a connection. For this, we recall the
properties of a connection. Let $\pi_E : E \to M$ be a vector bundle as before
with $X, Y \in \Gamma(TM)$, $s, t \in \Gamma(E)$, $r \in C^{\infty}(M, \R)$.
Then, $\nabla : \Gamma(TM \tensor E) \to \Gamma(E)$ is a conneciton if and only
if the following hold:
\[\begin{array}{lll}
\nabla((X + Y) \tensor s) &= \nabla(X \tensor s) + \nabla(Y \tensor s)
  & (\text{Additivity in $\Gamma(TM)$}) \\
\nabla(X \tensor (s + t)) &= \nabla(X \tensor s) + \nabla(X \tensor t)
  & (\text{Additivity in $\Gamma(E)$}) \\
\nabla((r \cdot X) \tensor s) &= r \cdot \nabla(X \tensor s)
  & (\text{Scalar-multiplicativity in $\Gamma(TM)$}) \\
\nabla(X \tensor (r \cdot s)) &= dr(X) \cdot s + r \cdot \nabla(X \tensor s)
  & (\text{Leibniz property in $\Gamma(E)$}) \\
\end{array}\]
\begin{rmk}
The map $dr(X)$ in the Leibniz property is defined as follows. If $r : M \to \R$
is a smooth function, then we have $dr : TM \to T\R$, where
$\pi_{T\R} : T\R \to \R$ is the tangent bundle on $\R$. Given a section
$X : M \to TM$, we obtain a smooth map $drX : M \to T\R$. As a result, $drX$
yields a smooth map $dr(X) := \pi_{T\R}drX : M \to \R$. That is, the derivative
$dr$ yields a map
\[
  dr : \Gamma(TM) \to C^{\infty}(M, \R)
\]
which is the one in Leibniz property above.
\end{rmk}

Now, let $X, Y \in \Gamma(TM')$, $s \in \Gamma(E')$, $r \in C^{\infty}(M', \R)$.
We recall some basic identities that will be of use in showing that
$f \odot \nabla$ is a connection:
\begin{align*}
(X + Y)u &= Xu + Yu && \text{by definition of pointwise addition} \\
(r \cdot X)u &= (ru) \cdot Xu && \text{by definition of pointwise scaling} \\
v(X + Y) &= vX + vY && \text{fibre-wise linearity of $v$} \\
v(r \cdot X) &= r \cdot vX && \text{fibre-wise linearity of $v$}
\end{align*}
We note that these identities hold for arbitrary bundle morphisms $(u, v)$,
smooth real-valued maps $r$ and sections $X, Y$, as long as the operations are
well-defined.

We now proceed to check that $f \odot \nabla$ is a connection. To this end, we
first check additivity in the $\Gamma(TM')$ coordinate.
\begin{align*}
(f \odot \nabla)((X + Y) \tensor s)
=& v\nabla(du^{-1}(X + Y)u \tensor v^{-1}su)u^{-1} \\
=& v\nabla((du^{-1}Xu + du^{-1}Yu) \tensor v^{-1}su)u^{-1} \\
=& v(\nabla(du^{-1}Xu \tensor v^{-1}su)
 + \nabla(du^{-1}Yu \tensor v^{-1}su))u^{-1} \\
=& v\nabla(du^{-1}Xu \tensor v^{-1}su)u^{-1}
 + v\nabla(du^{-1}Yu \tensor v^{-1}su))u^{-1} \\
=& (f \odot \nabla)(X \tensor s) + (f \odot \nabla)(Y \tensor s)
\end{align*}
Additivity in the $\Gamma(E')$ coordinate is similar. So, we check scalar
multiplicativity in the $\Gamma(TM')$ coordinate:
\begin{align*}
(f \odot \nabla)(r \cdot X \tensor s)
=& v\nabla(du^{-1}(r \cdot X)u \tensor v^{-1}su)u^{-1} \\
=& v\nabla(du^{-1}(ru \cdot Xu) \tensor v^{-1}su)u^{-1} \\
=& v\nabla((ru \cdot (du^{-1}Xu)) \tensor v^{-1}su)u^{-1} \\
=& v (ru \cdot \nabla(du^{-1}Xu \tensor v^{-1}su))u^{-1} \\
=& v (ruu^{-1} \cdot \nabla(du^{-1}Xu \tensor v^{-1}su)u^{-1}) \\
=& r \cdot (v\nabla(du^{-1}Xu \tensor v^{-1}su)u^{-1}) \\
=& r \cdot ((f \odot \nabla)(X \tensor s))
\end{align*}

We finally check the Liebnitz rule.
\begin{align*}
(f \odot \nabla)(X \tensor (r \cdot s))
=& v\nabla(du^{-1}Xu \tensor v^{-1}(r \cdot s)u)u^{-1} \\
=& v(\nabla(du^{-1}Xu \tensor v^{-1}(ru \cdot su))u^{-1} \\
=& v\nabla(du^{-1}Xu \tensor ru \cdot (v^{-1}su))u^{-1} \\
=& v(
      d(ru)(du^{-1}Xu) \cdot v^{-1}su
      + ru \cdot \nabla(du^{-1}Xu \tensor v^{-1}su)
    )u^{-1} \\
=& v(d(ru)(du^{-1}Xu) \cdot v^{-1}su)u^{-1}
 + v(ru \cdot \nabla(du^{-1}Xu \tensor v^{-1}su))u^{-1}
\end{align*}
It now suffices to show that the left summand is
\[
  dr(X) \cdot s
\]
and the right summand is
\[
  r \cdot (f \odot \nabla)(X \tensor s)
\]
For the right summand, we observe:
\begin{align*}
 & v(ru \cdot \nabla(du^{-1}Xu \tensor v^{-1}su))u^{-1} \\
=& v(ruu^{-1} \cdot \nabla(du^{-1}Xu \tensor v^{-1}su)u^{-1}) \\
=& v(r \cdot \nabla(du^{-1}Xu \tensor v^{-1}su)u^{-1}) \\
=&r \cdot (v\nabla(du^{-1}Xu \tensor v^{-1}su)u^{-1}) \\
=& r \cdot (f \odot \nabla)(X \tensor s)
\end{align*}
For the left summand, we first observe a useful property of the derivative
operator $d-$. If $g : L \to M$ and $h : M \to N$ are smooth maps, then by the
pasting of pushout diagrams to give pushout diagrams, we have
\[
  d(h \circ g) = dh \circ dg
\]
Then, we have:
\begin{align*}
v(d(ru)(du^{-1}Xu) \cdot v^{-1}su)u^{-1}
=& v(\pi_{T\R}d(ru)(du^{-1}Xu) \cdot v^{-1}su)u^{-1}\\
=& v(\pi_{T\R}d(ruu^{-1})Xu \cdot v^{-1}su)u^{-1}\\
=& v(\pi_{T\R}drXu \cdot v^{-1}su)u^{-1}\\
=& v((\pi_{T\R}drX \cdot v^{-1}s)u)u^{-1}\\
=& v(v^{-1}(\pi_{T\R}drX \cdot s)u)u^{-1}\\
=& dr(X) \cdot s
\end{align*}
as required. This completes the proof of the following theorem.
\begin{thm}
Let $\pi : E \to M$, $\pi' : E' \to M'$ be bundles with a bundle isomorphism
$f = (u, v) : E \to E'$. If $\nabla$ is a connection on $\pi$, then
$f \odot \nabla$ is a connection on $\pi'$.
\end{thm}
\begin{defn}
We call $f \odot \nabla$ the shift of $\nabla$ along $f$ and say that $f$
takes $\nabla$ to $f \odot \nabla$. From this point, we write $f\nabla$ as
opposed to $f \odot \nabla$, whenever there is no confusion.
\end{defn}

\subsection{Category of Connections}

Let $\pi_i : E_i \to M_i$ be a bundles equipped with a connections $\nabla_i$
for $i \in \set{1, 2, 3, 4}$. Let $f_{i, i + 1} = (u_{i, i + 1}, v_{i, i + 1})$
be bundle isomorphisms for $i \in \set{1, 2, 3}$. Then, the composite
$f_{1, 3} := f_{2, 3}f_{1, 2} = (u_{2, 3}u_{1, 2}, v_{2, 3}v_{1, 2})
=: (u_{1, 3}, v_{1, 3})$ is clearly a bundle isomorphism. We similarly define
composites $f_{i, j}$ for each $i < j \in \set{1, 2, 3, 4}$. Now, suppose
$\nabla_{i + 1} = f_{i, i + 1}\nabla_i$ for each $i \in \set{1, 2, 3}$. Then, we
immediately have
\[
  f_{i, k}\nabla_i = f_{j, k}f_{i, j}\nabla_i = f_{j, k}\nabla_j = \nabla_k
\]
for each $i < j < k$ in $\set{1, 2, 3, 4}$. In particular,
\[
  f_{3, 4}(f_{2, 3}f_{1, 2}\nabla_1) = \nabla_4
    = (f_{3, 4}f_{2, 3})f_{1, 2}\nabla_1
\]

We then observe the action of identity bundle morphisms. The identity bundle
morphism on $\pi_1$ is the pair $\id_{\pi_1} = (\id_{E_1}, \id_{M_1})$. Then,
\begin{align*}
   \id_{\pi_1}\nabla_1(X \tensor s)
&= \id_{E_1}\nabla_1(d(\id_{M_1})^{-1}X\id_{M_1}
                     \tensor \id_{E_1}^{-1}s\id_{M_1})\id_{M_1}^{-1} \\
&= \nabla_1(\id_{TM_1}^{-1}X \tensor s) \\
&= \nabla_1(X \tensor s)
\end{align*}
so that $\id_{\pi_1}\nabla = \nabla$. We finally observe that
$ff^{-1}\nabla = \id\nabla = \nabla$ for any connection $\nabla$ and any
compatible diffeomorphism $f$.

These observations motivate the following definition.
\begin{defn}
Let $\pi_1$ and $\pi_2$ be bundles equipped with connections $\nabla_1$ and
$\nabla_2$ respectively. Then, a bundle isomorphism
$f = (u, v) : \pi_1 \to \pi_2$ satisfying $f\nabla_1 = \nabla_2$ is called an
isomorphism, or simply morphism, of connections.
\end{defn}

From the work above, we have established the following results.
\begin{thm}[Category of Connections]
There exists a groupoid whose objects are connections and whose morphisms are
isomorphisms of connections.
\end{thm}

We observe that the category of connections has a monoidal structure. Let
$\nabla_1$ and $\nabla_2$ be two connections with underlying bundles
$\pi_1 : E_1 \to M_1$ and $\pi_2 : E_2 \to M_2$ respectively. Then, the
coproduct or disjoint union of these bundles is given by
$\pi_1 \amalg \pi_2 : E_1 \amalg E_2 \to M_1 \amalg M_2$, the transition
functions of which are defined component-wise. Then, we can also define a
function of sets $\nabla := \nabla_1 \amalg \nabla_2$. However, component-wise,
$\nabla$ satisfies the relations of a connection and is hence a connection.

\TODO{Show that there exist natural associators and unitors for this disjoint
union of connections. Does it even work for bundles?}

\end{document}

