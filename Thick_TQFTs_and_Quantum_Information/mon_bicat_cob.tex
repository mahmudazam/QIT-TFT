
\documentclass[./Thick_TQFTs_and_Quantum_Information.tex]{subfiles}

\begin{document}

\section{Category of Connections}

We now make precise the notion of a category of ``parallel transport machinery''
as mentioned in the previous section. For a vector bundle $\pi_E : E \to M$, we
write $\Gamma(E)$ to denote the set\footnote{We are not using any sheaf theory,
yet.} of sections of the bundle. We recall that a connection on $E$ is a
function
\[
  \nabla : \Gamma(TM \tensor E) \to \Gamma(E)
\]
satisfying some algebraic properties that are not important at the moment.
Suppose we have another bundle $\pi_{E'} : E \to M$ with a bundle morphism
$f = (u, v) : E \to E'$ -- a pair of maps  $u : M \to M'$ and $v : E \to E'$
with $v$ linear on each fibres of $E$, making the following diagram commute:
\[\begin{tikzpicture}[baseline=(a).base]
\node[scale=\diagscale] (a) at (0, 0){
\begin{tikzcd}
E \ar[d, "\pi_E" left] \ar[r, "v" above] & E' \ar[d, "\pi_{E'}" right]\\
M \ar[r, "u" below] & M'
\end{tikzcd}
};
\end{tikzpicture}\]
The derivative of $u$ induces a bundle morphism
$(du \tensor v, u) : TM \tensor E \to TM' \tensor E'$, where
$(du \tensor v)(x \tensor y) = du(x) \tensor v(x)$ for $x \in TM$ and
$y \in E$. If, in addition, $f$ has an inverse $f^{-1} = (u^{-1}, v^{-1})$,
we have an induced mapping
$\wh{f} : \Gamma(TM' \tensor E') \to \Gamma(TM \tensor E)$ given, for each $s
\in \Gamma(TM' \tensor E')$, by the composite:
\[\begin{tikzpicture}[baseline=(a).base]
\node[scale=\diagscale] (a) at (0, 0){
\begin{tikzcd}[column sep=55, row sep=large]
TM \tensor E &
TM' \tensor E' \ar[l, "d(u^{-1}) \tensor v^{-1}" above]\\
M \ar[u, "\wh{f}(s)" left] \ar[r, "u" below]
& M' \ar[u, "s" right]
\end{tikzcd}
};
\end{tikzpicture}\]
Note that $d(u^{-1}) = (du)^{-1}$ so we will unambiguously write $du^{-1}$ from
this point. Now, $\nabla(\wh{f}(s))$ is a section of $E$ while
$(f \odot \nabla)(s) := v\nabla(\wh{f}(s))u^{-1}$ is a section of $E'$:
\[\begin{tikzpicture}[baseline=(a).base]
\node[scale=\diagscale] (a) at (0, 0){
\begin{tikzcd}[column sep=large, row sep=large]
E \ar[r, "v" above] &
E'\\
M \ar[u, "\nabla(\wh{f}(s))" left] &
M' \ar[u, "(f \odot \nabla)(s)" right] \ar[l, "u^{-1}" below]
\end{tikzcd}
};
\end{tikzpicture}\]
This yields a function:
\[\begin{array}{ccccl}
f \odot \nabla
&:& \Gamma(TM' \tensor E') &\to& \Gamma(E) \\
&:& X \tensor s &\mapsto&
    v\nabla((du^{-1} \tensor v^{-1})(X \tensor s)u)u^{-1} \\
& &             & = & v\nabla(du^{-1}Xu \tensor v^{-1}su)u^{-1}
\end{array}\]
In the last equality, we used the fact that
\[
  (X \tensor s)u = Xu \tensor su
\]
which follows from the definition of the tensor product of bundles. Using this
construction, we will define our desired double category.

\subsection{Bundle Isomorphisms Transform Connections}

We wish to show that $f \odot \nabla$ is a connection. For this, we recall the
properties of a connection. Let $\pi_E : E \to M$ be a vector bundle as before
with $X, Y \in \Gamma(TM)$, $s, t \in \Gamma(E)$, $r \in C^{\infty}(M, \R)$.
Then, $\nabla : \Gamma(TM \tensor E) \to \Gamma(E)$ is a conneciton if and only
if the following hold:
\[\begin{array}{lll}
\nabla((X + Y) \tensor s) &= \nabla(X \tensor s) + \nabla(Y \tensor s)
  & (\text{Additivity in $\Gamma(TM)$}) \\
\nabla(X \tensor (s + t)) &= \nabla(X \tensor s) + \nabla(X \tensor t)
  & (\text{Additivity in $\Gamma(E)$}) \\
\nabla((r \cdot X) \tensor s) &= r \cdot \nabla(X \tensor s)
  & (\text{Scalar-multiplicativity in $\Gamma(TM)$}) \\
\nabla(X \tensor (r \cdot s)) &= dr(X) \cdot s + r \cdot \nabla(X \tensor s)
  & (\text{Leibniz property in $\Gamma(E)$}) \\
\end{array}\]
\begin{rmk}
The map $dr(X)$ in the Leibniz property is defined as follows. If $r : M \to \R$
is a smooth function, then we have $dr : TM \to T\R$, where
$\pi_{T\R} : T\R \to \R$ is the tangent bundle on $\R$. Given a section
$X : M \to TM$, we obtain a smooth map $drX : M \to T\R$. As a result, $drX$
yields a smooth map $dr(X) := \pi_{T\R}drX : M \to \R$. That is, the derivative
$dr$ yields a map
\[
  dr : \Gamma(TM) \to C^{\infty}(M, \R)
\]
which is the one in Leibniz property above.
\end{rmk}

Now, let $X, Y \in \Gamma(TM')$, $s \in \Gamma(E')$, $r \in C^{\infty}(M', \R)$.
We recall some basic identities that will be of use in showing that
$f \odot \nabla$ is a connection:
\begin{align*}
(X + Y)u &= Xu + Yu && \text{by definition of pointwise addition} \\
(r \cdot X)u &= (ru) \cdot Xu && \text{by definition of pointwise scaling} \\
v(X + Y) &= vX + vY && \text{fibre-wise linearity of $v$} \\
v(r \cdot X) &= r \cdot vX && \text{fibre-wise linearity of $v$}
\end{align*}
We note that these identities hold for arbitrary bundle morphisms $(u, v)$,
smooth real-valued maps $r$ and sections $X, Y$, as long as the operations are
well-defined.

We now proceed to check that $f \odot \nabla$ is a connection. To this end, we
first check additivity in the $\Gamma(TM')$ coordinate.
\begin{align*}
(f \odot \nabla)((X + Y) \tensor s)
=& v\nabla(du^{-1}(X + Y)u \tensor v^{-1}su)u^{-1} \\
=& v\nabla((du^{-1}Xu + du^{-1}Yu) \tensor v^{-1}su)u^{-1} \\
=& v(\nabla(du^{-1}Xu \tensor v^{-1}su)
 + \nabla(du^{-1}Yu \tensor v^{-1}su))u^{-1} \\
=& v\nabla(du^{-1}Xu \tensor v^{-1}su)u^{-1}
 + v\nabla(du^{-1}Yu \tensor v^{-1}su))u^{-1} \\
=& (f \odot \nabla)(X \tensor s) + (f \odot \nabla)(Y \tensor s)
\end{align*}
Additivity in the $\Gamma(E')$ coordinate is similar. So, we check scalar
multiplicativity in the $\Gamma(TM')$ coordinate:
\begin{align*}
(f \odot \nabla)(r \cdot X \tensor s)
=& v\nabla(du^{-1}(r \cdot X)u \tensor v^{-1}su)u^{-1} \\
=& v\nabla(du^{-1}(ru \cdot Xu) \tensor v^{-1}su)u^{-1} \\
=& v\nabla((ru \cdot (du^{-1}Xu)) \tensor v^{-1}su)u^{-1} \\
=& v (ru \cdot \nabla(du^{-1}Xu \tensor v^{-1}su))u^{-1} \\
=& v (ruu^{-1} \cdot \nabla(du^{-1}Xu \tensor v^{-1}su)u^{-1}) \\
=& r \cdot (v\nabla(du^{-1}Xu \tensor v^{-1}su)u^{-1}) \\
=& r \cdot ((f \odot \nabla)(X \tensor s))
\end{align*}

We finally check the Liebnitz rule.
\begin{align*}
(f \odot \nabla)(X \tensor (r \cdot s))
=& v\nabla(du^{-1}Xu \tensor v^{-1}(r \cdot s)u)u^{-1} \\
=& v(\nabla(du^{-1}Xu \tensor v^{-1}(ru \cdot su))u^{-1} \\
=& v\nabla(du^{-1}Xu \tensor ru \cdot (v^{-1}su))u^{-1} \\
=& v(
      d(ru)(du^{-1}Xu) \cdot v^{-1}su
      + ru \cdot \nabla(du^{-1}Xu \tensor v^{-1}su)
    )u^{-1} \\
=& v(d(ru)(du^{-1}Xu) \cdot v^{-1}su)u^{-1}
 + v(ru \cdot \nabla(du^{-1}Xu \tensor v^{-1}su))u^{-1}
\end{align*}
It now suffices to show that the left summand is
\[
  dr(X) \cdot s
\]
and the right summand is
\[
  r \cdot (f \odot \nabla)(X \tensor s)
\]
For the right summand, we observe:
\begin{align*}
 & v(ru \cdot \nabla(du^{-1}Xu \tensor v^{-1}su))u^{-1} \\
=& v(ruu^{-1} \cdot \nabla(du^{-1}Xu \tensor v^{-1}su)u^{-1}) \\
=& v(r \cdot \nabla(du^{-1}Xu \tensor v^{-1}su)u^{-1}) \\
=&r \cdot (v\nabla(du^{-1}Xu \tensor v^{-1}su)u^{-1}) \\
=& r \cdot (f \odot \nabla)(X \tensor s)
\end{align*}
For the left summand, we first observe a useful property of the derivative
operator $d-$. If $g : L \to M$ and $h : M \to N$ are smooth maps, then by the
pasting of pushout diagrams to give pushout diagrams, we have
\[
  d(h \circ g) = dh \circ dg
\]
Then, we have:
\begin{align*}
v(d(ru)(du^{-1}Xu) \cdot v^{-1}su)u^{-1}
=& v(\pi_{T\R}d(ru)(du^{-1}Xu) \cdot v^{-1}su)u^{-1}\\
=& v(\pi_{T\R}d(ruu^{-1})Xu \cdot v^{-1}su)u^{-1}\\
=& v(\pi_{T\R}drXu \cdot v^{-1}su)u^{-1}\\
=& v((\pi_{T\R}drX \cdot v^{-1}s)u)u^{-1}\\
=& v(v^{-1}(\pi_{T\R}drX \cdot s)u)u^{-1}\\
=& dr(X) \cdot s
\end{align*}
as required. This completes the proof of the following theorem.
\begin{thm}
Let $\pi : E \to M$, $\pi' : E' \to M'$ be bundles with a bundle isomorphism
$f = (u, v) : E \to E'$. If $\nabla$ is a connection on $\pi$, then
$f \odot \nabla$ is a connection on $\pi'$.
\end{thm}
\begin{defn}
We call $f \odot \nabla$ the shift of $\nabla$ along $f$ and say that $f$
takes $\nabla$ to $f \odot \nabla$. From this point, we write $f\nabla$ as
opposed to $f \odot \nabla$, whenever there is no confusion.
\end{defn}

\subsection{Category of Connections}

Let $\pi_i : E_i \to M_i$ be a bundles equipped with a connections $\nabla_i$
for $i \in \set{1, 2, 3, 4}$. Let $f_{i, i + 1} = (u_{i, i + 1}, v_{i, i + 1})$
be bundle isomorphisms for $i \in \set{1, 2, 3}$. Then, the composite
$f_{1, 3} := f_{2, 3}f_{1, 2} = (u_{2, 3}u_{1, 2}, v_{2, 3}v_{1, 2})
=: (u_{1, 3}, v_{1, 3})$ is clearly a bundle isomorphism. We similarly define
composites $f_{i, j}$ for each $i < j \in \set{1, 2, 3, 4}$. Now, suppose
$\nabla_{i + 1} = f_{i, i + 1}\nabla_i$ for each $i \in \set{1, 2, 3}$. Then, we
immediately have
\[
  f_{i, k}\nabla_i = f_{j, k}f_{i, j}\nabla_i = f_{j, k}\nabla_j = \nabla_k
\]
for each $i < j < k$ in $\set{1, 2, 3, 4}$. In particular,
\[
  f_{3, 4}(f_{2, 3}f_{1, 2}\nabla_1) = \nabla_4
    = (f_{3, 4}f_{2, 3})f_{1, 2}\nabla_1
\]

We then observe the action of identity bundle morphisms. The identity bundle
morphism on $\pi_1$ is the pair $\id_{\pi_1} = (\id_{E_1}, \id_{M_1})$. Then,
\begin{align*}
   \id_{\pi_1}\nabla_1(X \tensor s)
&= \id_{E_1}\nabla_1(d(\id_{M_1})^{-1}X\id_{M_1}
                     \tensor \id_{E_1}^{-1}s\id_{M_1})\id_{M_1}^{-1} \\
&= \nabla_1(\id_{TM_1}^{-1}X \tensor s) \\
&= \nabla_1(X \tensor s)
\end{align*}
so that $\id_{\pi_1}\nabla = \nabla$. We finally observe that
$ff^{-1}\nabla = \id\nabla = \nabla$ for any connection $\nabla$ and any
compatible diffeomorphism $f$.

These observations motivate the following definition.
\begin{defn}
Let $\pi_1$ and $\pi_2$ be bundles equipped with connections $\nabla_1$ and
$\nabla_2$ respectively. Then, a bundle isomorphism
$f = (u, v) : \pi_1 \to \pi_2$ satisfying $f\nabla_1 = \nabla_2$ is called an
isomorphism, or simply morphism, of connections.
\end{defn}

From the work above, we have established the following results.
\begin{thm}
There exists a groupoid, denoted $\Conn$, whose objects are connections and
whose morphisms are isomorphisms of connections.
\end{thm}
\begin{defn}[Category of Connections]
We will call the category of the above function the category or groupoid of
connections. We will denote this category $\Conn$.
\end{defn}

We now try to establish a monoidal structure for a subcategory of the category
of connections. Let $\nabla_1$ and $\nabla_2$ be two connections with underlying
bundles $\pi_1 : E_1 \to M_1$ and $\pi_2 : E_2 \to M_2$ respectively. We will
consider the coproduct or disjoint union of these bundles in the category of
manifolds. There exists a smooth map
$\pi_1 \amalg \pi_2 : E_1 \amalg E_2 \to M_1 \amalg M_2$ which we will give the
structure of a vector bundle as follows. For this, we additionally assume that
the fibres of $E_1$ and $E_2$ are the same vector space. Let
$U = U_1 \amalg U_2, V = V_1 \amalg V_2$ be open sets in $M_1 \amalg M_2$ with
$U_i, V_i \subset M_i$ for $i \in \set{1, 2}$, and consider
$(U_1 \amalg U_2) \cap (V_1 \amalg V_2) = (U_1 \cap V_1) \amalg (U_2 \cap V_2)$.
We have a transition function $G_{U_1, V_1}$ on $U_1 \cap V_1$ from the bundle
$\pi_1$ and one $H_{U_2, V_2}$ on $U_2 \cap V_2$ from $\pi_2$. We define a
function $(G \amalg H)_{U, V} : U \cap V \to \GL_n(\C)$ piecewise, as
follows:
\[
  (G \amalg H)_{U, V}(x) := \begin{cases}
    G_{U_1, V_1}(x), & x \in U_1 \cap V_1 \subset M_1 \\
    H_{U_2, V_2}(x), & x \in U_2 \cap V_2 \subset M_2
  \end{cases}
\]
which is smooth since it is a disjoint union of smooth functions. Therefore,
\[
  G \amalg H := \set[(G \amalg H)_{U, V}]
                    {U, V \subset M_1 \amalg M_2 \text{ are open}}
\]
is a vector bundle structure on $\pi_1 \amalg \pi_2$. A section of
$E_1 \amalg E_2$ is a smooth map
\[
  s : M_1 \amalg M_2 \to E_1 \amalg E_2
\]
satisfying $(\pi_1 \amalg \pi_2)s = \id_{M_1 \amalg M_2}$. We note that this
guarantees that the $s = s_1 \amalg s_2$ where $s_i$ is a section of
$E_i$, $i \in \set{1, 2}$.

Similarly, $TM_1 \amalg TM_2 \to M_1 \amalg M_2$ is a vector bundle when $M_1$
and $M_2$ have the same dimension, and we can take this to be the definition of
the tangent bundle $T(M_1 \amalg M_2)$ on $M_1 \amalg M_2$. Now, let
$\pi_3 : E_3 \to M_3$ be another bundle where all the $E_i$ have the same fibres
and all the $M_i$ are equidimensional.

We can pick a convention for disjoint unions of sets as follows:
\[
  A \amalg B = (A \times \set{0}) \cup (B \times \set{1})
\]
Under this convention,
\[
  E_1 \amalg (E_2 \amalg E_3)
    = \set[(x_1, 0)]{x_1 \in E_1}
      \cup \set[((x_2, 0), 1)]{x_2 \in E_2}
      \cup \set[((x_3, 1), 1)]{x_3 \in E_3}
\]
and
\[
  (E_1 \amalg E_2) \amalg E_3
    = \set[((x_1, 0), 0)]{x_1 \in E_1}
      \cup \set[((x_2, 1), 0)]{x_2 \in E_2}
      \cup \set[(x_3, 1)]{x_3 \in E_3}
\]
We have similar descriptions for the two distinct parenthesizations for
$M_1 \amalg M_2 \amalg M_3$. Now, the map
\[
  \alpha_{E_1, E_2, E_3} : E_1 \amalg (E_2 \amalg E_3)
                           \to (E_1 \amalg E_2) \amalg E_3
\]
defined by
\[
  (x_1, 0) \mapsto ((x_1, 0), 0),
  ((x_2, 0), 1) \mapsto ((x_2, 1), 0),
  ((x_3, 1), 1) \mapsto (x_3, 1)
\]
is easily seen to be bijective and fibre-preserving. Smoothness and naturality
in the subscripts follow from those of associators in $\Man$. We can make a
similar argument for similarly defined unitors $\rho_E$ and $\lambda_E$. We thus
have the following theorem.
\begin{thm}
The subcategory of the category of bundles formed by base spaces of a fixed
dimension $d$ and total spaces with equal fibres is monoidal under the disjoint
union of manifolds.
\end{thm}
\begin{defn}[Category of {$(V, d)$--bundles}]
The subcategory of the category of bundles in the above theorem is called the
category of $V$--fibred bundles on $d$--dimensional manifolds or of
$(V, d)$--bundles and is denoted $\Bund^V_d$.
\end{defn}

We now define a function
\[
  \nabla_1 \amalg \nabla_2
    : \Gamma(T(M_1 \amalg M_2) \tensor E_1 \amalg E_2)
    \to \Gamma(E_1 \amalg E_2)
\]
as follows, for $i \in \set{0, 1}$:
\[
  (\nabla_1 \amalg \nabla_2)((X_1 \amalg X_2) \tensor (s_1 \amalg s_2))(x, i)
    = \nabla_{i + 1}(X_{i + 1} \tensor s_{i + 1})(x)
\]
It is easy to see that this function satisfies the connection identities
piecewise so that it satisfies these identities on its entire domain. Thus,
$\nabla_1 \amalg \nabla_2$ is a connection. We then consider a connection
$\nabla_3$ on $\pi_3$. Letting
$f = (u, v) = (\alpha_{M_1, M_2, M_3}, \alpha_{E_1, E_2, E_3})$, we now wish to
verify that
\[
  f(\nabla_1 \amalg (\nabla_2 \amalg \nabla_2))
    = (\nabla_1 \amalg \nabla_2) \amalg \nabla_3
\]
For this, we will need to inspect the expression:
\begin{align*}
  & f(\nabla_1 \amalg (\nabla_2 \amalg \nabla_2))(
      (X_1 \amalg X_2) \amalg X_3
      \tensor (s_1 \amalg s_2) \amalg s_3
      ) \\
  =& v(\nabla_1 \amalg (\nabla_2 \amalg \nabla_3))\br{
    du^{-1}((X_1 \amalg X_2) \amalg X_3)u
    \tensor v^{-1}((s_1 \amalg s_2) \amalg s_3)u
  }u^{-1} \\
  =& v(\nabla_1 \amalg (\nabla_2 \amalg \nabla_3))\br{
    du^{-1}((X_1 \amalg X_2) \amalg X_3)u
    \tensor (s_1 \amalg (s_2 \amalg s_3))
  }u^{-1}
\end{align*}
We then observe the following basic fact.
\begin{lem}
For bundles $\pi : E_i \to M_i$, $i \in \set{1, 2, 3}$, we have
$d\alpha_{M_1, M_2, M_3} = \alpha_{E_1, E_2, E_3}$.
\end{lem}
\begin{proof}
We denote $\alpha_{TM} := \alpha_{TM_1, TM_2, TM_3}$,
$\alpha_{M} := \alpha_{M_1, M_2, M_3}$.
Let $R$ be a manifold, and $x$ and $y$, smooth maps making the following
diagram commute:
\[\begin{tikzpicture}[baseline=(a).base]
\node[scale=\diagscale] (a) at (0, 0){
\begin{tikzcd}[column sep=large, row sep=huge]
&
(TM_1 \amalg TM_2) \amalg TM_3
  \ar[d, "\pi_{(12)3}" description]
  \ar[rrd, "x" above right] & \\
TM_1 \amalg (TM_2 \amalg TM_3)
  \ar[ur, "\alpha_{TM}" above left]
  \ar[dr, "\pi_{1(23)}" below left] &
(M_1 \amalg M_2) \amalg M_3
  \ar[rr, "y\alpha_{M}^{-1}" description, dashed] & &
R \\ &
M_1 \amalg (M_2 \amalg M_3)
  \ar[u, "\alpha_{M}" description]
  \ar[urr, "y" below right] &
\end{tikzcd}
};
\end{tikzpicture}\]
The map $y\alpha_M^{-1}$ satisfies $(y\alpha_M^{-1})\alpha_M = y$ and
$(y\alpha_M^{-1})\pi_{(12)3} = y\pi_{1(23)}\alpha_{TM} = x$. Suppose another map
$r$ satisfies $r\alpha_M = y$. Then, $r = y\alpha_M^{-1}$ showing that
$y\alpha_M^{-1}$ is unique.

Therefore, $\alpha_{TM}$ makes part of the diagram excluding $R$ a pushout and
hence must be $d\alpha_M$, by the uniqueness of the derivative.
\end{proof}

The above theorem yields:
\begin{align*}
  & f(\nabla_1 \amalg (\nabla_2 \amalg \nabla_2))(
      (X_1 \amalg X_2) \amalg X_3
      \tensor (s_1 \amalg s_2) \amalg s_3
      ) \\
  =& v(\nabla_1 \amalg (\nabla_2 \amalg \nabla_3))\br{
    du^{-1}((X_1 \amalg X_2) \amalg X_3)u
    \tensor (s_1 \amalg (s_2 \amalg s_3))
  }u^{-1}\\
  =& v(\nabla_1 \amalg (\nabla_2 \amalg \nabla_3))\br{
    (X_1 \amalg (X_2 \amalg X_3))
    \tensor (s_1 \amalg (s_2 \amalg s_3))
  }u^{-1} \\
  =& \alpha_{TM_1, TM_2, TM_3}(\nabla_1 \amalg (\nabla_2 \amalg \nabla_3))\br{
    (X_1 \amalg (X_2 \amalg X_3))
    \tensor (s_1 \amalg (s_2 \amalg s_3))
  }\alpha_{M_1, M_2, M_3}^{-1}
\end{align*}
where the last expression is easily seen to be
\[
  ((\nabla_1 \amalg \nabla_2) \amalg \nabla_3)(
    (X_1 \amalg X_2) \amalg X_3 \tensor (s_1 \amalg s_2) \amalg s_3
  )
\]
We can similarly show that the unitors in $\Man$ yield unitors for disjoint
unions of connections of equidimensional bundles with equal fibres.
We have thus proved the following theorem.
\begin{thm}
For a vector space $V$ and a non-negative integer $d$, the subcategory of the
category of connections consisting of all connections on objects in
$\Bund_d^{V}$ and all morphisms of connections between them is a monoidal
category under disjoint union.
\end{thm}
\begin{defn}[Category of {$(V, d)$}--Connections]
We call the subcategory of the category of connections in the above theorem the
category of connections on $V$--fibred bundles on $d$--dimensional manifolds or
of $(V, d)$--connections. We denote this category $\Conn^V_d$.
\end{defn}

\subsection{Quantum Information Theoretic Cobordisms}

In order to obtain linear maps by parallel transport on manifolds, we need one
last piece on top of connections. These are collections of paths on manifolds
along which we will parallel transport vectors in the fibres of a bundle with
connection. We shall now formalize this apparatus into yet another category.

\end{document}

