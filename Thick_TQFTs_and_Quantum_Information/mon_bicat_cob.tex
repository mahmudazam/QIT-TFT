
\documentclass[./Thick_TQFTs_and_Quantum_Information.tex]{subfiles}

\begin{document}


\subsection*{The Issue with Thick Tangles}

Recall the definition of a cobordism category given in \cite{Mahmud2021}.
Following this pattern of definition, we define a category $\Thick_d$ whose
objects are smooth, orientable, compact $d$--manifolds with boundary. For any
two such objects $X, Y$, let $M$ be a $(d + 1)$--manifold with boundary $U
\amalg V$ and a pair of inclusions $X \monic[a] M$, $Y \monic[n] M$ such that
$\im a = U$, $\im b = V$ and both $a$ and $b$ are diffeomorphisms onto their
images. Let $M'$ be another such manifold with $\partial M' = U' \amalg V'$ and
inclusions $X \monic[a'] M', Y \monic[b'] M'$ satisfying the same conditions:
$\im a' = U'$, $\im b = V'$ and $a', b'$ are diffeomorphisms onto their images.
Then we say $M \eqcob M'$ if and only if there exists a diffeomorphism
$f : M \longleftrightarrow M' : f^{-1}$ satisfying:
\begin{equation}\label{diag:eqcob}
\begin{tikzpicture}[baseline=(a).base]
\node[scale=\diagscale] (a) at (0, 0){
\begin{tikzcd}
  & X \arrow[dl, "a" above left] \arrow[dr, "a'" above right] & \\
  M \arrow[rr, "f" above] &
  & M' \\
  & Y \arrow[ul, "b" below left] \arrow[ur, "b'" below right] &
\end{tikzcd}
};
\end{tikzpicture}
\end{equation}
In fact, making this diagram commute is our definition of preserving boundaries.
\begin{defn}
Given $X, Y, M, M', a, a', b, b'$ as above, a smooth map $f : M \to M'$ is said
to be boundary-preserving if it makes diagram \eqref{diag:eqcob} commute.
\end{defn}

It is straightforward to check that $\eqcob$ is an equivalence relation. Then,
if $M \in \Thick_d(X, Y)$ and $N \in \Thick_d(Y, Z)$ with boundary inclusions
$X \monic[a_M] M, Y \monic[b_M] M, Y \monic[a_N] N, Z \monic[b_N] N$, their
composite is the class of the manifold obtained by gluing $M$ and $N$ along $Y$
-- that is, $N \circ M \in \Thick_d(X, Z)$ is the pushout $M \amalg_Y N$:
\begin{equation}\label{diag:gluingdef}
\begin{tikzpicture}[baseline=(a).base]
\node[scale=\diagscale] (a) at (0, 0){
\begin{tikzcd}
  Y \arrow[r, "b_M" above] \arrow[d, "a_N" left] &
  M \arrow[d, "p_{NM}" right] \\
  N \arrow[r, "q_{NM}" below] &
  M \amalg_Y N := N \circ M
\end{tikzcd}
};
\end{tikzpicture}
\end{equation}
with boundary inclusions
$a_{N \circ M} = p_{NM}a_M : X \monic N \circ M$ and
$b_{N \circ M} = q_{NM}b_N : Z \monic N \circ M$. It is also straightforward to
check that $(\amalg, \varnothing)$ is a monoidal structure on $\Thick_d$.

$\Thick_2$, then, is the category of planar cobordisms or thick tangles, which
was shown in \cite{NonCommTQFT} to be the monoidal category freely generated by
by a Frobenius monoid -- the interval $I$ along with the pair-of-pants cobordism
$\fn{M}{I \otimes I}{I}$, the cap $I$ $\fn{E}{\varnothing}{I}$ and their duals,
$\fn{W}{I}{I \otimes I}$ and $\fn{C}{I}{\varnothing}$. Monoidal functors
$F : \Thick_2 \to \Vect_{\K}$ that map $I$ to a matrix algebra $\M_n$
and the pair-of-pants to matrix multiplication force all elements
$F(X) : \K \to \M_n$, for $\fn{X}{\varnothing}{I}$, to be of the form
$1 \mapsto n^{2k}I_n$, where $I_n$ is the identity matrix in $\M_n$ and $k$ is
the genus of $X$. This is because smooth manifolds with the same genus are
diffeomorphic and holes decompose as gluings $M \circ W$, as we have seen.

We would, therefore, like to relax the diffeomorphism equivalence of cobordisms.
However, when we do this, gluing is no longer associative, but it is associative
up to diffeomorphism.  This motivates us to look towards framing $\Thick_2$ as a
bicategory where $2$--morphisms $f : M \To M'$ between cobordisms $\fn{M,
M'}{X}{Y}$ are smooth maps $f : M \to M'$ making diagram \eqref{diag:eqcob}
commute. We formalize this in the next subsections.

\section{Bicategory of Cobordisms}

Our first observation is that we use almost none of the details of the manifold
structures on objects in $\Thick_2$ in defining its morphisms and composition.
What we use are boundary inclusions and diagrams involving boundary inclusions.
This motivates us to first abstract cobordism categories to purely categorical
constructions. We define the following kind of category as a category equipped
with a notion of ``gluing''.

\begin{defn}[Gluing Category]
Let $\s{C}$ be a category equipped with the following structures:
\begin{enumerate}[(i)]
\setlength{\itemsep}{0pt}
\item binary coproducts
\item binary pushouts
\item $\Ob \s{C} = O \amalg M$\footnote{This is meant to be an abstraction of
the subcategory of $\Man$ with only $d$-- and $(d + 1)$--manifolds.} with
$\s{O}$ and $\s{M}$, the full sub-categories generated by $O$ and $M$,
respectively
\item $\s{O}$ and $\s{M}$ are closed under binary coproducts and each containing
an initial object, $\varnothing_{\s{O}}$ and $\varnothing_{\s{M}}$ respectively
\end{enumerate}
Then, we say that $\s{C}$ is a gluing category with object category
$\s{O}$ and morphism category $\s{M}$.
\end{defn}

\begin{exm}
As intended, the sub-category of $\Man$ generated by the $d$-- and
$(d + 1)$--manifolds is a gluing category. This will be our main example.
\end{exm}

\begin{rmk}
Note that in this definition, we have not specified the existence of
``cylinders on'' objects in $\s{O}$ residing in $\s{M}$. We will address this as
we define our desired bicategory.
\end{rmk}

\begin{rmk}
We have not required $\varnothing_{\s{O}}$ and $\varnothing_{\s{M}}$ be initial
in $\s{C}$, although for our main example involving manifolds, they will be.
\end{rmk}

\begin{rmk}
We observe that each of $\s{O}$ and $\s{M}$ are cocartesian and hence have
monoidal structures $(\amalg, \varnothing_{\s{O}})$ and
$(\amalg, \varnothing_{\s{M}})$, respectively.
\end{rmk}

\begin{rmk}
We could generalize gluing categories to have
$\Ob \s{C} = \coprod_{n \in \N} C_N$ so that $\Man$ itself becomes an example
but we avoid this for simplicity.
\end{rmk}

This is enough structure\footnote{\dots we believe \dots} to define a reasonable
monoidal bicategory of cobordisms. We define the data of this bicategory,
without identity $1$--morphisms, as follows:

\begin{defn}[Gluing Pre-Bicategory]

Let $\s{C}$ be a gluing category with object category $\s{O}$ and
morphism category $\s{M}$. The gluing pre-bicategory associated to
$\s{C}$, denoted $\s{C}^*$, consists of the following data:

\begin{enumerate}[(i)]

\item We define a set $\Ob \s{C}^* := \Ob \s{O}$. We call the elements
of $\Ob \s{C}^*$ objects or $0$--morphisms or $0$--cells.

\item For each pair of objects $X, Y \in \Ob \s{C}^*$, a $1$--morphism or
$1$--cell $(M, a, b) : X \to Y$ is an object $M \in \s{M}$ along with
monomorphisms $a : X \monic M$ and $b : Y \monic M$ in $\s{C}$; we also write
$\fn{M}{X}{Y}$, when $a$ and $b$ are not important. We denote the set of
$1$--morphisms $X \to Y$ as $\s{C}^*_1(X, Y)$ or $\s{C}^*_{X, Y}$ or
$\s{C}^*(X, Y)$. We call the maps $a$ and $b$ boundary inclusions.

\item For each pair of objects $X, Y \in \Ob \s{C}^*$ and each pair of
$1$--morphisms $(M, a, b)$ and $(M', a', b')$ in $\s{C}^*_{X, Y}$, a
$2$--morphism $f : (M, a, b) \To (M', a', b')$ is a morphism $\fn{f}{M}{M'}$ in
$\s{C}$ making diagram \eqref{diag:eqcob} commute -- we include the diagram
again here for completeness:
\[
\begin{tikzpicture}[baseline=(a).base]
\node[scale=\diagscale] (a) at (0, 0){
\begin{tikzcd}
  & X \arrow[dl, "a" above left] \arrow[dr, "a'" above right] & \\
  M \arrow[rr, "f" above] &
  & M' \\
  & Y \arrow[ul, "b" below left] \arrow[ur, "b'" below right] &
\end{tikzcd}
};
\end{tikzpicture}
\]
We denote the set of
$2$--morphisms $M \To M'$ as $\s{C}^*_{X, Y}(M, M')$ or $\s{C}^*_2(M, M')$ or
$\s{C}^*(M, M')$ or $\s{C}^*_{M, M'}$.

\item For objects $X, Y, Z \in \Ob \s{C}^*$ and $1$--morphisms
$M \in \s{C}^*_{X, Y}$ and $N \in \s{C}^*_{Y, Z}$, we define horizontal
composition or \textit{gluing} as follows for objects:
\[\begin{array}{ccccc}
  - *_{X, Y, Z} -
  &:& \s{C}^*_{Y, Z} \times \s{C}^*_{X, Y}
  &  \to
  &  \s{C}^*_{X, Z} \\
  &:& ((N, a_N, b_N), (M, a_M, b_M))
  &  \mapsto
  &  (M \amalg_{Y} N, p_{NM}a_M, q_{N}b_N)
\end{array}\]
where $p_{NM}$ and $q_{NM}$ are pushed-out maps as in \eqref{diag:gluingdef}.
For $2$--morphisms as shown below:
\begin{equation}
\begin{tikzpicture}[baseline=(a).base]
\node[scale=\diagscale] (a) at (0, 0){
\begin{tikzcd}[column sep=85]
  X \arrow[r, bend left=35,
          "{(M_1, a_{M_1}, b_{M_1})}"{name=M1, description}]
    \arrow[r, bend right=35,
          "{(M_2, a_{M_2}, b_{M_2})}"{name=M2, description}] &
  Y \arrow[r, bend left=35,
          "{(N_1, a_{N_1}, b_{N_1})}"{name=N1, description}]
    \arrow[r, bend right=35,
          "{(N_2, a_{N_2}, b_{N_2})}"{name=N2, description}] &
  Z
  \arrow[Rightarrow, from=M1, to=M2, "f_M"]
  \arrow[Rightarrow, from=N1, to=N2, "f_N"]
\end{tikzcd}
};
\end{tikzpicture}
\end{equation}
we expand the diagram using the definition of a $2$--morphism and then push out
the boundary inclusions of Y to get the following diagram:
\begin{equation}\label{diag:horizontalcomp}
\begin{tikzpicture}[baseline=(a).base]
\node[scale=\diagscale] (a) at (0, 0){
\begin{tikzcd}[column sep=50, row sep=huge]
  & &
  N_1 * M_1 \arrow[dd, "\phi" description, dashed] & & \\
  & & & & \\&
  M_1 \arrow[ruu, "p_{N_1M_1}" above left] \arrow[dd, "f_M" description] &
  N_2 * M_2 &
  N_1 \arrow[luu, "q_{N_1M_1}" above right] \arrow[dd, "f_N" description] & \\
  X \arrow[ru, "a_{M_1}"] \arrow[rd, "a_{M_2}" below left] & &
  Y \arrow[ru, "a_{N_1}" description, near end]
    \arrow[lu, "b_{M_1}" description, near end]
    \arrow[ld, "b_{M_2}" below right]
    \arrow[rd, "a_{N_2}" below left] & &
  Z \arrow[lu, "b_{N_1}" above right] \arrow[ld, "b_{N_2}" below right] \\ &
  M_2 \arrow[ruu, "p_{N_2M_2}" description, crossing over] & &
  N_2 \arrow[luu, "q_{N_2M_2}" description, crossing over] &
\end{tikzcd}
};
\end{tikzpicture}
\end{equation}
This yields a unique map $\phi : N_1 * M_1 \to N_2 * M_2$ by the pushout
property of $N_1M_1$ in $\s{C}$. We define:
\[
  f_N *_{X, Y, Z} f_M = f_M \amalg_Y f_N := \phi
\]

\item For objects $X, Y \in \Ob \s{C}^*$ and $2$--morphisms
$f : P \To Q$, $g : Q \To R$ in $\s{C}^*_{X, Y}$, we define vertical
composition as the comoposition in $\s{C}$:
\[
  (g, f) \mapsto g \circ_{\s{C}} f = gf
\]

\end{enumerate}
\end{defn}

We now proceed to verify that the data above forms a bicategory when
supplemented with the appropriate notion of identity $1$--morphisms. We note the
most straightforward observation first:
\begin{lem}
Vertical composition is well-defined, strictly associative and unital making
each $\s{C}^*_{X, Y}$, for all $X, Y \in \Ob \s{C}^*$, a category.
\end{lem}
\begin{proof}
We need only paste the relevant analogues of diagram \eqref{diag:eqcob}.
\end{proof}

The next most immediate fact is as follows.
\begin{lem}
Let $X, Y, Z \in \s{C}^*$. Then,
$- *_{X, Y, Z} - : \s{C}^*_{Y, Z} \times \s{C}^*_{X, Y} \to \s{C}^*_{X, Z}$
is a functor of $1$--categories.
\end{lem}
\begin{proof}
We consider $2$--morphisms as shown below:
\begin{equation}
\begin{tikzpicture}[baseline=(a).base]
\node[scale=\diagscale] (a) at (0, 0){
\begin{tikzcd}[column sep=100]
  X \arrow[r, bend left=50,
          "{(M_1, a_{M_1}, b_{M_1})}"{name=M1, description}]
    \arrow[r, "{(M_2, a_{M_2}, b_{M_2})}"{name=M2, description}]
    \arrow[r, bend right=50,
          "{(M_3, a_{M_3}, b_{M_3})}"{name=M3, description}] &
  Y \arrow[r, bend left=50,
          "{(N_1, a_{N_1}, b_{N_1})}"{name=N1, description}]
    \arrow[r, "{(N_2, a_{N_2}, b_{N_2})}"{name=N2, description}]
    \arrow[r, bend right=50,
          "{(N_3, a_{N_3}, b_{N_3})}"{name=N3, description}] &
  Z
  \arrow[Rightarrow, from=M1, to=M2, "f_M"]
  \arrow[Rightarrow, from=M2, to=M3, "g_M"]
  \arrow[Rightarrow, from=N1, to=N2, "f_N"]
  \arrow[Rightarrow, from=N2, to=N3, "g_N"]
\end{tikzcd}
};
\end{tikzpicture}
\end{equation}
We extend \eqref{diag:horizontalcomp} to get:
\begin{equation}
\begin{tikzpicture}[baseline=(a).base]
\node[scale=\diagscale] (a) at (0, 0){
\begin{tikzcd}[column sep=65, row sep=huge]
& &
N_1 * M_1 \arrow[d, "f_N * g_N" description, blue!55!black]
          \arrow[dd, bend right=65, shift right=5, green!55!black,
                "g_Nf_N * g_Mf_M" left, near start] & & \\ & &
N_2 * M_2 \arrow[d, "g_N * g_M" description, red!55!black] & & \\ &
M_1 \arrow[ruu] \arrow[d, "f_M" description] &
N_3 * M_3 &
N_1 \arrow[luu] \arrow[d, "f_N" description] & \\
X \arrow[ru, "a_{M_1}" description]
  \arrow[r, "a_{M_2}" description]
  \arrow[rd, "a_{M_3}" description] &
M_2 \arrow[ruu] \arrow[d, "g_M" description] &
Y \arrow[ru, "a_{N_1}" description]
  \arrow[r, "a_{N_2}"  description, near start]
  \arrow[rd, "a_{N_3}" description]
  \arrow[lu, "b_{M_1}" description]
  \arrow[l, "b_{M_2}"  description, near start]
  \arrow[ld, "b_{M_3}" description] &
N_2 \arrow[luu, crossing over] \arrow[d, "g_N" description] &
Z \arrow[lu, "b_{N_1}" description]
  \arrow[l, "b_{N_2}"  description]
  \arrow[ld, "b_{N_3}" description] \\
& M_3 \arrow[ruu, crossing over] & & N_3 \arrow[luu, crossing over] &
\end{tikzcd}
};
\end{tikzpicture}
\end{equation}
where the unlabelled arrows are pushed-out maps. We get
\[
  g_Nf_N * g_Mf_M = (g_N * g_M)(f_N * f_M)
\]
by the universal property of pushouts in $\s{C}^*$, showing that
$- *_{X, Y, Z} -$ preserves composition. The universal property again yields
preservation of identities by taking $f_N$ and $f_M$ to be identities in
\eqref{diag:horizontalcomp}.
\end{proof}

We also have \textit{coherent} associativity of horizontal composition up to
$2$--isomorphisms as we show in the following two lemmas.
\begin{lem}
Let $W, X, Y, Z$ be objects of $\s{C}^*$ with $1$--morphisms, $L : W \to X$,
$M : X \to Y$ and $N : Y \to Z$. Then, there is a $2$--isomorphism
\[
  \alpha_{L, M, N}
  : N * (M * L) = (L \amalg_X M) \amalg_Y N
  \To L \amalg_X (M \amalg_Y N) = (N * M) * L
\]
natural in $L, M, N$.
\end{lem}
\begin{proof}
We let $a_R, b_R$ be the boundary inclusions and $p_S, q_S$ for suitable choices
of $R$ and $S$. For instance, $a_M$ is the boundary inclusion $X \monic M$,
$b_M$ is the boundary inclusion $Y \monic M$ while $p_{M * L}$ is the
pushout inclusion $L \monic M * L$ and $q_{M * L}$ is the pushout
inclusion $M \monic M * L$. As shown in \cite{Mahmud2021}, we get the
following pasting of pushout diagrams:
\begin{equation}\label{assoc1}
\begin{tikzpicture}[baseline=(a).base]
\node[scale=\diagscale] (a) at (0, 0){
\begin{tikzcd}[column sep=50, row sep=large]
  & X \arrow[r, "b_{L}" description] \arrow[d, "a_{M}" description]
  & L \arrow[d, green!55!black, "p_{ML}" description]
      \arrow[rdd, bend left, "p_{(NM)L}" description] \\
  Y \arrow[r, "b_{M}" description] \arrow[d, "a_{N}" description]
  & M \arrow[r, "q_{ML}" description] \arrow[d, "p_{NM}" description]
  & ML \arrow[dd, green!55!black, near end, "r_{N(ML)}" description]
      \arrow[rd, dashed, red, "\gamma" description]\\
  N \arrow[r, red, "q_{NM}" description]
    \arrow[rrd, bend right=20, shift left=1, "q_{N(ML)}" description]
  & NM
      \arrow[rr, near start, crossing over, red, "q_{(NM)L}" description]
      \arrow[rd, dashed, green!55!black, "\beta" description]
  &
  & (NM)L
      \arrow[dl, dashed, shift right=1ex, "\phi" description] \\
  & & N(ML)
      \arrow[ur, dashed, shift right=1ex, "\psi" description] &
\end{tikzcd}
};
\end{tikzpicture}
\end{equation}
where $\gamma$ and $\beta$ are maps obtained by the pushout properties of
$M * L$ and $N * M$ respectively in $\s{C}$. The green paths in the diagram
yield a map $\fn{\phi}{(N * M) * L}{N * (M * L)}$ by the pushout property of
$(N * M) * L$. Similarly, the red paths yield a map
$\fn{\psi}{N * (M * L)}{(N * M) * L}$. The maps $\psi$ and $\phi$ are then shown
to be $2$--morphisms as well as inverse to each other in $\s{C}$, in
\cite{Mahmud2021}. We let
$\alpha_{L, M, N} := \psi, \alpha^{-1}_{L, M, N} := \phi$.

For naturality, let
\[
\begin{array}{ccccc}
  f_L & : & (L', a_{L'}, b_{L'}) & \To & (L, a_L, b_L) \\
  f_M & : & (M', a_{M'}, b_{M'}) & \To & (M, a_M, b_M) \\
  f_N & : & (N', a_{N'}, b_{N'}) & \To & (N, a_N, b_N)
\end{array}
\]
be $2$--morphisms. Using the same argument as above, we obtain a diagram
involving $L', M', N'$ analogous to \eqref{assoc1}. Pasting this diagram to
\eqref{assoc1} using $f_L, f_M, f_N$ we get the following commutative
diagram\footnote{This diagram was generated using a tikzcd-editor
\cite{tikzcdeditor}.} in $\s{C}$:
\begin{figure}[H]\label{fig:glue_assoc}
  \begin{center}
  \begin{tikzpicture}[baseline=(a).base]
  \node[scale=0.825] (a) at (0, 0){
  \begin{tikzcd}[column sep=large, row sep=huge]
  & &
  Y
    \arrow[lld, "a_{N'}" description]
    \arrow[rrd, "b_{M'}" description]
    \arrow[llddd, "a_{N}" description, near start]
    \arrow[rrddd, "b_{M}" description, near start] & & & &
  X
    \arrow[rrd, "b_{L'}" description]
    \arrow[lld, "a_{M'}" description]
    \arrow[rrddd, "b_{L}" description, near start]
    \arrow[llddd, "a_{M}" description, near start] & & \\
  N'
    \arrow[dd, "f_{N}" description]
    \arrow[rrd, "q_{N'M'}" description]
    \arrow[rrdddd, "q_{N'(M'L')}" description, bend right=25, near end,
          shift left=2] & & & &
  M'
    \arrow[dd, "f_{M}" description]
    \arrow[lld, "p_{N'M'}" description]
    \arrow[rrd, "q_{M'L'}" description] & & & &
  L'
    \arrow[dd, "f_{L}" description]
    \arrow[lld, "p_{M'L'}" description]
    \arrow[lldddd, "p_{(N'M')L'}" description, near end, bend left=25,
          shift right=2] \\ &  &
  N'M'
    \arrow[dd, "f_{NM}" description, dashed]
    \arrow[rrrrddd, "q_{(N'M')L'}" description, near start]
    \arrow[ddd, "\beta'" description, bend left] & & & &
  M'L'
    \arrow[dd, "f_{ML}" description, dashed]
    \arrow[llllddd, "p_{N'(M'L')}" description, near start]
    \arrow[ddd, "\gamma'" description, bend right] & & \\
  N
    \arrow[rrd, "q_{NM}" description, near end]
    \arrow[rrdddd, "q_{N(ML)}" description, bend right] & & & &
  M
    \arrow[rrd, "q_{ML}" description]
    \arrow[lld, "p_{NM}" description] & & & &
  L
    \arrow[lld, "p_{ML}" description, near end]
    \arrow[lldddd, "p_{((NM)L)}" description, bend left] \\ & &
  NM
    \arrow[rrrrddd, "q_{(NM)L)}" description, near end]
    \arrow[ddd, "\beta" description, bend right=45] & & & &
  ML
    \arrow[llllddd, "p_{N(ML)}" description, near end]
    \arrow[ddd, "\gamma" description, bend left=45] & & \\ & &
  N'(M'L')
    \arrow[dd, "f_{N(ML)}" description, dashed]
    \arrow[rrrr, "\alpha_{L', M', N'}" above, shift left=2,
           crossing over] &  & & &
  (N'M')L'
    \arrow[dd, "f_{(NM)L}" description, dashed]
    \arrow[llll, "\alpha_{L', M', N'}^{-1}" below, near start,
           shift left=2, crossing over]
  & & \\ & & & & & & & & \\ & &
  N(ML)
    \arrow[rrrr, "\alpha_{L, M, N}" above, shift left=2] & & & &
  (NM)L
    \arrow[llll, "\alpha_{L, M, N}^{-1}" below, shift left=2] & &
  \end{tikzcd}
  };
  \end{tikzpicture}
  \end{center}
  \caption{Weak associativity of horizontal composition}
\end{figure}
Here, the dashed arrows are obtained by the universal properties of the relevant
pushout objects. By the uniqueness of these dashed arrows, we also have
\[
  f_{N * (M * L)} = f_N * f_{M * L}  \text{ and }
  f_{(N * M) * L} = f_{N * M} * f_L
\]
Thus, the square forming the front face of diagram \ref{fig:glue_assoc} shows
that $\alpha$ is a natural isomorphism
$- * (- * -) \To (- * -) * -$.
\end{proof}

\begin{lem}
Let $V, W, X, Y, Z$ be objects of $\s{C}^*$ with $1$--morphisms, $K : V \to W$,
$L : W \to X$, $M : X \to Y$ and $N : Y \to Z$. Then, $\alpha$ satisfies the
pentagon identity or associativity coherence:
\begin{equation}
\begin{tikzpicture}[baseline=(a).base]
\node[scale=\diagscale] (a) at (0, 0){
\begin{tikzcd}[column sep=1, row sep=huge]
  & &
  (NM)(LK) \arrow[rrd, "\alpha_{K, L, NM}" above right] & & \\
  N(M(LK)) \arrow[rru, "\alpha_{LK, M, N}" above left]
           \arrow[rd, "\id_N * \alpha_{K, L, M}" below left] & & & &
  ((NM)L)K \\ &
  N((ML)K) \arrow[rr, "\alpha_{K, ML, N}" below] & &
  (N(ML))K \arrow[ru, "\alpha_{L, M, N} * \id_K" below right] &
\end{tikzcd}
};
\end{tikzpicture}
\end{equation}
\end{lem}
\begin{proof}
Considering all relevant boundary inclusions and pushing out the maps needed to
get every node in the pentagon identity, we get the following
diagram\footnote{Here, we skip the labels for the boundary inclusions and the
pushed-out maps because their exact details are not necessary.}:
\begin{figure}[H]\label{fig:pushoutpentagon}
  \begin{center}
  \begin{tikzpicture}[baseline=(a).base]
  \node[scale=0.675] (a) at (0, 0){
  \begin{tikzcd}[row sep=30]
  &
  W
  \arrow[rrd]
  \arrow[ld] & & & &
  X
  \arrow[rrd]
  \arrow[lld] & & & &
  Y
  \arrow[rd]
  \arrow[lld] & \\
  K
  \arrow[rd]
  \arrow[r]
  \arrow[rrrddddddd, bend right]
  \arrow[rrrddddd, bend right, shift left] &
  LK
  \arrow[rrd]
  \arrow[rrrrdddd] & &
  L
  \arrow[rrrrd]
  \arrow[rr]
  \arrow[ll] & &
  ML
  \arrow[rrrrd]
  \arrow[lllld] & &
  M
  \arrow[lllld]
  \arrow[rr]
  \arrow[ll] & &
  NM
  \arrow[lld]
  \arrow[lllldddd] &
  N
  \arrow[ld]
  \arrow[l]
  \arrow[lllddddd, bend left, shift right]
  \arrow[lllddddddd, bend left] \\ &
  (ML)K
  \arrow[rrrrrrdddddd, bend right=15] & &
  M(LK)
  \arrow[rrrrdddd, bend left] & & & &
  (NM)L
  \arrow[lllldddd, bend right] & &
  N(ML)
  \arrow[lllllldddddd, bend left=15]
  & \\ & & & & & & & & & & \\ & & & & & & & & & & \\ & & & & &
  (NM)(LK)
  \arrow[lld, crossing over, green!55!black, "\alpha_{K, L, NM}" description]
  & & & & & \\ & & &
  ((NM)L)K & & & &
  N(M(LK))
  \arrow[llu, crossing over, green!55!black, "\alpha_{LK, M, N}" description]
  \arrow[dd, green!55!black, dashed, "\phi" description]
  & & & \\ & & & & & & & & & & \\ & & &
  (N(ML))K
  \arrow[uu, green!55!black, dashed, "\psi" description] & & & &
  N((ML)K)
  \arrow[llll, green!55!black, "\alpha_{K, ML, N}" description] & & &
  \end{tikzcd}
  };
  \end{tikzpicture}
  \end{center}
  \caption{Pentagon from pushouts}
\end{figure}
The arrows in green are unique arrows making the diagram commute, obtained by
the universal properties of the various pushouts -- their existence is easy to
show by following the paths leading to each node on the bottom pentagon and
recalling that boundary inclusions for horizontal composites are given by
composing existing boundary inclusions by some pushed-out map.
Furthermore, by construction, the top two and the bottom arrows are
associators. Now, it suffices to show that $\psi = \alpha_{L, M, N} * \id_K$
and $\phi = \id_N * \alpha_{K, L, M}$.

Consider the pasting of pushout diagrams in $\s{C}$ defining
$\alpha_{L, M, N} * \id_K$:
\[
\begin{tikzpicture}[baseline=(a).base]
\node[scale=\diagscale] (a) at (0, 0){
\begin{tikzcd}[column sep=50, row sep=15]
  & &
  N(ML) * K \arrow[dd, "\alpha_{L, M, N} * \id_K" description, dashed] & & \\
  & & & & \\&
  K \arrow[ruu, "p_{N(ML)K}" description] \arrow[dd, "\id_K" description] &
  (NM)L * K &
  N(ML) \arrow[luu, "q_{N(ML)K}" description]
        \arrow[dd, "\alpha_{L, M, N}" description] & \\
  W \arrow[ru, "a_{K}" description] \arrow[rd, "a_{K}" description] & &
  X \arrow[ru, "a_{N(ML)}" description, pos=0.725]
    \arrow[lu, "b_{K}" description, near end]
    \arrow[ld, "b_{K}" description]
    \arrow[rd, "a_{(NM)L}" description] & &
  Z \arrow[lu, "b_{N(ML)}" description]
    \arrow[ld, "b_{(NM)L}" description] \\ &
  K \arrow[ruu, "p_{(NM)LK}" description, crossing over] & &
  (NM)L \arrow[luu, "q_{(NM)LK}" description, crossing over] &
\end{tikzcd}
};
\end{tikzpicture}
\]
By construction, the boundary inclusions in this diagram are precisely the paths
in figure \ref{fig:pushoutpentagon} leading to $N(ML), (NM)L$ and $K$. By the
uniqueness of $\psi$ (or $\alpha_{L, M, N} * \id_K$), we must have
$\psi = \alpha_{L, M, N} * \id_K$. The argument for $\phi$ is identical.
\end{proof}

\begin{cor}
Manifolds of a fixed dimension taken as objects, cobordisms between them taken
as $1$--morphisms and boundary-preserving smooth maps between cobordisms as
$2$--morphisms form a gluing pre-bicategory.
\end{cor}

At this point, in order to obtain a bicategory of cobordisms, we may wish to
look for a way to turn a gluing pre-bicategory to a bicategory -- we only need
unitors for horizontal composition and a unity coherence axiom.  Recall,
however, that unitors in cobordism categories result from a sepcial property of
the category of smooth manifolds and smooth maps -- the smooth collar
neighbourhood theorem, which yields a cylinder $X \times I$ for every smooth
manifold $X$ that acts as an identity for gluing cobordism classes along $X$. In
the abstract setting we have been working in so far, we require the following.
\begin{defn}[Gluing Identity]\label{def:glueid}
Let $\s{C}$ be a pre-manifold category with gluing pre-bicategory $\s{C}^*$.
For an object $X$ of $\s{C}$, a gluing identity on $X$ is a functor
$1_X : 1 \to \s{C}^*_{X, X}$\footnote{$1$ is the one-object discrete category.},
sending the single object $*$ to a $1$--morphism $1_X : X \to X$ in $\s{C}^*$
and $\id_*$ to the $2$--morphism $\id_{1_X}$ in $\s{C}^*$, equipped with the
following structures:
\begin{enumerate}[(i)]
\setlength{\itemsep}{0pt}
\item for each $1$--morphism $N : X \to Y$ in $\s{C}^*$, a $2$--isomorphism
$r_{X, Y, N} : N * 1_X \To N$, natural in $N$
\item for each $1$--morphism $M : W \to X$ in $\s{C}^*$, a $2$--isomorphism
$l_{X, W, M} : 1_X * M \To M$, natural in $N$
\end{enumerate}
\end{defn}

We might wish to prove naturality without including it in the definition but it
is not immediate and we are ultimately interested in cobordism categories.
Hence, we check that given gluing identities, as defined above, in a gluing
pre-bicategory, the unity coherence axiom holds.
\begin{lem}
Let $\s{C}$ be a gluing category with associated gluing pre-bicategory
$\s{C}^*$. If $\s{C}^*$ is equipped with a gluing identity for every object,
then for each pair of $1$--morphisms $M : X \to Y$, $N : Y \to Z$ in $\s{C}^*$,
the following diagram, called the unity coherence axiom, commutes:
\[
\begin{tikzpicture}[baseline=(a).base]
\node[scale=\diagscale] (a) at (0, 0){
\begin{tikzcd}[row sep=35]
N(1_YM) \arrow[rr, "\alpha_{M, 1_Y, Z}" above, shift left]
        \arrow[rd, "\id_N * l_{X, Y, M}" below left] & &
(N1_Y)M \arrow[ll, "\alpha^{-1}_{M, 1_Y, Z}" below, shift left]
        \arrow[ld, "r_{Y, Z, N} * \id_M" below right] \\ &
NM &
\end{tikzcd}
};
\end{tikzpicture}
\]
\end{lem}
\begin{proof}
It suffices to observe that, by construction, the maps involved in the unity
coherence axiom are the unique maps making the following pasting of pushout
diagrams commute, by the universal property of pushouts:
\[
\begin{tikzpicture}[baseline=(a).base]
\node[scale=\diagscale] (a) at (0, 0){
\begin{tikzcd}[column sep=huge, row sep=huge]
& &
Y \arrow[lld] \arrow[d] \arrow[rrd] & & \\
M \arrow[r] \arrow[rrdd, bend right=35] &
1_YM \arrow[d] &
1_Y \arrow[r] \arrow[l] &
N1_Y \arrow[d] &
N \arrow[l] \arrow[lldd, bend left=35] \\ &
N(1_YM) \arrow[rr, green!55!black, "\alpha^{-1}_{M, 1_Y, N}" below, shift right]
        \arrow[rd, green!55!black, "\id_N * l_{Y, X, M}" below left] & &
(N1_Y)M \arrow[ld, green!55!black, "r_{Y, Z, N} * \id_M" below right]
        \arrow[ll, green!55!black, "\alpha_{M, 1_Y, N}" above, shift right]
        & \\ & &
NM & &
\end{tikzcd}
};
\end{tikzpicture}
\]
where the unlabelled arrows are the relevant bondary inclusions and pushed-out
maps.
\end{proof}

\begin{cor}
A gluing pre-bicategory with gluing identities for every object is a bicategory.
\end{cor}

This allows us to define our desired bicategory of which cobordism bicategories
are an example.

\begin{defn}
A gluing pre-bicategory with gluing identities for every object is called a
gluing bicategory.
\end{defn}

There is one fact that remains to be checked in order to show that cobordisms
between manifolds of a fixed dimension form a gluing bicategory.
\begin{lem}
Let $M, M' : X \to Y$ and $N, N' : Y \to Z$ be cobordisms for $d$--manifolds
$X, Y, Z$, with boundary preserving maps $f_M : M \to M', f_N : N \to N'$. Let
$1_Y$ be the cylinder $Y \times [0, 1]$ on $Y$ with diffeomorphisms
$r_{Y, Z, N} : N * 1_Y \to N$, $l_{Y, X, M} : 1_Y * M \to M$ from the collar
neighbourhood theorem, and likewise for $N'$ and $M'$. Then, following diagrams
commute:
\[
\begin{tikzpicture}[baseline=(a).base]
\node[scale=\diagscale] (a) at (0, 0){
\begin{tikzcd}[column sep=huge, row sep=huge]
N * 1_Y \arrow[r, "r_{Y, Z, N}" description]
        \arrow[d, "f_N * \id_{1_Y}" description] &
N \arrow[d, "f_N" description] \\
N' * 1_Y \arrow[r, "r_{Y, Z, N'}" description] &
N'
\end{tikzcd}
};
\end{tikzpicture}
\qquad
\begin{tikzpicture}[baseline=(a).base]
\node[scale=\diagscale] (a) at (0, 0){
\begin{tikzcd}[column sep=huge, row sep=huge]
1_Y * M \arrow[r, "l_{Y, X, M}" description]
        \arrow[d, "\id_{1_Y} * f_M" description] &
M \arrow[d, "f_M" description] \\
1_Y * M' \arrow[r, "l_{Y, X, M'}" description] &
M'
\end{tikzcd}
};
\end{tikzpicture}
\]
That is, $r_{Y, Z, N}$ and $l_{Y, X, M}$ are natural in $N$ and $M$
respectively.
\end{lem}
\begin{proof}
Question for Steve.
\end{proof}

We collect our results into the following theorem.
\begin{thm}
The following data form a bicategory:
\begin{enumerate}[(i)]
\setlength{\itemsep}{0pt}
\item $d$--manifolds as objects
\item ($(d + 1)$--dimensional) cobordisms between $d$--manifolds as
$1$--morphisms
\item boundary preserving smooth maps between cobordisms as $2$--morphisms
\item gluing at boundaries as horizontal composition
\item composition of maps as vertical composition
\item the natural associators for gluing in the category of manifolds as the
associators
\item cylinders on $d$--manifolds as gluing identities
\item the natural unitors for gluing in the category of manifolds as the unitors
\end{enumerate}
\end{thm}

\begin{defn}[Cobordism Bicategory]
For any $d$, the bicategory of the previous theorem is called the bicategory of
$(d + 1)$--dimensional cobordisms which we denote as $\BiCob_d$.
\end{defn}

\begin{defn}[Bicategory of {$2$}--dimensional Thick Tangles]
$2$--dimensional Thick tangles form a bicategory with the data of last theorem
which we call the bicategory of $2$--dimensional thick tangles which we denote
as $2\BiThick$.
\end{defn}

\end{document}

