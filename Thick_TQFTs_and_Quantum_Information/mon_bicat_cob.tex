
\documentclass[./Thick_TQFTs_and_Quantum_Information.tex]{subfiles}

\begin{document}

\section{Monoidal Bicategory of Cobordisms}

\subsection{The Issue with Thick Tangles}

Recall the definition of a cobordism category given in \cite{Mahmud2021}.
Following this pattern of definition, we define a category $\Thick_d$ whose
objects are smooth, orientable, compact $d$--manifolds with boundary. For any
two such objects $X, Y$, let $M$ be a $(d + 1)$--manifold with boundary $U
\amalg V$ and a pair of inclusions $X \monic[a] M$, $Y \monic[n] M$ such that
$\im a = U$, $\im b = V$ and both $a$ and $b$ are diffeomorphisms onto their
images. Let $M'$ be another such manifold with $\partial M' = U' \amalg V'$ and
inclusions $X \monic[a'] M', Y \monic[b'] M'$ satisfying the same conditions:
$\im a' = U'$, $\im b = V'$ and $a', b'$ are diffeomorphisms onto their images.
Then we say $M \eqcob M'$ if and only if there exists a diffeomorphism
$f : M \longleftrightarrow M' : f^{-1}$ satisfying:
\begin{equation}\label{diag:eqcob}
  \begin{tikzcd}
    & X \arrow[dl, "a" above left] \arrow[dr, "a'" above right] & \\
    M \arrow[rr, "f" above] &
    & M' \\
    & Y \arrow[ul, "b" below left] \arrow[ur, "b'" below right] &
  \end{tikzcd}
\end{equation}

It is straightforward to check that $\eqcob$ is an equivalence relation. Then,
if $M \in \Thick_d(X, Y)$ and $N \in \Thick_d(Y, Z)$ with boundary inclusions
$X \monic[a_M] M, Y \monic[b_M] M, Y \monic[a_N] N, Z \monic[b_N] N$, their
composite is the class of the manifold obtained by gluing $M$ and $N$ along $Y$
-- that is, $N \circ M \in \Thick_d(X, Z)$ is the pushout $M \amalg_Y N$:
\begin{equation}\label{diag:gluingdef}
  \begin{tikzcd}
    Y \arrow[r, "b_M" above] \arrow[d, "a_N" left] &
    M \arrow[d, "p_{NM}" right] \\
    N \arrow[r, "q_{NM}" below] &
    M \amalg_Y N := N * M
  \end{tikzcd}
\end{equation}
with boundary inclusions
$a_{N \circ M} = p_{NM}a_M : X \monic N * M$ and
$b_{N \circ M} = q_{NM}b_N : Z \monic N * M$. It is also straightforward to
check that $(\amalg, \varnothing)$ is a monoidal structure on $\Thick_d$.

$\Thick_2$, then, is the category of planar cobordisms or thick tangles, which
was shown in \cite{NonCommTQFT} to be the monoidal category freely generated by
by a Frobenius monoid -- the interval $I$ along with the pair-of-pants cobordism
$\fn{M}{I \otimes I}{I}$, the cap $I$ $\fn{E}{\varnothing}{I}$ and their duals,
$\fn{W}{I}{I \otimes I}$ and $\fn{C}{I}{\varnothing}$. Monoidal functors
$F : \Thick_2 \to \Vect_{\K}$ that map $I$ to a matrix algebra $\M_n$
and the pair-of-pants to matrix multiplication force all elements
$F(X) : \K \to \M_n$, for $\fn{X}{\varnothing}{I}$, to be of the form
$1 \mapsto n^{2k}I_n$, where $I_n$ is the identity matrix in $\M_n$ and $k$ is
the genus of $X$. This is because smooth manifolds with the same genus are
diffeomorphic and holes decompose as gluings $M \circ W$, as we have seen.

We would, therefore, like to relax the diffeomorphism equivalence of cobordisms.
However, when we do this, gluing is no longer associative, but it is associative
up to diffeomorphism.  This motivates us to look towards framing $\Thick_2$ as a
bicategory where $2$--morphisms $f : M \To M'$ between cobordisms $\fn{M,
M'}{X}{Y}$ are smooth maps $f : M \to M'$ making diagram \eqref{diag:eqcob}
commute. We formalize this in the next subsections.

\subsection{The Base Bicategory}

Our first observation is that we use almost none of the details of the manifold
structures on objects in $\Thick_2$ in defining its morphisms and composition.
What we use are boundary inclusions and diagrams involving boundary inclusions.
This motivates us to first abstract cobordism categories to purely categorical
constructions.

\begin{defn}[Pre-cobordism Category]
Let $\s{C}$ be a category with binary coproducts and binary pushouts.
Furthermore, let $\Ob \s{C} = O \amalg M$\footnote{This is meant to be an
abstraction of the subcategory of $\Man$ with only $d$-- and
$(d + 1)$--manifolds.} with $\s{O}$ and $\s{M}$, the full
sub-categories generated by $C_1$ and $C_2$, respectively. In addition, let
$\s{O}$ and $\s{M}$ be closed under binary coproducts and each
containing an initial object, $\varnothing_{\s{O}}$ and $\varnothing_{\s{M}}$
respectively. Then, we say that $\s{C}$ is a pre-cobordism category with object
category $\s{O}$ and morphism category $\s{M}$.
\end{defn}

\begin{exm}
As intended, the sub-category of $\Man$ generated by the $d$-- and
$(d + 1)$--manifolds is a pre-cobordism category. This will be our main example.
\end{exm}

\begin{rmk}
Note that in this definition, we have not specified the existence of "cylinders
on" objects in $\s{O}$ residing in $\s{M}$. We will address this
as we define our desired bicategory.
\end{rmk}

\begin{rmk}
We have not required $\varnothing_{\s{O}}$ and $\varnothing_{\s{M}}$ be initial
in $\s{C}$, although for our main example involving manifolds, they will be.
\end{rmk}

\begin{rmk}
We observe that each of $\s{O}$ and $\s{M}$ are cocartesian and hence have
monoidal structures $(\amalg, \varnothing_{\s{O}})$ and
$(\amalg, \varnothing_{\s{M}})$, respectively.
\end{rmk}

\begin{rmk}
We could generalize pre-cobordism categories to have
$\Ob \s{C} = \coprod_{n \in \N} C_N$ so that $\Man$ itself becomes an example
but we avoid this for simplicity.
\end{rmk}

This is enough structure\footnote{\dots we believe \dots} to define a reasonable
monoidal bicategory of cobordisms. Let $\s{C}$ be a pre-cobordism category with
object category $\s{O}$ and morphism category $\s{M}$. We define the data
of this bicategory, denoted $\s{C}^*$, as follows:

\begin{enumerate}[(i)]

\item We define a set $\Ob \s{C}^* := \Ob \s{O}$. We call the elements
of $\Ob \s{C}^*$ objects or $0$--morphisms or $0$--cells.

\item For each pair of objects $X, Y \in \Ob \s{C}^*$, a $1$--morphism or
$1$--cell $(M, a, b) : X \to Y$ is an object $M \in \s{M}$ along with
monomorphisms $a : X \monic M$ and $b : Y \monic M$ in $\s{C}$; we also write
$\fn{M}{X}{Y}$, when $a$ and $b$ are not important. We denote the set of
$1$--morphisms $X \to Y$ as $\s{C}^*_1(X, Y)$ or $\s{C}^*_{X, Y}$ or
$\s{C}^*(X, Y)$.

\item For each pair of objects $X, Y \in \Ob \s{C}^*$ and each pair of
$1$--morphisms $(M, a, b)$ and $(M', a', b')$ in $\s{C}^*_{X, Y}$, a
$2$--morphism $f : (M, a, b) \To (M', a', b')$ is a morphism $\fn{f}{M}{M'}$ in
$\s{C}$ making diagram \eqref{diag:eqcob} commute -- we include the diagram
again here for completeness:
\[\begin{tikzcd}
  & X \arrow[dl, "a" above left] \arrow[dr, "a'" above right] & \\
  M \arrow[rr, "f" above] &
  & M' \\
  & Y \arrow[ul, "b" below left] \arrow[ur, "b'" below right] &
\end{tikzcd}\]
We denote the set of
$2$--morphisms $M \To M'$ as $\s{C}^*_{X, Y}(M, M')$ or $\s{C}^*_2(M, M')$ or
$\s{C}^*(M, M')$ or $\s{C}^*_{M, M'}$.

\item For objects $X, Y, Z \in \Ob \s{C}^*$ and $1$--morphisms
$M \in \s{C}^*_{X, Y}$ and $N \in \s{C}^*_{Y, Z}$, we define horizontal
composition or \textit{gluing} as follows for objects:
\[\begin{array}{ccccc}
  - *_{X, Y, Z} -
  &:& \s{C}^*_{Y, Z} \times \s{C}^*_{X, Y}
  &  \to
  &  \s{C}^*_{X, Z} \\
  &:& ((N, a_N, b_N), (M, a_M, b_M))
  &  \mapsto
  &  (M \amalg_{Y} N, p_{NM}a_M, q_{N}b_N)
\end{array}\]
where $p_{NM}$ and $q_{NM}$ are pushed-out maps as in \eqref{diag:gluingdef}.
For $2$--morphisms as shown below:
\begin{equation}
\begin{tikzcd}[column sep=85]
  X \arrow[r, bend left=35,
          "{(M_1, a_{M_1}, b_{M_1})}"{name=M1, description}]
    \arrow[r, bend right=35,
          "{(M_2, a_{M_2}, b_{M_2})}"{name=M2, description}] &
  Y \arrow[r, bend left=35,
          "{(N_1, a_{N_1}, b_{N_1})}"{name=N1, description}]
    \arrow[r, bend right=35,
          "{(N_2, a_{N_2}, b_{N_2})}"{name=N2, description}] &
  Z
  \arrow[Rightarrow, from=M1, to=M2, "f_M"]
  \arrow[Rightarrow, from=N1, to=N2, "f_N"]
\end{tikzcd}
\end{equation}
we expand the diagram using the definition of a $2$--morphism and then push out
the boundary inclusions of Y to get the following diagram:
\begin{equation}\label{diag:horizontalcomp}
\begin{scriptsize}
\begin{tikzcd}[column sep=50, row sep=huge]
  & &
  N_1 * M_1 \arrow[dd, "\phi" description, dashed] & & \\
  & & & & \\&
  M_1 \arrow[ruu, "p_{N_1M_1}" above left] \arrow[dd, "f_M" description] &
  N_2 * M_2 &
  N_1 \arrow[luu, "q_{N_1M_1}" above right] \arrow[dd, "f_N" description] & \\
  X \arrow[ru, "a_{M_1}"] \arrow[rd, "a_{M_2}" below left] & &
  Y \arrow[ru, "a_{N_1}" description, near end]
    \arrow[lu, "b_{M_1}" description, near end]
    \arrow[ld, "b_{M_2}" below right]
    \arrow[rd, "a_{N_2}" below left] & &
  Z \arrow[lu, "b_{N_1}" above right] \arrow[ld, "b_{N_2}" below right] \\ &
  M_2 \arrow[ruu, "p_{N_2M_2}" description, crossing over] & &
  N_2 \arrow[luu, "q_{N_2M_2}" description, crossing over] &
\end{tikzcd}
\end{scriptsize}
\end{equation}
This yields a unique map $\phi : N_1 * M_1 \to N_2 * M_2$ by the pushout
property of $N_1M_1$ in $\s{C}$. We define:
\[
  f_N *_{X, Y, Z} f_M = f_M \amalg_Y f_N := \phi
\]

\item For objects $X, Y \in \Ob \s{C}^*$ and $2$--morphisms
$f : P \To Q$, $g : Q \To R$ in $\s{C}^*_{X, Y}$, we define vertical
composition as the comoposition in $\s{C}$:
\[
  (g, f) \mapsto g \circ_{\s{C}} f = gf
\]

\end{enumerate}

We now proceed to verify that the data above forms a bicategory. We note the
most straightforward observation first:
\begin{lem}
Vertical composition is well-defined, strictly associative and unital making
each $\s{C}^*_{X, Y}$, for all $X, Y \in \Ob \s{C}^*$, a category.
\end{lem}
\begin{proof}
We need only paste the relevant analogues of diagram \eqref{diag:eqcob}.
\end{proof}

The next most immediate fact is as follows.
\begin{lem}
Let $X, Y, Z \in \s{C}^*$. Then,
$- *_{X, Y, Z} - : \s{C}^*_{Y, Z} \times \s{C}^*_{X, Y} \to \s{C}^*_{X, Z}$
is a functor of $1$--categories.
\end{lem}
\begin{proof}
We consider $2$--morphisms as shown below:
\begin{equation}
\begin{tikzcd}[column sep=100]
  X \arrow[r, bend left=50,
          "{(M_1, a_{M_1}, b_{M_1})}"{name=M1, description}]
    \arrow[r, "{(M_2, a_{M_2}, b_{M_2})}"{name=M2, description}]
    \arrow[r, bend right=50,
          "{(M_3, a_{M_3}, b_{M_3})}"{name=M3, description}] &
  Y \arrow[r, bend left=50,
          "{(N_1, a_{N_1}, b_{N_1})}"{name=N1, description}]
    \arrow[r, "{(N_2, a_{N_2}, b_{N_2})}"{name=N2, description}]
    \arrow[r, bend right=50,
          "{(N_3, a_{N_3}, b_{N_3})}"{name=N3, description}] &
  Z
  \arrow[Rightarrow, from=M1, to=M2, "f_M"]
  \arrow[Rightarrow, from=M2, to=M3, "g_M"]
  \arrow[Rightarrow, from=N1, to=N2, "f_N"]
  \arrow[Rightarrow, from=N2, to=N3, "g_N"]
\end{tikzcd}
\end{equation}
We extend \eqref{diag:horizontalcomp} to get:
\begin{equation}
\begin{scriptsize}
\begin{tikzcd}[column sep=65, row sep=huge]
& &
N_1 * M_1 \arrow[d, "f_N * g_N" description, blue!55!black]
          \arrow[dd, bend right=65, shift right=5, green!55!black,
                "g_Nf_N * g_Mf_M" left, near start] & & \\ & &
N_2 * M_2 \arrow[d, "g_N * g_M" description, red!55!black] & & \\ &
M_1 \arrow[ruu] \arrow[d, "f_M" description] &
N_3 * M_3 &
N_1 \arrow[luu] \arrow[d, "f_N" description] & \\
X \arrow[ru, "a_{M_1}" description]
  \arrow[r, "a_{M_2}" description]
  \arrow[rd, "a_{M_3}" description] &
M_2 \arrow[ruu] \arrow[d, "g_M" description] &
Y \arrow[ru, "a_{N_1}" description]
  \arrow[r, "a_{N_2}"  description, near start]
  \arrow[rd, "a_{N_3}" description]
  \arrow[lu, "b_{M_1}" description]
  \arrow[l, "b_{M_2}"  description, near start]
  \arrow[ld, "b_{M_3}" description] &
N_2 \arrow[luu, crossing over] \arrow[d, "g_N" description] &
Z \arrow[lu, "b_{N_1}" description]
  \arrow[l, "b_{N_2}"  description]
  \arrow[ld, "b_{N_3}" description] \\
& M_3 \arrow[ruu, crossing over] & & N_3 \arrow[luu, crossing over] &
\end{tikzcd}
\end{scriptsize}
\end{equation}
where the unlabelled arrows are pushed-out maps. We get
\[
  g_Nf_N * g_Mf_M = (g_N * g_M)(f_N * f_M)
\]
by the universal property of pushouts in $\s{C}^*$, showing that
$- *_{X, Y, Z} -$ preserves composition. The universal property again yields
preservation of identities by taking $f_N$ and $f_M$ to be identities in
\eqref{diag:horizontalcomp}.
\end{proof}

We also have associativity of horizontal composition up to $2$--isomorphisms.
\begin{lem}
Let $W, X, Y, Z$ be objects of $\s{C}^*$ with $1$--morphisms, $L : W \to X$,
$M : X \to Y$ and $N : Y \to Z$. Then, there is a $2$--isomorphism
\[
  \alpha_{L, M, N}
  : N \circ (M \circ L) = (L \amalg_X M) \amalg_Y N
  \To L \amalg_X (M \amalg_Y N) = (N \circ M) \circ L
\]
natural in $L, M, N$.
\end{lem}
\begin{proof}
We let $a_R, b_R$ be the boundary inclusions and $p_S, q_S$ for suitable choices
of $R$ and $S$. For instance, $a_M$ is the boundary inclusion $X \monic M$,
$b_M$ is the boundary inclusion $Y \monic M$ while $p_{M * L}$ is the
pushout inclusion $L \monic M * L$ and $q_{M * L}$ is the pushout
inclusion $M \monic M * L$. As shown in \cite{Mahmud2021}, we get the
following pasting of pushout diagrams:
\begin{equation}\label{assoc1}
\begin{tikzcd}[column sep=large, row sep=large]
  & X \arrow[r, "b_{L}" above] \arrow[d, "a_{M}" left]
  & L \arrow[d, green!55!black, "p_{ML}" right]
      \arrow[rdd, bend left, "p_{(NM)L}"] \\
  Y \arrow[r, "b_{M}" above] \arrow[d, "a_{N}" left]
  & M \arrow[r, "q_{ML}" below] \arrow[d, "p_{NM}" right]
  & ML \arrow[dd, green!55!black, near end, "r_{N(ML)}"]
      \arrow[rd, dashed, red, "\gamma" description]\\
  N \arrow[r, red, "q_{NM}" below]
    \arrow[rrd, bend right, "q_{N(ML)}"]
  & NM
      \arrow[rr, near start, crossing over, red, "q_{(NM)L}"]
      \arrow[rd, dashed, green!55!black, "\beta" description]
  &
  & (NM)L
      \arrow[dl, dashed, shift right=1ex, "\phi" description] \\
  & & N(ML)
      \arrow[ur, dashed, shift right=1ex, "\psi" description] &
\end{tikzcd}
\end{equation}
where $\gamma$ and $\beta$ are maps obtained by the pushout properties of
$M * L$ and $N * M$ respectively in $\s{C}$. The green paths in the diagram
yield a map $\fn{\phi}{(N * M) * L}{N * (M * L)}$ by the pushout property of
$(N * M) * L$. Similarly, the red paths yield a map
$\fn{\psi}{N * (M * L)}{(N * M) * L}$. The maps $\psi$ and $\phi$ are then shown
to be $2$--morphisms as well as inverse to each other in $\s{C}$, in
\cite{Mahmud2021}. We let
$\alpha_{L, M, N} := \psi, \alpha^{-1}_{L, M, N} := \phi$.

For naturality, let
\[
\begin{array}{ccccc}
  f_L & : & (L', a_{L'}, b_{L'}) & \To & (L, a_L, b_L) \\
  f_M & : & (M', a_{M'}, b_{M'}) & \To & (M, a_M, b_M) \\
  f_N & : & (N', a_{N'}, b_{N'}) & \To & (N, a_N, b_N)
\end{array}
\]
be $2$--morphisms. Using the same argument as above, we obtain a diagram
involving $L', M', N'$ analogous to \eqref{assoc1}. Pasting this diagram to
\eqref{assoc1} using $f_L, f_M, f_N$ we get the following commutative
diagram\footnote{This diagram was generated using a tikzcd-editor
\cite{tikzcdeditor}.} in $\s{C}$:
\begin{figure}[H]\label{fig:glue_assoc}
\begin{center}
\begin{scriptsize}
\begin{tikzpicture}[baseline=(a).base]
\node[scale=0.825] (a) at (0, 0){
\begin{tikzcd}[column sep=large, row sep=huge]
& &
Y
  \arrow[lld, "a_{N'}" description]
  \arrow[rrd, "b_{M'}" description]
  \arrow[llddd, "a_{N}" description, near start]
  \arrow[rrddd, "b_{M}" description, near start] & & & &
X
  \arrow[rrd, "b_{L'}" description]
  \arrow[lld, "a_{M'}" description]
  \arrow[rrddd, "b_{L}" description, near start]
  \arrow[llddd, "a_{M}" description, near start] & & \\
N'
  \arrow[dd, "f_{N}" description]
  \arrow[rrd, "q_{N'M'}" description]
  \arrow[rrdddd, "q_{N'(M'L')}" description, bend right=25, near end,
        shift left=2] & & & &
M'
  \arrow[dd, "f_{M}" description]
  \arrow[lld, "p_{N'M'}" description]
  \arrow[rrd, "q_{M'L'}" description] & & & &
L'
  \arrow[dd, "f_{L}" description]
  \arrow[lld, "p_{M'L'}" description]
  \arrow[lldddd, "p_{(N'M')L'}" description, near end, bend left=25,
        shift right=2] \\ &  &
N'M'
  \arrow[dd, "f_{NM}" description, dashed]
  \arrow[rrrrddd, "q_{(N'M')L'}" description, near start]
  \arrow[ddd, "\beta'" description, bend left] & & & &
M'L'
  \arrow[dd, "f_{ML}" description, dashed]
  \arrow[llllddd, "p_{N'(M'L')}" description, near start]
  \arrow[ddd, "\gamma'" description, bend right] & & \\
N
  \arrow[rrd, "q_{NM}" description, near end]
  \arrow[rrdddd, "q_{N(ML)}" description, bend right] & & & &
M
  \arrow[rrd, "q_{ML}" description]
  \arrow[lld, "p_{NM}" description] & & & &
L
  \arrow[lld, "p_{ML}" description, near end]
  \arrow[lldddd, "p_{((NM)L)}" description, bend left] \\ & &
NM
  \arrow[rrrrddd, "q_{(NM)L)}" description, near end]
  \arrow[ddd, "\beta" description, bend right=45] & & & &
ML
  \arrow[llllddd, "p_{N(ML)}" description, near end]
  \arrow[ddd, "\gamma" description, bend left=45] & & \\ & &
N'(M'L')
  \arrow[dd, "f_{N(ML)}" description, dashed]
  \arrow[rrrr, "\alpha_{L', M', N'}" above, shift left=2,
         crossing over] &  & & &
(N'M')L'
  \arrow[dd, "f_{(NM)L}" description, dashed]
  \arrow[llll, "\alpha_{L', M', N'}^{-1}" below, near start,
         shift left=2, crossing over]
& & \\ & & & & & & & & \\ & &
N(ML)
  \arrow[rrrr, "\alpha_{L, M, N}" above, shift left=2] & & & &
(NM)L
  \arrow[llll, "\alpha_{L, M, N}^{-1}" below, shift left=2] & &
\end{tikzcd}
};
\end{tikzpicture}

\end{scriptsize}
\end{center}
\caption{Weak Associativity of Gluing}
\end{figure}
Here, the dashed arrows are obtained by the universal properties of the relevant
pushout objects. By the uniqueness of these dashed arrows, we also have
\[
  f_{N * (M * L)} = f_N * f_{M * L}  \text{ and }
  f_{(N * M) * L} = f_{N * M} * f_L
\]
Thus, the square forming the front face of diagram \ref{fig:glue_assoc} shows
that $\alpha$ is a natural isomorphism
$- \circ (- \circ -) \To (- \circ -) \circ -$.

\end{proof}

\end{document}

