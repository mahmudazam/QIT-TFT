
\section{The Issue with Thick Tangles}

Recall the definition of a cobordism category given in \cite{Mahmud2021}.
Following this pattern of definition, we define a category $\Thick_d$ whose
objects are smooth, orientable, compact $d$--manifolds with boundary. For any
two such objects $X, Y$, let $M$ be a $(d + 1)$--manifold with boundary $U
\amalg V$ and a pair of inclusions $X \monic[a] M$, $Y \monic[n] M$ such that
$\im a = U$, $\im b = V$ and both $a$ and $b$ are diffeomorphisms onto their
images. Let $M'$ be another such manifold with $\partial M' = U' \amalg V'$ and
inclusions $X \monic[a'] M', Y \monic[b'] M'$ satisfying the same conditions:
$\im a' = U'$, $\im b = V'$ and $a', b'$ are diffeomorphisms onto their images.
Then we say $M \eqcob M'$ if and only if there exists a diffeomorphism
$f : M \longleftrightarrow M' : f^{-1}$ satisfying:
\begin{equation}\label{diag:eqcob}
\begin{tikzpicture}[baseline=(a).base]
\node[scale=\diagscale] (a) at (0, 0){
\begin{tikzcd}
  & X \arrow[dl, "a" above left] \arrow[dr, "a'" above right] & \\
  M \arrow[rr, "f" above] &
  & M' \\
  & Y \arrow[ul, "b" below left] \arrow[ur, "b'" below right] &
\end{tikzcd}
};
\end{tikzpicture}
\end{equation}
In fact, making this diagram commute is our definition of preserving boundaries.
\begin{defn}
Given $X, Y, M, M', a, a', b, b'$ as above, a smooth map $f : M \to M'$ is said
to be boundary-preserving if it makes diagram \eqref{diag:eqcob} commute.
\end{defn}

It is straightforward to check that $\eqcob$ is an equivalence relation. Then,
if $M \in \Thick_d(X, Y)$ and $N \in \Thick_d(Y, Z)$ with boundary inclusions
$X \monic[a_M] M, Y \monic[b_M] M, Y \monic[a_N] N, Z \monic[b_N] N$, their
composite is the class of the manifold obtained by gluing $M$ and $N$ along $Y$
-- that is, $N \circ M \in \Thick_d(X, Z)$ is the pushout $M \amalg_Y N$:
\begin{equation}\label{diag:gluingdef}
\begin{tikzpicture}[baseline=(a).base]
\node[scale=\diagscale] (a) at (0, 0){
\begin{tikzcd}
  Y \arrow[r, "b_M" above] \arrow[d, "a_N" left] &
  M \arrow[d, "p_{NM}" right] \\
  N \arrow[r, "q_{NM}" below] &
  M \amalg_Y N := N \circ M
\end{tikzcd}
};
\end{tikzpicture}
\end{equation}
with boundary inclusions
$a_{N \circ M} = p_{NM}a_M : X \monic N \circ M$ and
$b_{N \circ M} = q_{NM}b_N : Z \monic N \circ M$. It is also straightforward to
check that $(\amalg, \varnothing)$ is a monoidal structure on $\Thick_d$.

$\Thick_2$, then, is the category of planar cobordisms or thick tangles, which
was shown in \cite{NonCommTQFT} to be the monoidal category freely generated by
by a Frobenius monoid -- the interval $I$ along with the pair-of-pants cobordism
$\fn{M}{I \otimes I}{I}$, the cap $I$ $\fn{E}{\varnothing}{I}$ and their duals,
$\fn{W}{I}{I \otimes I}$ and $\fn{C}{I}{\varnothing}$. Monoidal functors
$F : \Thick_2 \to \Vect_{\K}$ that map $I$ to a matrix algebra $\M_n$
and the pair-of-pants to matrix multiplication force all elements
$F(X) : \K \to \M_n$, for $\fn{X}{\varnothing}{I}$, to be of the form
$1 \mapsto n^{2k}I_n$, where $I_n$ is the identity matrix in $\M_n$ and $k$ is
the genus of $X$. This is because smooth manifolds with the same genus are
diffeomorphic and holes decompose as gluings $M \circ W$, as we have seen.

We would, therefore, like to relax the diffeomorphism equivalence of cobordisms.
However, when we do this, gluing is no longer associative, but it is associative
up to diffeomorphism.  This motivates us to look towards framing $\Thick_2$ as a
bicategory where $2$--morphisms $f : M \To M'$ between cobordisms $\fn{M,
M'}{X}{Y}$ are smooth maps $f : M \to M'$ making diagram \eqref{diag:eqcob}
commute. We formalize this in the next subsections.

