
\begin{abstract}
Quantum computing is captured in the formalism of the monoidal subcategory of
$\Vect_{\C}$ generated by $\C^2$ -- in particular, quantum circuits are diagrams
in $\Vect_{\C}$ -- while topological quantum field theories, in the sense of
Atiyah, are diagrams in $\Vect_{\C}$ indexed by cobordisms. We outline a program
to formalize this connection. In doing so, we first equip cobordisms with
machinery for producing linear maps by parallel transport along curves under a
connection and then assemble these structures into a double category. Finite
dimensional complex vector spaces and linear maps between them are given a
suitable double categorical structure which we call $\FFVect_{\C}$. Finally, we
realize quantum circuits as images of cobordisms under monoidal double functors
from these modified cobordisms to $\FFVect_{\C}$, which are computed by taking
parallel transports of vectors and then combining the results in a pattern
encoded in the domain double category.
%We connect \cite{CatQChan} with \cite{NonCommTQFT}.
\end{abstract}

