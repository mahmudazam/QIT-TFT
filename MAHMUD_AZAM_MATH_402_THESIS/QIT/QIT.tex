
\pagebreak

\section{Quantum Information}

In this section we discuss a completely categorical formulation of finite
dimensional $C^*$ algebras. This formulation is one of the main results of
\cite{channels} where the authors prove that the category of finite dimensional
$C^*$ algebras is equivalent to a category constructed from a certain class of
Frobenius algebras. Using this formulation, we proceed to outline the connection
of PFTs with $C^*$ algebras and, by extension, quantum information.

%A qubit in quantum information theory is modelled as a variable taking values in
%the complex vector space $\C^2$.\footnote{We are simplifying the description for
%this initial discussion.} A register of length $n$, then, is a variable taking
%values in the $n$--fold tensor product $Q_n := \bigotimes_{i = 1}^n \C^2$. In
%quantum information, we are primarily interested in linear maps
%$\bigotimes_{j = 1}^N Q_{n_i} \to \bigotimes_{k = 1}^M Q_{n_k}$, without getting
%into the details. Now, under the inner product $(a, b) \mapsto \overline{a^T}b$,
%$Q_n$ is a Hilbert space for all $n = 1, 2, \dots$ and, hence, so are tensor
%products $\bigotimes_{j = 1}^N Q_{n_j}$. A standard result in linear algebra is
%that all linear transformations between finite dimensional complex vector spaces
%are bounded and that the space $\s{L}(V) := \Hom_{\Vect_{\C}}(V, V)$ of bounded
%operators on a finite dimensional space $V$ form a $C^*$ algebra with the
%product given by composition and the involution given by the conjugate
%transpose.

%Therefore, the central objects of study in QIT are finite dimensional
%$C^*$ algebras


\subsection{Abstract \texorpdfstring{$C^*$}{C--star} Algebras}

For any finite dimensional $\mathbf{k}$--vector space $V$, the evaluation
$\fn{\eps_V}{V \m V^*}{\mathbf{k}}$ defined on any basis of
$V \m V^*$ by $a \m f \mapsto f(a)$ and the coevaluation
$\fn{\eta_V}{\mathbf{k}}{V^* \m V}$ defined by
$z \mapsto z\sum b_i^* \m b_i$ (where, the $b_i$ form a basis of $V$ and
the $b_i^*$ are the corresponding dual basis elements) can be verified to
satisfy the non-degeneracy condition of \ref{vect:nondegen}. That very theorem
states that if there is a non-degerate pairing of $V$ and $W$, then they are
each (isomorphic) to the dual of the other. The notion of duality is thus a
categorical one -- we can define the dual of an object $V$ in a monoidal
category to be another object $V^*$ that admits a non-degerate pairing with $V$.

\begin{defn}[Compact Category]
A monoidal category in which each object has a dual and a non-degenerate pairing
with its dual is called a compact category.
\end{defn}

\begin{exm}
$\Cob_d$ and $\Thick$ are compact categories precisely because of theorem
\ref{tqft:nondegen} (and its planar analogue), with duals given by orientation
reversal.
\end{exm}

\begin{exm}
$\Vect_{\mathbf{k}}$ is obviously compact.
\end{exm}

Given a $\mathbf{k}$--linear map $\fn{f}{A}{B}$, we can define a linear map
$\fnr{f^*}{B^*}{A^*}{g}{g \circ f}$. A straightforward but lengthy computation
then shows that the following diagram commutes:
\begin{eqnarray*}
\begin{tikzcd}[column sep=large]
& A^* \m \mathbf{k} \arrow[drr, bend left=10] & & \\
B^* \cong B^* \m \mathbf{k}
  \arrow[r, "1_{B^*} \m \eta_{A^*}"]
  \arrow[ur, bend left=20, "f^* \m 1_{\mathbf{k}}"]
  \arrow[dr, bend right=20] &
B^* \m A \m A^* \arrow[r, "1 \m f \m 1"] &
B^* \m B \m A^* \arrow[r, "\eps_{B^*} \m 1_{A^*}"] &
\mathbf{k} \m A^* \cong A^* \\
& \mathbf{k} \m B^*
  \arrow[urr, bend right=10, "1_{\mathbf{k}} \m f^*"] & &
\end{tikzcd}
\end{eqnarray*}
where the unlabelled arrows are the respective commutators for the tensor
product. By the various isomorphisms, we are also guaranteed that $f^*$ is the
unique map making the above commute. Hence we can take this to be the definition
of $f^*$, as done in \cite[6]{channels}.

Carrying on in this direction, if we denote the conjugate transpose of some
linear map $f$ as $f^{\dagger}$, we have $f^{\dagger \dagger} = f$. Furthermore,
we can verify that the conjugate of $f$ is $f_* = (f^{\dagger})^* =
(f^*)^{\dagger}$ If we now define $A^{\dagger} = A$ for each $A \in
\Vect_{\mathbf{k}}$, then it is easy to verify that $(-)^{\dagger}$ is a
contravariant functor on $\Vect_{\mathbf{k}}$ that is the identity on objects
and is involutive on morphisms and which satisfies that the associators, unitors
and commutators of $\m$ are all unitary\footnote{A linear map $u$ with
$u^{\dagger}$ acting as its two sided inverse is said to be unitary. One can
verify that these are precisely the norm-preserving maps on Hilbert spaces.},
$\eps_{A}^{\dagger} = \eta_{A^*}$ and
$(f \m g)^{\dagger} = f^{\dagger} \m g^{\dagger}$.

While the proofs of these claims may require us to use properties of vector
spaces and linear maps, the constructs themsleves are fundamentally categorical,
allowing us to define the following structures in monoidal categories that
fascilitate the formulation of $C^*$ algebras in completely catgorical terms
\cite{channels}. In what follows, we assume all objects and morphisms are in a
monoidal category $\s{V}$ with monoidal product $\m$ and unit object $U$.

\begin{defn}[Dagger Compact Category]
A contravariant functor $(-)^{\dagger}$ that is the identity on objects and is
involutive on morphisms is called a dagger. A category equipped with a dagger is
called a dagger category. A compact category equipped with a dagger satisfying
all the conditions as above is said to be dagger compact.
\end{defn}

\begin{defn}[Frobenius Monoid]
An object $A$ equipped with a $\m$--monoid structure $(m, e)$ and a
$\m$--comonoid\footnote{The structure obtained by reversing the arrows of a
monoid structure.} structure $(w, c)$ is said to be a Frobenius $\m$--monoid if
the following diagrams called Frobenius identities commute:
\begin{eqnarray*}
\begin{tikzcd}
A \m A \arrow[r, "w \m 1_A"] \arrow[d, "m" left] &
A \m A \m A \arrow[d, "1_A \m m" right] \\
A \arrow[r, "w" below] &
A \m A
\end{tikzcd} \hspace{5em} &
\begin{tikzcd}
A \m A \arrow[r, "1_A \m w"] \arrow[d, "m" left] &
A \m A \m A \arrow[d, "m \m 1_A" right] \\
A \arrow[r, "w" below] &
A \m A
\end{tikzcd}
\end{eqnarray*}
We say that $A$ is symmetric if $c \circ m \circ \sgm = c \circ m$, where $\sgm$
is the commutator $A \m A \to A \m A$.
\end{defn}

\begin{rmk}
This formulation of Frobenius monoids is equivalent to the one in terms of a
non-degenerate pairing \cite[Prop. 1]{NonCommTQFT}.
\end{rmk}

\begin{defn}[Dagger Frobenius Algebra]
If $\s{V}$ is equipped with a dagger, a Frobenius $\m$--monoid $(A, m, e, w, c)$
satisfying $w = m^{\dagger}$ and $c = e^{\dagger}$ is called a dagger Frobenius
algebra.
\end{defn}

\begin{defn}[Positive Definiteness]
A map $g$ in $\s{V}$ is said to be positive if there exists some map $h$ in the
category with $g = h^{\dagger} \circ h$, assuming $\s{V}$ has a dagger. Positive
isomorphisms are said to be positive definite.
\end{defn}

\begin{defn}[Centrality]
Given a $\m$--monoid $(A, m, e)$, a map $\fn{z}{A}{A}$ is said to be central if
$m \circ (z \m \id_A) = z \circ m = m \circ (\id_A \m z)$.
\end{defn}

\begin{defn}[Normalizable Frobenius Algebra]
A (not necessarily dagger) Frobenius algebra $(A, m, e, w, c)$ is said to be
normalizable if it is equipped with a central positive definite map
$\fn{z}{A}{A}$ satisfying:
\begin{eqnarray*}
\begin{tikzcd}[column sep=large]
A \m U \arrow[r, "c" above] \arrow[d, "z^2 \m \eta_{A^*}" left]&
U \\
A \m A \m A^* \arrow[r, "m \m \id_{A^*}" below] &
A \m A^* \arrow[u, "\eps_{A}" right]
\end{tikzcd}
\end{eqnarray*}
where $z^2 = z \circ z$.
\end{defn}

Coecke et al. \cite[10]{channels} have shown that finite direct sums of complex
square matrix algebras of the form $\bigoplus_{i = 1}^k \M_{n_i}$ (for some
$n_1, \dots, n_k \in \N$) with product given by matrix multiplication in every
component $\M_{n_i}$ are normalizable dagger Frobenius algebras. Furthermore, it
is known that every finite dimensional $C^*$ algebra is $*$--isomorphic to such
a direct sum algebra. Finally, \cite[Thm. 2.8]{channels} shows precisely that in
$\FHilb$ every normalizable dagger Frobenius algebra is a finite dimensional
$C^*$ algebra. It is then immediate that the finite dimensional complex $C^*$
algebras are precisely the normalizable dagger Frobenius algebras in $\FHilb$.
We build on this in the next subsection.



\subsection{The \texorpdfstring{$\CPS$}{CP Star} Construction}

In the usual sense, in $\Vect_{\mathbf{k}}$, a completely positive map
$\fn{f}{A}{B}$ of $C^*$ algebras is one which satisfies the condition that
$(f \m 1_{\M_n})(a)$ is a positive element of $B \m \M_{n}$ for all positive
elements $a$ of $A \m \M_n$, for all $n \in \N$. Using only morphisms, a
a map $f$ is completely positive if for all $n \in \N$ and for all
positive $\fn{a}{\mathbf{k}}{A \m \M_n}$,
$(f \m 1_{\M_n}) \circ a$ is a positive map $\mathbf{k} \to B \m \M_n$. In the
abstract setting, we may define complete positivity with an analogous statement
but with an arbitrary dagger Frobenius algebra $C$ in place of $\M_n$. Taking a
slightly different perspective, for any $X \in \Vect_{\mathbf{k}}$, it can be
verified that the map
$\fn{1_{X^*} \m \eps_X \m 1_X}{X^* \m X \m X^* \m X}{X^* \m X}$ gives $X^* \m X$
the structure of a normalizable dagger Frobenius algebra
\cite[Prop. 2.11]{channels} with unit $\eta_X$. We then observe that
$X^* \m X \cong \Hom_{\Vect_{\mathbf{k}}}(X, X)$ so that $X^* \m X$ can be
regarded as a matrix algebra and this formulation of matrix algebras works in
any abstract dagger compact category.  Then, we may define $f$ to be
completely positive if and only if $(f \m 1_{X^* \m X}) \circ a$ is a positive
map $\mathbf{k} \to B \m X^* \m X$ whenever $\fn{a}{\mathbf{k}}{A \m X^* \m X}$
is positive for every object $X$ in the category.

Coecke et al. have shown in \cite[Prop. 3.4]{channels} that both of these
abstract formulations are equivalent to satisfying a certain ``$\CPS$
condition'' in any dagger compact category and that they specialize to the usual
notion for finite dimensional complex $C^*$ algebras. Consider any dagger
compact category $\s{V}$ with product $\m$, unit $U$ and the usual symbols for
the dagger and the dual. Given any $C^*$ algebra -- by which we mean a
normalizable dagger Frobenius algebra -- $(A, m, e, w, c)$ in $\s{V}$ we define
the following notation:
\begin{eqnarray*}
  \wh{m} :=&
  \begin{tikzcd}[column sep=large]
    U \m A \arrow[r, "\eta_A \m \id_A"] &
    A^* \m A \m A \arrow[r, "\id_{A^*} \m m"] &
    A^* \m A
  \end{tikzcd}\\
  \wh{w} :=&
  \begin{tikzcd}[column sep=large]
    A^* \m A \arrow[r, "\id_{A^*} \m w"] &
    A^* \m A \m A \arrow[r, "\eps_{A^*} \m \id_{A}"] &
    U \m A
  \end{tikzcd}
\end{eqnarray*}
We now let $(A, m, e, w, c)$ and $(B, m', e', w', c')$ be $C^*$ algebras in
$\s{V}$ with a map $\fn{f}{A}{B}$ and observe that $\wh{m'} f \wh{w}$ can be a
well defined composition when unitors are placed appropriately.
\begin{defn}[{$\CPS$} Condition]
If there exists a map $\fn{g}{A}{X \m B}$ for some object $X \in \s{V}$
satisfying
\begin{equation}\label{cpstar:cond}
  \wh{m'}f\wh{w} = (1_{B*} \m \eps_{X^*} \m 1_{B}) \circ (g_* \m g)
\end{equation}
then $f$ is said to satisfy the $\CPS$ condition.
\end{defn}

\begin{rmk}
This is equation (2) is \cite[13]{channels}.
\end{rmk}

We thus define:
\begin{defn}[Completely Positive Map]
A map $\fn{f}{A}{B}$ of normalizable dagger Frobenius algebras in a dagger
compact category is said to be completely positive if it satisfies
\eqref{cpstar:cond}.
\end{defn}

Let $\CPS[\s{V}]$ denote the collection of normalizable dagger Frobenius
algebras in $\s{V}$ together with the completely positive maps
between them. The central results of \cite{channels} are as follows.
The completely positive maps in $\s{V}$ compose to give completely positive,
making $\CPS[\s{V}]$ a category. Next, the associators, unitors and commutators
for normalizable dagger Frobenius algebras in $\s{V}$ are completely positive,
making $(A, m, e, w, c) \m (B, m', e', w', c')
:= (A \m B, m \m m', e \m e', w \m w', c \m c')$ a symmetric monoidal product
for $\CPS[\s{V}]$ with unit $\mathbf{k}$\footnote{We skip the description of the
Frobenius structure of $\mathbf{k}$ because it contributes little to our
discussion.}.  Finally, daggers and duals of completely positive maps are also
completely positive and preserve normalizable dagger Frobenius structures while
satisfying all compatibility conditions to make $\CPS[\s{V}]$ dagger compact.
All of this can then be summarized in the following fundamental theorem
\cite[Thm. 3.3]{channels}.\footnote{The proofs of the claims made in this
paragraph can be found on pages 14--16 in \cite{channels}.}

\begin{thm}[{$\CPS$} Construction]
Given any dagger compact category $\s{V}$, the category whose objects are
normalizable dagger Frobenius algebras
$(A, m, e, w, c)\footnote{Note that the whole dagger Frobenius structure is an
object, as opposed to just $A$.} \in \s{V}$ and whose morphisms are completely
positive maps in $\s{V}$ is a well-defined dagger compact category, denoted as
$\CPS[\s{V}]$, whose monoidal product and unit are the same as in $\s{V}$.
\end{thm}

The central result in \cite{channels} concerning QIT, then, is the following.
\begin{thm}
There is an equivalence of categories $\CPS[\FHilb] \lr \CStar$ where $\CStar$
is the subcategory of usual $C^*$ algebras and completely positive maps in
$\Vect_{\C}$.
\end{thm}

Since the $\CPS$ construction completely describes $C^*$ algebras in the
abstract and at the same time Planar Field Theories are in bijection with
(not necessarily normalizable dagger) Frobenius algebras in $\Vect_{\C}$, it is
natural to ask: what happens when the image of the interval $I$ is a nomalizable
dagger Frobenius algebra? We proceed towards an answer to this question next.



\subsection{Normalizers for Planar Cobordisms}\label{pft}

We begin by observing that $\Thick$ (as well as $\Cob_d$ for all $d$) comes with
a natural dagger structure as follows. For a (planar) cobordism
$\fn{(M, a, b)}{X}{Y}$, if we consider $M^*$, then $a$ yields an orientation
preserving boundary inclusion $X \to M^*$ and $b$, an orientation reversing
boundary inclusion $Y \to M^*$. Thus, defining
$X^{\dagger} := X$, $Y^{\dagger} := Y$ and $M^{\dagger} := M^*$, it is easy to
see that $(-)^{\dagger}$ is a dagger on $\Thick$ (the same argument applies for
$\Cob_d$ for all $d$). A series of routine checks can be carried out to show
that this dagger structure is compatible with the duality in the following
sense:
\begin{align*}
(M \amalg N)^{\dagger} &\eqcob M^{\dagger} \amalg N^{\dagger}
  \text{, for all cobordisms } M, N, \\
\eps_{A}^{\dagger} &\eqcob \eta_{A^*}
  \text{, for every object } A \text{ and} \\
\text{the} & \text{ associators and unitors are all unitary.}
\end{align*}
so that $\Thick$ is a dagger compact category. We can further verify that PFT's
preserve this dagger compact structure. 

Normalizability for a Frobenius algebra $(A, m, e, w, c)$ in $\Vect_{\C}$
requires a central positive definite map $z$ on $A$ satisfying
$\eps_A(m \m \id_{A^*})(z^2 \m \eta_{A^*}) = c$. Now, suppose that $F$ is
the PFT defined by $F(I) = A$. For convenience, we define $M$ to be the planar
pair of pants; $E$, the planar disk or the cup; $W = M^*$, the planar
co-pair-of-pants; $C = E^*$, the planar codisk or the cap; $R$, the rectangle or
the identity on $I$. Just as for $(2, \C)$--TQFTs, we can see that the following
identities must hold:
\begin{align*}
  F(M) &= m &
  F(W) &= w \\
  F(E) &= e &
  F(C) &= c
\end{align*}

At this point, the fundamental question is this: when $A$ is a normalizable
dagger Frobenius algebra with normalizer $z$, is there a cobordism $I \to I$
whose image under $F$ is $z$. Let us suppose there is -- call it $Z$. Then, $Z$,
being a cobordism $I \to I$, must be (the class of) a surface embedded in the
plane with boundary $I^* \amalg I$ and $n$ holes for some $n \in \N$. In this
case, by the dagger strong monoidal functoriality of $F$, we have
\begin{align*}
F(\eps_{I} \circ (M \amalg E) \circ (Z^2 \amalg \eta_{I^*}))
  &= F\eps_I(FM \m F\id_{I^*})((FZ)^2 \m F\eta_{I^*})\\
  &= \eps_{A}(m \m \id_{A^*})(z^2 \m \eta_{A^*})\\
  &= c
\end{align*}

However, what is this manifold
$\eps_{I} \circ (M \amalg E) \circ (Z^2 \amalg \eta_{I^*})$? It is not hard to
see that it is the manifold obtained by glueing $Z^2$ to one of the legs of the
pair-of-pants and then glueing the ``output'' or ``waiste'' end to the other
leg -- a cobordism from $I$ to $\varnothing$ with $2n + 1$ holes, where $n$ is
the number of holes in $Z$. We will refer to such a manifold as a
$(2n + 1)$--pod\footnote{This is meant to be a reference to the picture of the
manifold on the plane -- a pea-pod with $n$ ``seeds''.}. The existence of such a
pod means that the PFT $F$ identifies a $(2n + 1)$--pod with the cap. We must
keep in mind that the $(2n + 1)$--pod is not the cap simply because of the
difference in genus -- a PFT that accounts for the normalizability of a
Frobenius algebra must simply make one of these identifications. Thus a morphism
in $\Thick$ (or $\Cob_d$) cannot be a normalizer as defined for algebras but,
accepting this, we may as well call a manifold $Z$ that maps to a normalizer
under a PFT, a normalizer of the interval in $\Thick$.

We note that this is by no means a complete picture. We have not said anything
about the existence of such a $Z$ for a given PFT $F$ nor about any method for
finding one or all of such manifolds. We also have not addressed the question of
whether the positive definiteness of a normalizer is accounted for by any
cobordism, although this is not strictly necessary for the possible connection
described above. Nevertheless, without much work, we can take the analogy a
little further. The following is a well known result \cite[Def. 2.5]{channels}:

\begin{thm}
For any $\mathbf{k}$--algebra A, a map $\fn{f}{A}{A}$ is central if and only if
there exists a central element $\fn{a}{\mathbf{k}}{A}$ of $A$ with
$f \circ x = m \circ (a \m x)$ for every element $\fn{x}{\mathbf{k}}{A}$.
\end{thm}

If $A$ is a normalizable Frobenius algebra with normalizer $z$, then there is an
element $\fn{a}{\mathbf{k}}{A}$ which gives rise to $z$ by multiplication, since
$z$ is central. Now, if under some PFT $F$, a cobordism $Z$ has image $z$, then
we see that $Z' := Z \circ E$ is a cobordism $\varnothing \to I$ with the same
number of holes as $Z$, say $n$. It is easy to see that
$M \circ (Z' \amalg E) \eqcob Z$. Thus, $Z'$ must have image $a$ under $F$ since
$a$ uniquely determines $z$, which is consistent with the analogy.

We finally note that the cobordisms $\varnothing \to I$, up to diffeomorphism,
are in bijection with the natural numbers -- each cobordism is mapped to its
genus. Therefore, given a Frobenius algebra $A$, the number of central maps of
$A$ and, by extension, normalizers that are accounted for by PFTs are countable
so that if there are uncountably many normalizers for a given Frobenius algebra,
the language of PFTs, as is, will not capture the normalizability of it
completely.


%
\subsection{Analogues of Common Constructs}



