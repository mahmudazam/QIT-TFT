
\subsection{Abstract \texorpdfstring{$C^*$}{C--star} Algebras}

For any finite dimensional $\mathbf{k}$--vector space $V$, the evaluation
$\fn{\eps_V}{V \m V^*}{\mathbf{k}}$ defined on any basis of
$V \m V^*$ by $a \m f \mapsto f(a)$ and the coevaluation
$\fn{\eta_V}{\mathbf{k}}{V^* \m V}$ defined by
$z \mapsto z\sum b_i^* \m b_i$ (where, the $b_i$ form a basis of $V$ and
the $b_i^*$ are the corresponding dual basis elements) can be verified to
satisfy the non-degeneracy condition of \ref{vect:nondegen}. That very theorem
states that if there is a non-degerate pairing of $V$ and $W$, then they are
each (isomorphic) to the dual of the other. The notion of duality is thus a
categorical one -- we can define the dual of an object $V$ in a monoidal
category to be another object $V^*$ that admits a non-degerate pairing with $V$.

\begin{defn}[Compact Category]
A monoidal category in which each object has a dual and a non-degenerate pairing
with its dual is called a compact category.
\end{defn}

\begin{exm}
$\Cob_d$ and $\Thick$ are compact categories precisely because of theorem
\ref{tqft:nondegen} (and its planar analogue), with duals given by orientation
reversal.
\end{exm}

\begin{exm}
$\Vect_{\mathbf{k}}$ is obviously compact.
\end{exm}

Given a $\mathbf{k}$--linear map $\fn{f}{A}{B}$, we can define a linear map
$\fnr{f^*}{B^*}{A^*}{g}{g \circ f}$. A straightforward but lengthy computation
then shows that the following diagram commutes:
\begin{eqnarray*}
\begin{tikzcd}[column sep=large]
& A^* \m \mathbf{k} \arrow[drr, bend left=10] & & \\
B^* \cong B^* \m \mathbf{k}
  \arrow[r, "1_{B^*} \m \eta_{A^*}"]
  \arrow[ur, bend left=20, "f^* \m 1_{\mathbf{k}}"]
  \arrow[dr, bend right=20] &
B^* \m A \m A^* \arrow[r, "1 \m f \m 1"] &
B^* \m B \m A^* \arrow[r, "\eps_{B^*} \m 1_{A^*}"] &
\mathbf{k} \m A^* \cong A^* \\
& \mathbf{k} \m B^*
  \arrow[urr, bend right=10, "1_{\mathbf{k}} \m f^*"] & &
\end{tikzcd}
\end{eqnarray*}
where the unlabelled arrows are the respective commutators for the tensor
product. By the various isomorphisms, we are also guaranteed that $f^*$ is the
unique map making the above commute. Hence we can take this to be the definition
of $f^*$, as done in \cite[6]{channels}.

Carrying on in this direction, if we denote the conjugate transpose of some
linear map $f$ as $f^{\dagger}$, we have $f^{\dagger \dagger} = f$. Furthermore,
we can verify that the conjugate of $f$ is $f_* = (f^{\dagger})^* =
(f^*)^{\dagger}$ If we now define $A^{\dagger} = A$ for each $A \in
\Vect_{\mathbf{k}}$, then it is easy to verify that $(-)^{\dagger}$ is a
contravariant functor on $\Vect_{\mathbf{k}}$ that is the identity on objects
and is involutive on morphisms and which satisfies that the associators, unitors
and commutators of $\m$ are all unitary\footnote{A linear map $u$ with
$u^{\dagger}$ acting as its two sided inverse is said to be unitary. One can
verify that these are precisely the norm-preserving maps on Hilbert spaces.},
$\eps_{A}^{\dagger} = \eta_{A^*}$ and
$(f \m g)^{\dagger} = f^{\dagger} \m g^{\dagger}$.

While the proofs of these claims may require us to use properties of vector
spaces and linear maps, the constructs themsleves are fundamentally categorical,
allowing us to define the following structures in monoidal categories that
fascilitate the formulation of $C^*$ algebras in completely catgorical terms
\cite{channels}. In what follows, we assume all objects and morphisms are in a
monoidal category $\s{V}$ with monoidal product $\m$ and unit object $U$.

\begin{defn}[Dagger Compact Category]
A contravariant functor $(-)^{\dagger}$ that is the identity on objects and is
involutive on morphisms is called a dagger. A category equipped with a dagger is
called a dagger category. A compact category equipped with a dagger satisfying
all the conditions as above is said to be dagger compact.
\end{defn}

\begin{defn}[Frobenius Monoid]
An object $A$ equipped with a $\m$--monoid structure $(m, e)$ and a
$\m$--comonoid\footnote{The structure obtained by reversing the arrows of a
monoid structure.} structure $(w, c)$ is said to be a Frobenius $\m$--monoid if
the following diagrams called Frobenius identities commute:
\begin{eqnarray*}
\begin{tikzcd}
A \m A \arrow[r, "w \m 1_A"] \arrow[d, "m" left] &
A \m A \m A \arrow[d, "1_A \m m" right] \\
A \arrow[r, "w" below] &
A \m A
\end{tikzcd} \hspace{5em} &
\begin{tikzcd}
A \m A \arrow[r, "1_A \m w"] \arrow[d, "m" left] &
A \m A \m A \arrow[d, "m \m 1_A" right] \\
A \arrow[r, "w" below] &
A \m A
\end{tikzcd}
\end{eqnarray*}
We say that $A$ is symmetric if $c \circ m \circ \sgm = c \circ m$, where $\sgm$
is the commutator $A \m A \to A \m A$.
\end{defn}

\begin{rmk}
This formulation of Frobenius monoids is equivalent to the one in terms of a
non-degenerate pairing \cite[Prop. 1]{NonCommTQFT}.
\end{rmk}

\begin{defn}[Dagger Frobenius Algebra]
If $\s{V}$ is equipped with a dagger, a Frobenius $\m$--monoid $(A, m, e, w, c)$
satisfying $w = m^{\dagger}$ and $c = e^{\dagger}$ is called a dagger Frobenius
algebra.
\end{defn}

\begin{defn}[Positive Definiteness]
A map $g$ in $\s{V}$ is said to be positive if there exists some map $h$ in the
category with $g = h^{\dagger} \circ h$, assuming $\s{V}$ has a dagger. Positive
isomorphisms are said to be positive definite.
\end{defn}

\begin{defn}[Centrality]
Given a $\m$--monoid $(A, m, e)$, a map $\fn{z}{A}{A}$ is said to be central if
$m \circ (z \m \id_A) = z \circ m = m \circ (\id_A \m z)$.
\end{defn}

\begin{defn}[Normalizable Frobenius Algebra]
A (not necessarily dagger) Frobenius algebra $(A, m, e, w, c)$ is said to be
normalizable if it is equipped with a central positive definite map
$\fn{z}{A}{A}$ satisfying:
\begin{eqnarray*}
\begin{tikzcd}[column sep=large]
A \m U \arrow[r, "c" above] \arrow[d, "z^2 \m \eta_{A^*}" left]&
U \\
A \m A \m A^* \arrow[r, "m \m \id_{A^*}" below] &
A \m A^* \arrow[u, "\eps_{A}" right]
\end{tikzcd}
\end{eqnarray*}
where $z^2 = z \circ z$.
\end{defn}

Coecke et al. \cite[10]{channels} have shown that finite direct sums of complex
square matrix algebras of the form $\bigoplus_{i = 1}^k \M_{n_i}$ (for some
$n_1, \dots, n_k \in \N$) with product given by matrix multiplication in every
component $\M_{n_i}$ are normalizable dagger Frobenius algebras. Furthermore, it
is known that every finite dimensional $C^*$ algebra is $*$--isomorphic to such
a direct sum algebra. Finally, \cite[Thm. 2.8]{channels} shows precisely that in
$\FHilb$ every normalizable dagger Frobenius algebra is a finite dimensional
$C^*$ algebra. It is then immediate that the finite dimensional complex $C^*$
algebras are precisely the normalizable dagger Frobenius algebras in $\FHilb$.
We build on this in the next subsection.

