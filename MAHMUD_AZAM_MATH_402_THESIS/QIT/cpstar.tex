
\subsection{The \texorpdfstring{$\CPS$}{CP Star} Construction}

In the usual sense, in $\Vect_{\mathbf{k}}$, a completely positive map
$\fn{f}{A}{B}$ of $C^*$ algebras is one which satisfies the condition that
$(f \m 1_{\M_n})(a)$ is a positive element of $B \m \M_{n}$ for all positive
elements $a$ of $A \m \M_n$, for all $n \in \N$. Using only morphisms, a
a map $f$ is completely positive if for all $n \in \N$ and for all
positive $\fn{a}{\mathbf{k}}{A \m \M_n}$,
$(f \m 1_{\M_n}) \circ a$ is a positive map $\mathbf{k} \to B \m \M_n$. In the
abstract setting, we may define complete positivity with an analogous statement
but with an arbitrary dagger Frobenius algebra $C$ in place of $\M_n$. Taking a
slightly different perspective, for any $X \in \Vect_{\mathbf{k}}$, it can be
verified that the map
$\fn{1_{X^*} \m \eps_X \m 1_X}{X^* \m X \m X^* \m X}{X^* \m X}$ gives $X^* \m X$
the structure of a normalizable dagger Frobenius algebra
\cite[Prop. 2.11]{channels} with unit $\eta_X$. We then observe that
$X^* \m X \cong \Hom_{\Vect_{\mathbf{k}}}(X, X)$ so that $X^* \m X$ can be
regarded as a matrix algebra and this formulation of matrix algebras works in
any abstract dagger compact category.  Then, we may define $f$ to be
completely positive if and only if $(f \m 1_{X^* \m X}) \circ a$ is a positive
map $\mathbf{k} \to B \m X^* \m X$ whenever $\fn{a}{\mathbf{k}}{A \m X^* \m X}$
is positive for every object $X$ in the category.

Coecke et al. have shown in \cite[Prop. 3.4]{channels} that both of these
abstract formulations are equivalent to satisfying a certain ``$\CPS$
condition'' in any dagger compact category and that they specialize to the usual
notion for finite dimensional complex $C^*$ algebras. Consider any dagger
compact category $\s{V}$ with product $\m$, unit $U$ and the usual symbols for
the dagger and the dual. Given any $C^*$ algebra -- by which we mean a
normalizable dagger Frobenius algebra -- $(A, m, e, w, c)$ in $\s{V}$ we define
the following notation:
\begin{eqnarray*}
  \wh{m} :=&
  \begin{tikzcd}[column sep=large]
    U \m A \arrow[r, "\eta_A \m \id_A"] &
    A^* \m A \m A \arrow[r, "\id_{A^*} \m m"] &
    A^* \m A
  \end{tikzcd}\\
  \wh{w} :=&
  \begin{tikzcd}[column sep=large]
    A^* \m A \arrow[r, "\id_{A^*} \m w"] &
    A^* \m A \m A \arrow[r, "\eps_{A^*} \m \id_{A}"] &
    U \m A
  \end{tikzcd}
\end{eqnarray*}
We now let $(A, m, e, w, c)$ and $(B, m', e', w', c')$ be $C^*$ algebras in
$\s{V}$ with a map $\fn{f}{A}{B}$ and observe that $\wh{m'} f \wh{w}$ can be a
well defined composition when unitors are placed appropriately.
\begin{defn}[{$\CPS$} Condition]
If there exists a map $\fn{g}{A}{X \m B}$ for some object $X \in \s{V}$
satisfying
\begin{equation}\label{cpstar:cond}
  \wh{m'}f\wh{w} = (1_{B*} \m \eps_{X^*} \m 1_{B}) \circ (g_* \m g)
\end{equation}
then $f$ is said to satisfy the $\CPS$ condition.
\end{defn}

\begin{rmk}
This is equation (2) is \cite[13]{channels}.
\end{rmk}

We thus define:
\begin{defn}[Completely Positive Map]
A map $\fn{f}{A}{B}$ of normalizable dagger Frobenius algebras in a dagger
compact category is said to be completely positive if it satisfies
\eqref{cpstar:cond}.
\end{defn}

Let $\CPS[\s{V}]$ denote the collection of normalizable dagger Frobenius
algebras in $\s{V}$ together with the completely positive maps
between them. The central results of \cite{channels} are as follows.
The completely positive maps in $\s{V}$ compose to give completely positive,
making $\CPS[\s{V}]$ a category. Next, the associators, unitors and commutators
for normalizable dagger Frobenius algebras in $\s{V}$ are completely positive,
making $(A, m, e, w, c) \m (B, m', e', w', c')
:= (A \m B, m \m m', e \m e', w \m w', c \m c')$ a symmetric monoidal product
for $\CPS[\s{V}]$ with unit $\mathbf{k}$\footnote{We skip the description of the
Frobenius structure of $\mathbf{k}$ because it contributes little to our
discussion.}.  Finally, daggers and duals of completely positive maps are also
completely positive and preserve normalizable dagger Frobenius structures while
satisfying all compatibility conditions to make $\CPS[\s{V}]$ dagger compact.
All of this can then be summarized in the following fundamental theorem
\cite[Thm. 3.3]{channels}.\footnote{The proofs of the claims made in this
paragraph can be found on pages 14--16 in \cite{channels}.}

\begin{thm}[{$\CPS$} Construction]
Given any dagger compact category $\s{V}$, the category whose objects are
normalizable dagger Frobenius algebras
$(A, m, e, w, c)\footnote{Note that the whole dagger Frobenius structure is an
object, as opposed to just $A$.} \in \s{V}$ and whose morphisms are completely
positive maps in $\s{V}$ is a well-defined dagger compact category, denoted as
$\CPS[\s{V}]$, whose monoidal product and unit are the same as in $\s{V}$.
\end{thm}

The central result in \cite{channels} concerning QIT, then, is the following.
\begin{thm}
There is an equivalence of categories $\CPS[\FHilb] \lr \CStar$ where $\CStar$
is the subcategory of usual $C^*$ algebras and completely positive maps in
$\Vect_{\C}$.
\end{thm}

Since the $\CPS$ construction completely describes $C^*$ algebras in the
abstract and at the same time Planar Field Theories are in bijection with
(not necessarily normalizable dagger) Frobenius algebras in $\Vect_{\C}$, it is
natural to ask: what happens when the image of the interval $I$ is a nomalizable
dagger Frobenius algebra? We proceed towards an answer to this question next.

