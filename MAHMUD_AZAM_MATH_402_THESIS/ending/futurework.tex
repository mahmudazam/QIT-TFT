
\pagebreak

\section{Future Work}

The crux in connecting topological field theories with quantum information is
answering the following question. Given a PFT $F$ defined by $F(I) = A$ for some
normalizable dagger Frobenius $\C$--algebra $A$ with normalizer $z$, when is $z$
the image of some planar cobordism $Z$ under $F$? It is unlikely that the answer
to this question is ``always'' for if the same algebra $A$ has more than one
normalizer, at most one can arise in this way since exactly one PFT maps the
interval $I$ to $A$. Apart from identifying the necessary and sufficient
conditions for normalizers of dagger Frobenius algebras arising from PFTs, we
may explore what modifications we can make to the definition of PFTs and their
domain categories that facillitates accounting for normalizers.

To this end, we may examine the category obtained by identifying a $(2n +
1)$--pod, as defined in \ref{pft}, with the cup -- can we make sense of this as
a well defined category?  Does the dagger compact monoidal structure carry over?
Given that we can answer these questions in the affirmative, if there is a
functor from this category to $\Vect_{\C}$ taking $I$ to a Frobenius algebra, we
see that this algebra must be normalizable. In this case, we may examine what
different choices of $n$ imply for this new theory. Finally, we may examine what
the graphical calculus outlined in \cite{NonCommTQFT} for $\Thick$ transforms
into for our new category.

With these aspects solidified, we may translate common constructs of quantum
information, framed in terms of $C^*$ algebras of operators, into constructs of
this new category of cobordisms. For instance, we immediately see that preimages
of operators must be cobordisms $\varnothing \to I$ but it is not yet clear
where quantum registers fit into this picture. Having clarified these essential
relations, an interesting direction to take the theory further will be to
develop a graphical calculus for quantum circuits in terms of cobordisms and to
explore if this new calculus yields simplifications of circuits.

