
\subsection{Planar TQFTs}

A generalization of TQFTs to a form that captures all Frobenius algebras is in
order. This begins with the observation that a $\mathbf{k}$--algebra is exactly
a monoid object with respect to the tensor product of $\Vect_{\mathbf{k}}$ and
in defining a Frobenius algebra we did not refer to any property specific to
vector spaces. This allows us to generalize Frobenius algebras to arbitrary
monoidal categories. If a monoid object in any monoidal category is equipped a
pairing and a copairing, as in the case of vector spaces, which satisfy
non-degeneracy identities analogous to diagram \eqref{frob:s1nondegen}, we call
it a Frobenius monoid. Thus, $S^1$ is precisely a Frobenius monoid in $\Cob_2$.

\begin{defn}
Objects $O_1, \dots, O_k$, morphisms $f_1, \dots, f_m$ and propositions
$P_1, \dots, P_n$ involving the given objects and morphisms in a category
$\s{C}$, is called a structure in $\s{C}$. The smallest subcategory $\s{D}$ of
$\s{C}$ containing a structure is called the category generated by that
structure. Finally, $\s{D}$ is said to be freely generated if functors (possibly
with additional structure) $F : \s{D} \to \s{E}$ are in bijection with the
possible choices for images under $F$ of the objects defining the structure.
\end{defn}

The last paragraph in the prvious section argues precisely that $\Cob_2$ is the
symmetric monoidal category freely generated by the commutative Frobenius monoid
$S^1$. The fact that the image $A$ of $S^1$ under a TQFT is a commutative
Frobenius algebra is precisely because the commutator cobordism
$S^1 \amalg S^1 \to S^1 \amalg S^1$ under the TQFT enforces commutativity on the
multiplication of $A$. We note that this commutator can be embedded in three
real dimensions but not two for this commutator is made of two cylinders, one
going ``over'' the other, which, when embedded in a two dimensional surface,
must overlap. As we will see shortly, this is precisely the way to remove
commutators from the category of cobordisms.

As noted before, the compact $1$--manifolds, up to diffeomorphism, are precisely
disjoint unions of $S^1$ or disjoint unions of the closed interval. We now
define a thick tangle or planar cobordism to be an oriented smooth $2$--manifold
embedded in the plane and whose boundaries are disjoint unions of the unit
interval $I$.  Analogous arguments as in the development of $\Cob_d$ then show
that planar cobordisms form a monoidal category with composition given by gluing
at common boundaries and monoidality given by disjoint union and the empty set.

\begin{defn}[Thick Tangles]
We denote the category of planar cobordisms (or thick tangles) as $\Thick$.
Monoidal functors out of $\Thick$ are said to be planar TQFTs or simply planar
field theories or PFTs.
\end{defn}

It can be shown, by very similar arguments as for $S^1$ in $\Cob_2$, that $I$ is
a Frobenius monoid in $\Thick$. In \cite{NonCommTQFT}, Lauda shows that $\Thick$
is equivalent to the monoidal category freely generated by a Frobenius monoid
and notes that a restatement of this is the following:

\begin{thm}
Given a PFT, $\fn{Z}{\Thick}{\Vect_{\mathbf{k}}}$, there is a unique Frobenius
algebra $A = Z(I)$. Conversely, given a Frobenius algebra $A$, we have a unique
PFT $Z$ determined by $Z(I)$.
\end{thm}

This theorem establishes that all Frobeniusm algebras, commutative or not, are
accounted for by PFTs. A corollary of this is that $\Thick$ is monoidal but
cannot have a commutator because if it did, the same argument as for TQFTs would
show that PFTs are equivalent to commutative Frobenius algebras, a
contradiction.

