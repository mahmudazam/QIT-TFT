
\subsection{Commutative Frobenius Algebras}\label{sec:frob}

TQFTs of dimension $2$ give rise to (and arise from) a simple well-known
algebraic structure. We now build our way to this result which will prove to be
the inital point of connection with quantum information. It is known that
connected $1$--dimensional manifolds are diffeomorphic to either the circle
$S^1$ or some real interval \cite{MilnorDiff}. Since a real interval either has
a boundary or is non-compact, the objects in $\Cob_2$ are diffeomorphic to
disjoint unions of $S^1$. Given that diffeomorphisms are isomorphisms in
$\textbf{Diff}_{d - 1}$ and functors preserve isomorphisms, the functor $R$
defined in \ref{disjunion:cylinderconst} yields isomorphisms in $\Cob_2$ from
any object to a disjoint union of circles. The monoidal functoriality of
$2$--dimensional TQFTs, then, establishes the following fundamental facts as
easy corollaries.

\begin{cor}
Let $Z$ be a $(2, \mathbf{k})$--TQFT and $X$, any object in $\Cob_2$. Then, for
some index set $I$, we have $X \cong_{\Cob_d} \coprod_{i \in I} S^1$ and hence,
$Z(X) \cong_{\Vect_{\mathbf{k}}} \bigotimes_{i \in I}Z(S^1)$.
\end{cor}

\begin{cor}
The object function of a $(2, \mathbf{k})$--TQFT is determined completely (up to
isomorphism) by the image of $S^1$.
\end{cor}

For a particular $(2, \mathbf{k})$--TQFT $Z$, we let $A := Z(S^1)$ and then
examine the action of $Z$ on cobordisms. It is well known that every
$2$--dimensional manifold with boudary has a collection of loops that cut it to
yield a decomposition into the following shapes (up to diffeomorphism)
\cite{Atiyah2}:
\begin{enumerate}[(i)]
  \item the cylinder $C$ (on $S^1$),
  \item the pair of pants $P$ (a cobordism $S^1 \amalg S^1 \to S^1$ of genus
  $0$) and
  \item the disk $D$ (a cobordism $S^1 \to \varnothing$ or vice versa).
\end{enumerate}
Thus, the action of a $(2, \mathbf{k})$--TQFT $Z$ on cobordisms is given by
composing $Z(C)$, $Z(P)$ and $Z(D)$ in all possible ways.

Visibly, $\fn{Z(P)}{A \otimes A}{A}$ uniquely determines a bilinear map
$\fn{m}{A \times A}{A}$. It is easy to construct (componentwise) a
diffeomorphism $P \circ (C \amalg P) \to P \circ (P \amalg C)$ that restricts
to the identity on the boundary such that these are equivalent cobordisms. Under
a TQFT, this yields:
\begin{align*}
  Z(P) \circ (Z(C) \otimes Z(P)) = Z(P) \circ (Z(P) \otimes Z(P))
\end{align*}
which extends to the associativity of $m$, seen as a bilinear multiplication on
$A$. If we take $E$ as the cobordism $\varnothing \to S^1$ obtained by reversing
the orientation of $D$, then by observing that
$P \circ (E \amalg C) \eqcob C \eqcob P \circ (C \amalg E)$, we get:
\begin{align*}
  Z(P) \circ (Z(E) \otimes \id_{A})
    &= Z(P) \circ (Z(E) \otimes Z(C))\\
    &= Z(P \circ (E \amalg C))\\
    &= Z(C)\\
    &= Z(P \circ (C \amalg E))\\
    &= Z(P) \circ (Z(C) \otimes Z(E))\\
    &= Z(P) \circ (\id_{A} \otimes Z(E))
\end{align*}
so that $Z(E)$ acts as the identity when $Z(P)$ is seen as a bilinear
multiplication. Furhtermore, let $\sgm_{X, Y}$ be the commutator diffeomorphism
$X \amalg Y \to Y \amalg X$ in $\textbf{Diff}_{1}$. Then, $R(\sgm_{X, Y})$ is
the commutator cobordism in $\Cob_2$. We then observe that
$R(\sgm_{A, A}) \circ P \eqcob P$ yielding
\[
  Z(R(\sgm_{A, A})) \circ Z(P) = Z(P)
\]
Since $R(\sgm_{A, A})$ is a commutator in $\Cob_2$ by construction and $Z$ is a
symmetric monoidal functor, $Z(R(\sgm_{A, A}))$ must be the commutator of
$\Vect_{\mathbf{k}}$. Together, these establish:

\begin{thm}
$Z(P)$ induces an associative and commutative bilinear multiplication $m$ on $A$
with unit $Z(C)$, making $A$ a commutative $\mathbf{k}$--algebra.
\end{thm}

An additional structure on $A$ gives it the form which allows us to bridge TQFTs
with quantum information. For this, we define:
\begin{defn}
For a $\mathbf{k}$--algebra $A$ with multipliction $m$, a linear map
$\fn{\beta}{A \otimes A}{\mathbf{k}}$ satisfying $\beta \circ (m \otimes \id_A)
= \beta \circ (\id_A \otimes m)$ is called an associative pairing of $A$ with
itself.
\end{defn}

\begin{defn}[Frobenius Algebra]
A $\mathbf{k}$--algebra $A$ equipped with an associative pairing which is also
non-degenerate in the sense of \ref{vect:nondegen:def} is called a Frobenius
algebra.
\end{defn}

We note that for an associative pairing to be non-degenerate, it must have an
associated copairing, by definition. Setting $B = D \circ P$ and $\beta = Z(B)$,
we observe that $B \circ (C \amalg P) \eqcob B \circ (P \amalg C)$ -- the
argument is similar to that for the associativity of $P$ -- yielding
\[
  \beta \circ (Z(P) \amalg Z(C)) = \beta \circ (Z(C) \amalg Z(P))
\]
We can further set $B' := (D \circ P)^* \eqcob P^* \circ D^*$ and
$\alpha := Z(B')$, and show that the following diagrams commute using an
argument identical to the proof of \ref{tqft:nondegen}:
\begin{eqnarray}\label{frob:s1nondegen}
\begin{tikzcd}[column sep=large]
  S^1 \amalg \varnothing
    \arrow[rd, "\sgm" above right]
    \arrow[d, "\id_{S^1} \amalg B" left] & \\
  S^1 \amalg S^1 \amalg S^1
    \arrow[r, "B' \amalg \id_{S^1}" below] &
  \varnothing \amalg S^1
\end{tikzcd} &
\begin{tikzcd}[column sep=large]
  \varnothing \amalg S^1
    \arrow[r, "B' \amalg \id_{S^1}" above]
    \arrow[rd, "\sgm" below left] &
  S^1 \amalg S^1 \amalg S^1
    \arrow[d, "\id_{S^1} \amalg B"]\\
  & S^1 \amalg \varnothing
\end{tikzcd}
\end{eqnarray}
Applying $Z$ to these diagrams shows that $\beta$ is an associative,
non-degenerate pairing of $A = Z(S^1)$ with itself with copairing $\alpha$,
establishing $A$ as a commutative Frobenius algebra with multiplication $m$
induced by $Z(P)$, unit $e := Z(C)$ and pairing $\beta$. We stress that the
commutativity is due to the commutator in $\Cob_2$.

Conversely, if we have a commutative Frobenius algebra $(A, m, e, \beta)$, we
can set $Z$ accordingly, giving rise to a $(2, \mathbf{k})$--TQFT by (finite)
glueing of disjoint unions and orientation reversals of the generating objects
and morphisms in $\Cob_2$. Finally, we can further show that monoidal natural
transformations between $2$--dimensional TQFTs give rise to homomorphisms of
commutative Frobenius algebras and vice versa. This correspondence can be
further shown to be an equivalence of categories, the details of which can be
found in \cite{Jorge}.

