
\pagebreak

\section{Topological Quantum Field Theories}

A topological quantum field theory or TQFT is a functor with some additional
structure from a certain category of manifolds to the category of vector spaces
over some field $\mathbf{k}$. We first make precise what the domain categories
of TQFTs must be but we will only develop the theory to a level sufficient to
arrive at the result we require -- namely that $2$--dimensional topological
quantum field theories are equivalent to a certain class of algebras.

In the following, we take all manifolds to be smooth and oriented, and all
diffeomorphisms to be orientation preserving, unless otherwise stated. For
closed manifolds, $X$ and $Y$, we will write $X \amalg Y$ to denote their
disjoint union. $X^*$ is defined to be the manifold obtained by reversing the
orientation of $X$. We write $I$ to denote the real interal $[0, 1]$. If $X$ has
a boundary, we write $\partial X$ to denote the boundary of $X$. With these
notational conventions we construct the domain category of a TQFT, based on
\cite{Corominas}, \cite{Jorge} and \cite{MayTQFT}.


\subsection{Cobordisms}

We first define some fundamental structures arising from manifolds. For $d > 0$
and closed $(d - 1)$--manifolds $X$ and $Y$, let $M$ be a compact $d$--manifold
such that $\partial M = W_0 \amalg W_1$ and $\fn{\iota_0}{X}{M}$ and
$\fn{\iota_1}{Y}{M}$ are smooth maps that are diffeomorphisms onto $W_0$ and
$W_1$, respectively, with the added assumption that $\iota_0$ is
\textit{orientation reversing}\footnote{This is one of the rare situations,
where we assume a diffeomorphism does not preserve orientation.}. It should be
noted that we could take $\iota_0$ to be an orientation preserving
diffeomorphism $X^* \to M$ but this would complicate our notation and diagram
chases later on.

\begin{defn}[{$d$}--Cobordism]
In the above scenario, $(M, \iota_0, \iota_1)$ or simply $M$ is called a
$d$--cobordism or simply a cobordism (when $d$ is clear from context) from $X$
to $Y$, and is denoted as $\fn{(M, \iota_0, \iota_1)}{X}{Y}$ or simply
$\fn{M}{X}{Y}$.
\end{defn}

\begin{defn}[Equivalent Cobordisms]\label{cob:equiv}
Let $M$ and $M'$ be compact $d$--manifolds with $\partial M = W_0 \amalg W_1,
\partial M' = W_0' \amalg W_1'$ such that diffeomorphisms
$\fn{\iota_0}{X}{M}, \fn{\iota_1}{Y}{M}, \fn{\iota_0'}{X}{M'},
\fn{\iota_1'}{Y}{M'}$ realize $M$ and $M'$ as cobordisms $X \to Y$. Let
$\fn{f}{M}{M'}$ be a diffeomorphism making the following sqaure commute:
\[\begin{tikzcd}
  & X \arrow{dl}[above, xshift=-5pt]{\iota_0} \arrow[dr, "\iota_0'"] & \\
  M \arrow[rr, "f"] & & M'\\
  & Y \arrow[ul, "\iota_1"] \arrow[ur, "\iota_1'" below] &
\end{tikzcd}\]
Then, we say $M$ and $M'$ are equivalent cobordisms $X \to Y$ and write this as
the relation $(M, \iota_0, \iota_1) \eqcob (M', \iota_0', \iota_1')$
or simply $M \eqcob M'$.
\end{defn}

\begin{rmk}
If we only have $\fn{(M, \iota_0, \iota_1)}{X}{Y}$ but a diffeomorphism
$\fn{f}{M}{M'}$ for an arbitrary $d$--manifold $M'$ such that $f$ restricts to
the identity function on $\partial M$, then we must have
$\partial M' = \partial M$. As a result, $M'$ is also a cobordism $X \to Y$
with boundary inclusions $f \circ \iota_i$ for $i = 0, 1$ and $M \eqcob M'$.
\end{rmk}

\begin{thm}
$\eqcob$ is an equivalence relation.
\end{thm}
\begin{proof}
Reflexivity is obvious by taking the identity diffeomorphism and symmetry
follows from taking the inverse of $f$ in the following square:
\[\begin{tikzcd}
  & X \arrow{dl}[above, xshift=-5pt]{\iota_0} \arrow[dr, "\iota_0'"] & \\
  M & & M' \arrow[ll, "f^{-1}"] \\
  & Y \arrow[ul, "\iota_1"] \arrow[ur, "\iota_1'" below] &
\end{tikzcd}\]
which commutes because the square in \ref{cob:equiv} commutes. The following
pasting shows transitivity:
\[\begin{tikzcd}[row sep=large, column sep=large]
  & X
    \arrow{dl}[above, xshift=-5pt]{\iota_0}
    \arrow[d, "\iota_0'"]
    \arrow[dr, "\iota_0''"] &\\
  M \arrow[r, "f"] & M' \arrow[r, "g"] & M''\\
  & Y
    \arrow[ul, "\iota_1"]
    \arrow[u, "\iota_1'"]
    \arrow{ur}[below, xshift=5pt]{\iota_1''} &
\end{tikzcd}\]
\end{proof}

We then define what will be the morphisms of our domain category by updating our
definition of cobordism to refer to the equivalence classes under
$\eqcob$.
\begin{defn}[{$d$}--Cobordism (Class)]
Given a compact $d$--manifold $M$ with $\fn{M}{X}{Y}$ as above, from this point
onwards, we will also call the equivalence class of $M$ under $\eqcob$ a
$d$--cobordism or simply a cobordism from $X$ to $Y$ when $d$ is clear from
context. We continue to write $\fn{M}{X}{Y}$ to mean that the class of $M$ is a
cobordism from $X$ to $Y$.
\end{defn}



\subsection{Gluing Cobordisms}

If we have cobordisms $\fn{(M, \iota_{0, M}, \iota_{1, M})}{X}{Y}$ and
$\fn{(N, \iota_{0, N}, \iota_{1, N})}{Y}{Z}$, we can define a cobordism
$\fn{(N \circ M, \iota_{0, N \circ M}, \iota_{1, N \circ M})}{X}{Z}$ by taking a
$d$--manifold $N \circ M := N \amalg_{Y} M$ obtained by gluing $M$ and $N$ along
the common boundary component (diffeomorphic to) $Y$ -- we note that we have to
make precise what the inclusions $\iota_{k, N \circ M}$ ($k = 0, 1$) must be but
we defer this until we have some more context. We first precisely define gluing
\cite{Jorge}.
\begin{defn}[Gluing of Cobordisms]
Given $\fn{(M, \iota_{0, M}, \iota_{1, M})}{X}{Y}$ and
$\fn{(N, \iota_{0, N}, \iota_{1, N})}{Y}{Z}$. We define $N \circ M$ to be the
pushout of $N \ot[\iota_{0, N}] Y \to[\iota_{1, M}] M$ in $\Man$, the category
of manifolds and smooth maps:
\[\begin{tikzcd}
  Y \arrow[r, "\iota_{1, M}"] \arrow[d, "\iota_{0, N}" left]
  & M \arrow[d, "p_0"]\\
  N \arrow[r, "p_1" below]
  & N \circ M
\end{tikzcd}\]
\end{defn}

\begin{rmk}
We can make this definition concrete if needed. We consider the sets
$S_1 = \im \iota_{1, M} \subset M$ and $S_0 = \im \iota_{0, N} \subset N$
diffeomorphic to $Y$. Then,
$N \circ M = (N \setminus S_0) \cup (M \setminus S_1) \cup S$ where $S$ can be
either $S_0$ or $S_1$. Without loss of generlity, we can take $S = S_1$. We then
have $p_0(x) = x$ for all $x \in M$ and $p_1(x) = x$ if $x \not\in S_0$ and
$p_1(x) = p_0(\iota_{1, M}^{-1}(x))$ otherwise -- noting that
$\iota_{1, M}^{-1}(x)$ is a unique element of $Y$ when $x \in S_0$ since
$\iota_{1, M}$ is injective. It is easy to see that $\im p_0 = M$,
$\im p_1 = (N \setminus S_0) \cup S_1$, $\im p_1 \cap \im p_0 = S_1$ and
$N \circ M = \im p_1 \cup \im p_0$. Finally, although this is the same
definition as in $\Set$, it can be shown that $p_0$ and $p_1$ are smooth maps --
in fact, diffeomorphisms onto their images.
\end{rmk}

It is not clear that $N \circ M$ is well-defined since it seems possible that,
given any other representatives $(M', \iota_{0, M'}, \iota_{1, M'})$ and $(N',
\iota_{0, N'}, \iota_{1, N'})$ of the cobordism classes of $M$ and $N$
respectively, the class of $N' \circ M'$ might be different from that of $N
\circ M$, but we have the following lemma.

\begin{lem}\label{cobglue:welldef}
If $M \eqcob M'$ and $N \eqcob N'$, then $N \circ M \eqcob N' \circ M'$.
\end{lem}
\begin{proof}
We observe that there are diffeomorphisms
$\fn{f}{M}{M'}$ and $\fn{g}{N}{N'}$ making the relevant analogues of the square
in \ref{cob:equiv} commute so that $f \circ \iota_{1, M} = \iota_{1, M'}$ and
$g \circ \iota_{1, N} = \iota_{1, N'}$. Then, we have that $N \circ M$ is the
pushout of $M \ot[\iota_{1, M}] Y \to[\iota_{0, N}] N$ in $\Man$ and similarly
$N' \circ M'$ is the pushout of $M' \ot[\iota_{1, M'}] Y \to[\iota_{0, N'}] N'$
in $\Man$, by the definition of gluing. Together, these yield the following
pasting of commutative diagrams:
\begin{equation}\label{cob:gluing}
\begin{tikzcd}[column sep=large, row sep=large]
  Y \arrow{r}{\iota_{1, M}} \arrow{d}[right]{\iota_{0, N}}
      \arrow[rr, bend left, "\iota_{1, M'}"]
      \arrow[dd, bend right, "\iota_{0, N'}" left]
  & M \arrow{r}{f} \arrow{d}{p_0}
  & M' \arrow{dd}{p_0'} \\
  N \arrow{r}[below]{p_1} \arrow{d}[right]{g}
  & N \circ M \arrow[dr, dashed, shift left=1ex, "\phi" description]
  & \\
  N' \arrow{rr}[below]{p_1'}
  &
  & N' \circ M' \arrow[ul, dashed, shift left=1ex, "\psi" description]
\end{tikzcd}
\end{equation}
where the squares $Y, M, N \circ M, N$ and $Y, M', N' \circ M', N'$ being
pushouts yield unique smooth maps $\phi$ and $\psi$ making the above diagram
commute. At this point, if we define boundary inclusions
$\iota_{0, N \circ M} := p_0 \circ \iota_{0, M}$ and
$\iota_{1, N \circ M} := p_1 \circ \iota_{1, N}$ with similar definitions for
$\iota_{k, N' \circ M'}$ ($k = 0, 1$), by pasting the triangles for $X$ and $Z$
which realize $M \eqcob M'$ and $N \eqcob N'$ to the
squares $M, M', N' \circ M', N \circ M$ and $N, N \circ M, N' \circ M', N'$
respectively in diagram \ref{cob:gluing}, we get:
\[
\begin{tikzcd}
  & X \arrow{dl}[above,xshift=-2.5ex]{\iota_{0, M}} \arrow[dr, "\iota_{0, M'}"]
    & \\
  M \arrow[rr, "f"] \arrow[d, "p_0" left] & & M' \arrow[d, "p_0'" right]\\
  N \circ M \arrow[rr, shift left=1ex, "\phi"] &
    & N' \circ M' \arrow[ll, shift left=1ex, "\psi"]\\
  N \arrow[rr, "g"] \arrow[u, "p_1" left] & & N' \arrow[u, "p_1'" right]\\
  & Z \arrow[ul, "\iota_{1, N}"] \arrow{ur}[below,xshift=2.5ex]{\iota_{1, N'}} &
\end{tikzcd} \implies
\begin{tikzcd}
  & X
    \arrow{dl}[left, yshift=5pt]{\iota_{0, N \circ M}}
    \arrow[dr, "\iota_{0, N' \circ M'}"] & \\
  N \circ M \arrow[rr, shift left=1ex, "\phi"] &
    & N' \circ M' \arrow[ll, shift left=1ex, "\psi"]\\
  & Z
    \arrow{ul}{\iota_{1, N \circ M}}
    \arrow{ur}[right, yshift=-5pt]{\iota_{1, N' \circ M'}}
\end{tikzcd}
\]

It now suffices to show that $\phi$ and $\psi$ are inverses in $\Man$. We first
observe from the diagram above that $\psi p_0' f = p_0$ and
$\phi p_0 f^{-1} = p_0'$ yielding $\psi \phi p_0 = p_0$. We can similarly show
that $\psi \phi p_1 = p_1$. The pushout property of $N \circ M$ now yields that
$\psi\phi = \id_{N \circ M}$. A similar argument shows that
$\phi\psi = \id_{N' \circ M'}$.
\end{proof}

We can prove that $- \circ -$ is an associative operation when defined.
\begin{lem}\label{cobglue:assoc}
For cobordisms $\fn{(L, \iota_{0, L}, \iota_{1, L})}{W}{X}$,
$\fn{(M, \iota_{0, N}, \iota_{1, N})}{X}{Y}$ and
$\fn{(N, \iota_{0, N}, \iota_{1, N})}{Y}{Z}$,
\[
  N \circ (M \circ L) \eqcob (N \circ M) \circ L
\]
\end{lem}
\begin{proof}
The pushout diagrams for $M \circ L$ and $N \circ M$ paste to give:
\[\begin{tikzcd}
  & X \arrow[r, "\iota_{1, L}" above] \arrow[d, "\iota_{0, M}" left]
  & L \arrow[d, "p_0" right] \\
  Y \arrow[r, "\iota_{1, M}" above] \arrow[d, "\iota_{0, N}" left]
  & M \arrow[r, "p_1" below] \arrow[d, "q_0" right]
  & M \circ L \\
  N \arrow[r, "q_1" below]
  & N \circ M & \\
\end{tikzcd}\]

Pushing out $N \ot[\iota_{0, N}] Y \to[\iota_{1, M}] M \to[p_1] M \circ L$
and $N \circ M \ot[q_0] M \ot[\iota_{0, M}] X \to[\iota_{1, L}] L$, we get:
\[\begin{tikzcd}
  & X \arrow[r, "\iota_{1, L}" above] \arrow[d, "\iota_{0, M}" left]
  & L \arrow[d, green!55!black, "p_0" right] \arrow[rdd, bend left, "s_0"] \\
  Y \arrow[r, "\iota_{1, M}" above] \arrow[d, "\iota_{0, N}" left]
  & M \arrow[r, "p_1" below] \arrow[d, "q_0" right]
  & M \circ L \arrow[dd, green!55!black, near end, "r_0"]
      \arrow[rd, dashed, red, "\alpha" description]\\
  N \arrow[r, red, "q_1" below] \arrow[rrd, bend right, "r_1"]
  & N \circ M \arrow[rr, near start, crossing over, red, "s_1"]
      \arrow[rd, dashed, green!55!black, "\beta" description]
  &
  & (N \circ M) \circ L
      \arrow[dl, dashed, shift right=1ex, "\phi" description] \\
  & & N \circ (M \circ L)
      \arrow[ur, dashed, shift right=1ex, "\psi" description] &
\end{tikzcd}\]
where $\alpha$ and $\beta$ are smooth maps obtained by the pushout properties of
$M \circ L$ and $N \circ M$ respectively. The green paths in the diagram yield
a smooth map $\fn{\phi}{(N \circ M) \circ L}{N \circ (M \circ L)}$ by the
pushout property of $(N \circ M) \circ L$. Similarly, the red paths yield a
smooth map $\fn{\psi}{N \circ (M \circ L)}{(N \circ M) \circ L}$.

By our definition of gluing of cobordisms, the above diagram yields:
\begin{align*}
  \iota_{0, (N \circ M) \circ L} =& s_0 \iota_{0, L}
    & \iota_{1, (N \circ M) \circ L} =& s_1 \iota_{1, N \circ M} \\
  =& \psi r_0 p_0 \iota_{0, L}
    & =& s_1 q_1 \iota_{1, N} \\
  =& \psi r_0 \iota_{0, M \circ L}
    & =& \psi r_1 \iota_{1, N} \\
  =& \psi \iota_{0, N \circ (M \circ L)}
    & =& \psi \iota_{1, N \circ (M \circ L)}
\end{align*}
This and a similar argument involving $\phi$, $\iota_{k, (N \circ M) \circ L}$
and $\iota_{k, N \circ (M \circ L)}$ for $k \in \set{0, 1}$ shows that the
following diagram commutes:
\[\begin{tikzcd}
  & W
      \arrow{ld}[left, yshift=5pt]{\iota_{0, (N \circ M) \circ L}}
      \arrow[rd, "\iota_{0, N \circ (M \circ L)}"] & \\
  (N \circ M) \circ L
      \arrow[rr, shift left=1ex, "\phi"]
  &
  & N \circ (M \circ L)
      \arrow[ll, shift left=1ex, "\psi"] \\
  & Z
      \arrow[lu, "\iota_{1, (N \circ M) \circ L}"]
      \arrow{ru}[right, yshift=-5pt]{\iota_{1, N \circ (M \circ L)}} &
\end{tikzcd}\]

Similar to \ref{cobglue:welldef}, using the universal property of pushouts, we
can show that $\psi\phi = \id_{(N \circ M) \circ L}$ and
$\phi\psi = \id_{N \circ (M \circ L)}$.
\end{proof}

\begin{rmk}
The proofs of \ref{cobglue:welldef} and \ref{cobglue:assoc} apply in any
category with pushouts and show that the pushing out of two maps can be
``extended via isomorphisms'' as in \ref{cobglue:welldef} and, when seen as a
binary operation, it is associative up to isomorphism. Reversing all arrows in
these proofs yields dual results for pullbacks.
\end{rmk}

Furthermore, for any $(d - 1)$--manifold $X$, the $d$--manifold $X \times I$,
called the cylinder on $X$, has boundary
$(X \times \{0\})^* \amalg (X \times \{1\})$ so that
$\fn{(X \times I, \iota_0, \iota_1)} {X}{X}$ is a $d$--cobordism where $\iota_0$
is the orientation reversing diffeomorphism $X \to (X \times \{0\})^*$ and
$\iota_1$, the inclusion of $X$ into $X \times I$ with image $X \times \{1\}$.
By \cite[339, Thm. 2]{collar}, for some cobordism $\fn{M}{W}{X}$ there is an
open set $U \subset M$ such that $X \times I$ is diffeomorphic to the closure of
$U$, allowing us to extend the identity function on $M$ to a diffeomorphism
$M \to M \circ (X \times I)$ which restricts to the identities on $X$ and $W$
establishing $M \circ (X \times I) \eqcob M$. A similar construction yields
$(X \times I) \circ N \eqcob N$ for some cobordism $\fn{N}{X}{Y}$. This shows:

\begin{lem}\label{cobglue:id}
For any manifold $X$, $X \times I$ acts as a two sided identity for $X$ with
respect to $- \circ -$.
\end{lem}

\ref{cobglue:welldef}, \ref{cobglue:assoc} and $\ref{cobglue:id}$ then establish
the following as a corollary.

\begin{thm}\label{cobcat}
$d$--cobordisms taken as morphisms form a category with $(d - 1)$--manifolds as
the objects and composition given by the gluing operation $- \circ -$.
\end{thm}

\begin{defn}[{$\Cob_d$}]
The category in \ref{cobcat} is called the category of $d$--cobordisms and is
denoted $\Cob_d$.
\end{defn}

\begin{rmk}
We note that gluing of cobordisms is not a pushout in $\Cob_d$ but rather of
smooth maps in $\Man$. The cobordism so obtained is the pushed out manifold. We
further note that gluing preserves the dimensionality of the cobordisms
involved.
\end{rmk}

The domain of a TQFT will be $\Cob_d$ for some $d \geq 0$. However, there is
additional structure that $\Cob_d$ can be given and TQFTs can preserve (up to
isomorphism) and we consider this next.



\subsection{Disjoint Unions of Cobordisms}

The category of $d$--cobordisms has additional structure that we can build from
the coproduct in $\Man$ -- the disjoint union of manifolds. For any cobordism
$\fn{(M, \iota_{0, M}, \iota_{1, M})}{X}{Y}$, we adopt the notation
$a_M := \iota_{0, M}$ and $b_M := \iota_{1, M}$ for convenience. We then prove
a useful lemma involving the disjoint union of cobordisms.

\begin{thm}\label{disjunion:welldef}
If $\fn{M, M'}{U}{V}$ and $\fn{N, N'}{X}{Y}$ are cobordisms with $M \eqcob M'$
and $N \eqcob N'$, then $M \amalg N \eqcob M' \amalg N'$ as cobordisms
$U \amalg X \to V \amalg Y$.
\end{thm}
\begin{proof}
The universal property of $\amalg$ yields unique maps $a_{M} \amalg a_{N}$ and
$b_{M} \amalg b_{N}$ as below:
\begin{eqnarray*}
\begin{tikzcd}[column sep=small]
& & M \amalg N & & \\
& M \arrow[ur, "q_M"] & & N \arrow{ul}[above, xshift=1.5ex]{q_N} & \\
U \arrow[ur, "a_{M}"] \arrow[rr, "q_U" below]
& & U \amalg X
    \arrow[uu, dashed, "a_{M} \amalg a_{N}" description]
& & X \arrow{ul}[right, yshift=1ex]{a_{N}} \arrow[ll, "q_X"]
\end{tikzcd} & \hspace{1.5em} &
\begin{tikzcd}[column sep=small]
& & M \amalg N & & \\
& M \arrow[ur, "q_M"] & & N \arrow{ul}[above, xshift=1.5ex]{q_N} & \\
V \arrow[ur, "b_{M}"] \arrow[rr, "q_V" below]
& & V
    \amalg Y \arrow[uu, dashed, "b_{M} \amalg b_{N}" description]
& & Y \arrow{ul}[right, yshift=1ex]{b_{N}} \arrow[ll, "q_Y"]
\end{tikzcd}
\end{eqnarray*}
where the $q_R$ maps are the natrual inclusions into the respective coproducts.
If $\partial (M \amalg N) = R_0 \amalg R_1$, then it is easy to see that
$a_{M} \amalg a_{N}$ is a diffeomorphism from $U \amalg X$ onto
$R_0$ and $b_{M} \amalg b_{N}$ is a diffeomorphism from
$V \amalg Y$ onto $R_1$. Hence, if we set
$a_{M \amalg N} := a_{M} \amalg a_{N}$ and
$b_{M \amalg N} := b_{M} \amalg b_{N}$, then
$\fn{(M \amalg N, a_{M \amalg N}, b_{M \amalg N})}
{U \amalg X}{V \amalg Y}$ is a $d$--cobordism. A similar construction yields
a cobordism
$\fn{(M' \amalg N', a_{M' \amalg N'}, b_{M' \amalg N'})}
{U \amalg X}{V \amalg Y}$.

Let $\phi : M \longleftrightarrow M' : \phi^{-1}$ and
$\psi : N \longleftrightarrow N' : \psi^{-1}$ be the diffeomorpsisms realizing
the respective equivalences of cobordisms. By the universal property of $\amalg$
again, we have unique morphisms $\phi \amalg \psi$ and
$\phi^{-1} \amalg \psi^{-1}$ making the following diagram commute:
\begin{eqnarray*}
\begin{tikzcd}[column sep=huge, row sep=large]
  U \arrow[r, "a_{M}"] \arrow[d, "q_{U}" left]
  & M \arrow[r, shift left=1ex, "\phi" above]
      \arrow[d, "q_M" left]
  & M' \arrow[l, shift left=1ex, "\phi^{-1}" below]
       \arrow[d, "q_{M'}" right]
  & U \arrow[l, "a_{M'}" above] \arrow[d, "q_{U}" right]\\
    U \amalg X \arrow[r, "a_{M \amalg N}"]
  & M \amalg N \arrow[r, dashed, shift left=1ex, "\phi \amalg \psi"]
  & M' \amalg N' \arrow[l, dashed, shift left=1ex, "\phi^{-1} \amalg \psi^{-1}"]
  & U \amalg X \arrow[l, "a_{M' \amalg N'}" above]\\
    X \arrow[r, "a_{N}"] \arrow[u, "q_{X}" left]
  & N \arrow[r, shift left=1ex, "\psi" above]
      \arrow[u, "q_N" left]
  & N' \arrow[l, shift left=1ex, "\psi^{-1}" below]
       \arrow[u, "q_{N'}" right]
  & X \arrow[l, "a_{N'}" below] \arrow[u, "q_X" right]
\end{tikzcd}
\end{eqnarray*}

In particular, we observe that $a_{M' \amalg N'}
= (\phi \amalg \psi) \circ a_{M \amalg N}$ and $a_{M \amalg N}
= (\phi^{-1} \amalg \psi^{-1}) \circ a_{M' \amalg N'}$. A similar pasting
shows the analogous identities involving $b_{M \amalg N}$ and $b_{M' \amalg N'}$
so that it now suffices to show that $\phi \amalg \psi$ and
$\phi^{-1} \amalg \psi^{-1}$ are inverses. This then follows from the
observation that their composites and the respective identity maps both make
the same coproduct diagrams commute.
\end{proof}

If, in addition to $M$ and $N$ as
above, we have $d$--cobordisms $\fn{(P, a_{P}, b_{P})}{V}{W}$ and
$\fn{(Q, a_{Q}, b_{Q})}{Y}{Z}$, we again have a cobordism
$\fn{(P \amalg Q, a_{P \amalg Q}, b_{P \amalg Q})}{V \amalg Y}
{W \amalg Z}$. We then see that

\begin{thm}\label{disjunion:glue}
$(P \amalg Q) \circ (M \amalg N) \eqcob (P \circ M) \amalg (Q \circ N)$
\end{thm}
\begin{proof}
By taking the disjoint union of the pushout diagrams of $P \circ M$ and
$Q \circ N$, and simultaneously pushing out
$M \amalg N \ot[b_{M \amalg N}] V \amalg Y \to[a_{P \amalg Q}] P \amalg Q$, we
get:
\begin{eqnarray*}
\begin{tikzcd}
V \amalg Y
  \arrow[r, "a_{P \amalg Q}" above]
  \arrow[d, "b_{M \amalg N}" left]
& P \amalg Q
  \arrow[d, "r_0 \amalg s_0" right]
  \arrow[rdd, bend left, "h_0"]
& \\
M \amalg N
  \arrow[r, "r_1 \amalg s_1" below]
  \arrow[rrd, bend right, "h_1" below]
& (P \circ M) \amalg (Q \circ N)
  \arrow[rd, dashed, shift left=1ex, "\phi" description]
& \\
& & (P \amalg Q) \circ (M \amalg N)
  \arrow[ul, dashed, shift left=1ex, "\psi" description]
\end{tikzcd}
\end{eqnarray*}
where $a_{P} \amalg a_{Q} = a_{P \amalg Q}$ and
$b_{M} \amalg b_{N} = b_{M \amalg N}$, as defined before,
and the $r_i, s_j$ and $h_k$ are the respective pushed out maps. Furthermore,
$\psi$ is a unique smooth map derived from the pushout property of
$(P \amalg Q) \circ (M \amalg N)$ and $\phi$ is a unique smooth map derived from
the universal property of the coproduct $(P \circ M) \amalg (Q \circ N)$.
Leveraging the universality of the respective identity maps and the composites
$\phi\psi$ and $\psi\phi$, we can show that $\phi$ and $\psi$ are inverses.
Finally, using similar pastings as in the proof of \ref{disjunion:welldef} or
\ref{cobglue:welldef}, we can produce a commutative diagram involving $\phi$ and
$\psi$ that realizes $(P \circ M) \amalg (Q \circ N) \eqcob
(P \amalg Q) \circ (M \amalg N)$.
\end{proof}

As a corollary of the above theorems, we have the following:
\begin{thm}\label{disjunion:id}
$((U \times I) \amalg (X \times I)) \circ (M \amalg N) \eqcob M \amalg N$
and $((M \amalg N) \circ (V \times I) \amalg (Z \times I)) \eqcob M \amalg N$
\end{thm}
\begin{proof}
For the first equivalence, by \ref{disjunion:glue}, it suffices to
show $((W \times I) \circ M) \amalg ((X \times I) \circ N) \eqcob M \amalg N$.
We know that $(W \times I) \circ M \eqcob M$ and
$(X \times I) \circ N \eqcob N$. Thus, by $\ref{disjunion:welldef}$, we get the
desired equivalence. The second equivalence is similar.
\end{proof}

\begin{cor}
$\fn{\amalg}{\Cob_d \times \Cob_d}{\Cob_d}$ is a bifunctor.
\end{cor}

We know that $\Set$ is symmetric monoidal with respect to $\amalg$ with unit
$\varnothing$. The associators, unitors and commutators of $\amalg$ in $\Set$
can be verified to be diffeomorphisms when we restrict attention to $\Man$ --
this involves simple computations involving charts. Furthermore, given the
naturality of these maps in $\Man$ (following from their naturality in $\Set$),
we can show that they yield associator, unitor and commutator cobordisms for
$\amalg$ in $\Cob_d$ as a consequence of the following results \cite{RayanCor1}
whose proofs we only sketch.

\begin{thm}\label{frob:diffcob}
For each diffeomorphism $\fn{f}{X}{Y}$ for $X, Y \in \Cob_d$, there is a
cobordism $\fn{R(f)}{X}{Y}$ in $\Cob_d$.
\end{thm}
\begin{proof}[Proof Sketch]
We observe that $f$ yields a smooth structure on $X$ as follows. Given a chart
$y_V$ on some open set $V \subset Y$, we have a chart $y_V \circ f$ on the
the open set $f^{-1}(V)$ of $X$. Since $f$ is a homeomorphism, the sets
$f^{-1}(V)$ cover $X$ and $\set[y_V \circ f]{y_V \text{ is in the atlas of } Y}$
is a smooth atlas for $X$. Thus, we can view $Y$ as another smooth structure on
$X$ and vice versa.

Let Smooth($X$) be the space of smooth structures on $X$ and Diff($X$) be the
group of diffeomorphisms of $X$ which acts on Smooth($X$) by precomposition with
charts. Then, $X$ and $Y$ are in the same orbit under this action. Since each
orbit is a path connected subspace of $\text{Smooth}(X)$, this yields a path
$\fn{\psi}{[0, 1]}{\text{Diff}(X)}$ such that $\psi(0) = \id_X$ and
$\psi(1) = f$. This path yields a $d$--manifold
$\coprod_{t \in [0, 1]} \psi(t)(X)$ with boundary
$\psi(0)(X) \amalg \psi(1)(X) = X \amalg Y$. We let this $d$--manifold be
$R(f)$.
\end{proof}

\begin{rmk}
The path $\psi$ in the above proof is, loosely, a homotopy
$\id_{X} \Rightarrow f$, if we consider $X$ and $Y$ to be the same topological
space with possibly different atlases.
\end{rmk}

\begin{cor}[Cylinder Construction]\label{disjunion:cylinderconst}
Let $\textbf{Diff}_d$ be the groupoid of diffeomorphisms between closed
$d$--manifolds. Then, $R$ is a functor $\textbf{Diff}_{d - 1} \to \Cob_d$.
\end{cor}
\begin{proof}[Proof Sketch]
Let $f$ and $g$ be composable pairs of diffeomorphisms. Then the path in
Diff($X$) associated to $gf$ by the construction in \ref{frob:diffcob} can be
shown to be a concatenation of the paths associated to $g$ and $f$, after
reparametrization. This corresponds to a glueing that shows
$R(g) \circ R(f) \eqcob R(gf)$. The identity diffeomorphism on a manifold must
produce a path $\psi$ in Diff($X$) with the same value at each $t \in [0, 1]$
for all $(d - 1)$--manifolds $X$. In particular, $\psi(t)(x) = \psi(0)(x) =
\id_X(x) = x$ for all $t \in [0, 1]$ and $x \in X$ so that
$\coprod_{t \in [0, 1]} \psi(t)(X) = \coprod_{t \in [0, 1]} X$ which is
diffeomorphic to $X \times I$. This shows that $R(\id_X) \eqcob X \times I$.
\end{proof}

\begin{rmk}
As in \cite{Jorge}, we could also take the cobordism $R(f)$ to be
$(X \times I, a_{X \times I}, f^{-1})$ where $a_{X \times I}$ is the orientation
reversal map. However, this would involve a significant amount of diagram
chasing to prove the functoriality of $R$.
\end{rmk}

The theorems proved so far yield the following result.
\begin{thm}
$(\Cob_d, \amalg, \varnothing)$ is a symmetric monoidal category.
\end{thm}



\subsection{Topological Quantum Field Theories}

We are now equipped to define topological quantum field theories in enough
generality to connect them to quantum information.

\begin{defn}[{$(d, \mathbf{k})$}--TQFT]
A topological quantum field theory of dimension $d$ over a field $\mathbf{k}$
or a $(d, \mathbf{k})$--TQFT is a strong symmetric monoidal functor
$\fn{Z}{(\Cob_d, \amalg, \varnothing)}
{(\Vect_{\mathbf{k}}, \otimes, \mathbf{k})}$.
\end{defn}

This definition captures the original definition of TQFTs as formulated by
Atiyah \cite{Atiyah1}. According to this definition, a TQFT:
\begin{enumerate}[(i)]
\item associates to each closed manifold $X$ of dimension $d - 1$, a
$\mathbf{k}$--vector space (or module) $Z(X)$

\item associates to each $d$--manifold $M$ with boundary $X^* \amalg Y$, an
element $Z(M) \in Z(X^*) \otimes Z(Y)$

\item requires, for $(d - 1)$--manifolds $X$ and $Y$,
$Z(X \amalg Y) \cong Z(X) \otimes Z(Y)$

\item requires, for each $(d - 1)$--manifold $X$, $Z(X^*) \cong Z(X)^*$ where
$(-)^*$ is orientation reversal on the left and taking the dual vector space on
the right.
\end{enumerate}

Property (i) is clearly captured by our definition. For (ii), we observe that if
we can prove (iv), then $Z(X^*) \otimes Z(Y) \cong Z(X)^* \otimes Z(Y) \cong
\Hom_{\Vect_{\mathbf{k}}}(Z(X), Z(Y))$ so that $Z(M)$ is, in fact, a linear map
$Z(X) \to Z(Y)$ which is captured by the functoriality in our definition.
Property (iii) is implied in the preservation of monoidal products (up to
natural isomorphism) by a symmetric monoidal functor. In order to prove that
(iv) is captured by our definition, we first note a sufficient condition for a
vector space $V$ to be isomorphic to the dual of a vector space $W$ -- the proof
of this can be found in \cite{Corominas}.

\begin{lem}\label{vect:nondegen}
For any $V \in \Vect_{\mathbf{k}}$, let
$\rho_V : V \otimes \mathbf{k} \longleftrightarrow \mathbf{k} \otimes V :
\lambda_V$ be the unitor isomorphisms arising from the symmetric monoidal
structure of $\Vect_{\mathbf{k}}$.
For $V, W \in \Vect_{\mathbf{k}}$, if a pair of linear maps
$\fn{\alpha}{V \otimes W}{\mathbf{k}}$ and
$\fn{\beta}{\mathbf{k}}{W \otimes V}$ satisfy the following:
\begin{eqnarray*}
\begin{tikzcd}[column sep=large]
V \otimes \mathbf{k}
  \arrow[d, "\id_{V} \otimes \beta" left]
  \arrow[rd, "\rho_V" above right] & \\
V \otimes W \otimes V \arrow[r, "\alpha \otimes \id_V" below]
& \mathbf{k} \otimes V
\end{tikzcd} &
\begin{tikzcd}[column sep=large]
\mathbf{k} \otimes W
  \arrow[r, "\beta \otimes \id_{W}" above]
  \arrow[rd, "\lambda_{W}" below left] &
W \otimes V \otimes W \arrow[d, "\id_W \otimes \alpha" right] \\
& W \otimes \mathbf{k}
\end{tikzcd}
\end{eqnarray*}
then $V \cong W^*$ and $W \cong V^*$.
\end{lem}

\begin{defn}\label{vect:nondegen:def}
A map $\alpha$ satisfying the previous theorem is called a non-degenerate
pairing between $V$ and $W$. Given that $\alpha$ is non-degenerate, $\beta$ is
called a copairing between $V$ and $W$.
\end{defn}

We now observe that for any $d$--manifold $X$, $X \times I$ is a cobordism
$\varnothing \to X^* \amalg X$ for the boundary of $X \times I$ can be written
$\varnothing \amalg ((X \times \{0\})^* \amalg (X \times \{1\}))$. Taking
$a_{X \times I}$ as the empty map $\varnothing \to X \times I$ and
$b_{X \times I}(x) $ to be $(x, 0)$ when $x \in X^*$ and $(x, 1)$ when
$x \in X$,
$\fn{(X \times I, a_{X \times I}, b_{X \times I})}{\varnothing}{X^* \amalg X}$
is the desired cobordism -- we denote this $\eta_X$. An analogous construction
shows that $X \times I$ is also a cobordism $X \amalg X^* \to \varnothing$ with
the appropriate boundary inclusions -- we denote this $\eps_X$.

\begin{defn}[Evaluation/Coevaluation]
The cobordisms $\eps_X$ and $\eta_X$ above are respectively called the
evaluation and coevaluation cobordisms of $X$.
\end{defn}

We now sketch the proof of a fundamental theorem concerning evaluation and
coevaluation cobordisms \cite[\S 1.1.8]{KockFA} which yields (iv) in Atiyah's
definition as a corollary.

\begin{thm}[Snake Identities]\label{tqft:nondegen}
For any object $X \in \Cob_d$, if $\sgm_X$ is the commutator
$\varnothing \amalg X \cong X \amalg \varnothing$, then the following diagrams
commute in $\Cob_d$:
\begin{eqnarray*}
\begin{tikzcd}[column sep=large]
X \amalg \varnothing
  \arrow[d, "\id_{X} \amalg \eta_X" left]
  \arrow[rd, "\sgm_X" above right] & \\
X \amalg X^* \amalg X \arrow[r, "\eps_X \amalg \id_X" below]
& \varnothing \amalg X
\end{tikzcd} &
\begin{tikzcd}[column sep=large]
\varnothing \amalg X^*
  \arrow[r, "\eta_X \amalg \id_{X^*}" above]
  \arrow[rd, "\sgm_{X^*}" below left] &
X^* \amalg X \amalg X^* \arrow[d, "\id_X \amalg \eps_X" right] \\
& X^* \amalg \varnothing
\end{tikzcd}
\end{eqnarray*}
\end{thm}
\begin{proof}[Proof Sketch]
By construction, $\sgm_X$ arises from a homotopy
$\phi : \id_{\varnothing \amalg X} \Rightarrow \id_{X \amalg \varnothing}$ with
each $\im \phi_t$ for $t \in [0, 1]$ diffeomorphic to $X$. This allows us to
construct a diffeomorphism $\fn{g}{\sgm_X}{X \times I}$ such that
$g(\varnothing \amalg X) = X = g(X \amalg \varnothing)$. Now, the cylinder
$X \times I$ is a four-fold gluing of itself:
\[
  X \times I \eqcob X_4 \circ X_3 \circ X_2 \circ X_1 \text{, where each }
    X_i \eqcob X \times I
\]
We then have diffeomorphisms $X_1, X_4 \to X \times I$, $X_2 \to \eps_X$
and $X_3 \to \eta_X$ which allow us to construct a diffeomorphism
$\fn{f}{X \times I}{(\id_X \amalg \eps_X) \circ (\eta_X \amalg \id_X)}$,
piecewise. Finally, it can be checked that $gf$ restricts to the identity on
$\partial \sgm_X$ showing
$\sgm_X \eqcob (\id_X \amalg \eps_X) \circ (\eta_X \amalg \id_X)$.
The diagram on the right is proved by duality.
\end{proof}

\begin{cor}
If $Z$ is a $(d, \mathbf{k})$--TQFT, then for each object $X \in \Cob_d$,
$Z(X^*) \cong Z(X)^*$.
\end{cor}
\begin{proof}
By applying $Z$ to the diagrams in \ref{tqft:nondegen}, we obtain a
non-degenerate pairing of $Z(X)$ with $Z(X^*)$ so that, by \ref{vect:nondegen},
$Z(X^*) \cong Z(X)^*$.
\end{proof}

\begin{rmk}
As noted in \cite{Corominas}, we have proved a stronger statement: $Z(X)$ must
be finite dimensional as well.
\end{rmk}

%In addition, orientation reversal is strictly involutive on objects and easily
%verified to be contravariant on morphisms in $\Cob_d$. If
%$\fn{(M, a, b)}{X}{Y} \in \Cob_d$ with $\partial M = W_0 \amalg W_1$, then
%$\partial M^* = W_0^* \amalg W_1^*$ and $a$ becomes orientation preserving
%while $b$ becomes orientation reversing yielding $(M^*, b, a)$ as a cobordism
%$Y^* \to X^*$. $(-)^*$ is easily seen to be functorial. Finally,
%\ref{tqft:nondegen} yields:
%
%\begin{thm}
%$\Cob_d$ is a compact category \cite{channels}.
%\end{thm}
%
%For completeness, we note that a compact category is a symmetric monoidal
%category with an involutive, contravariant endofunctor as $(-)^*$ above,
%satisfying the snake identities.



\subsection{Commutative Frobenius Algebras}\label{sec:frob}

TQFTs of dimension $2$ give rise to (and arise from) a simple well-known
algebraic structure. We now build our way to this result which will prove to be
the inital point of connection with quantum information. It is known that
connected $1$--dimensional manifolds are diffeomorphic to either the circle
$S^1$ or some real interval \cite{MilnorDiff}. Since a real interval either has
a boundary or is non-compact, the objects in $\Cob_2$ are diffeomorphic to
disjoint unions of $S^1$. Given that diffeomorphisms are isomorphisms in
$\textbf{Diff}_{d - 1}$ and functors preserve isomorphisms, the functor $R$
defined in \ref{disjunion:cylinderconst} yields isomorphisms in $\Cob_2$ from
any object to a disjoint union of circles. The monoidal functoriality of
$2$--dimensional TQFTs, then, establishes the following fundamental facts as
easy corollaries.

\begin{cor}
Let $Z$ be a $(2, \mathbf{k})$--TQFT and $X$, any object in $\Cob_2$. Then, for
some index set $I$, we have $X \cong_{\Cob_d} \coprod_{i \in I} S^1$ and hence,
$Z(X) \cong_{\Vect_{\mathbf{k}}} \bigotimes_{i \in I}Z(S^1)$.
\end{cor}

\begin{cor}
The object function of a $(2, \mathbf{k})$--TQFT is determined completely (up to
isomorphism) by the image of $S^1$.
\end{cor}

For a particular $(2, \mathbf{k})$--TQFT $Z$, we let $A := Z(S^1)$ and then
examine the action of $Z$ on cobordisms. It is well known that every
$2$--dimensional manifold with boudary has a collection of loops that cut it to
yield a decomposition into the following shapes (up to diffeomorphism)
\cite{Atiyah2}:
\begin{enumerate}[(i)]
  \item the cylinder $C$ (on $S^1$),
  \item the pair of pants $P$ (a cobordism $S^1 \amalg S^1 \to S^1$ of genus
  $0$) and
  \item the disk $D$ (a cobordism $S^1 \to \varnothing$ or vice versa).
\end{enumerate}
Thus, the action of a $(2, \mathbf{k})$--TQFT $Z$ on cobordisms is given by
composing $Z(C)$, $Z(P)$ and $Z(D)$ in all possible ways.

Visibly, $\fn{Z(P)}{A \otimes A}{A}$ uniquely determines a bilinear map
$\fn{m}{A \times A}{A}$. It is easy to construct (componentwise) a
diffeomorphism $P \circ (C \amalg P) \to P \circ (P \amalg C)$ that restricts
to the identity on the boundary such that these are equivalent cobordisms. Under
a TQFT, this yields:
\begin{align*}
  Z(P) \circ (Z(C) \otimes Z(P)) = Z(P) \circ (Z(P) \otimes Z(P))
\end{align*}
which extends to the associativity of $m$, seen as a bilinear multiplication on
$A$. If we take $E$ as the cobordism $\varnothing \to S^1$ obtained by reversing
the orientation of $D$, then by observing that
$P \circ (E \amalg C) \eqcob C \eqcob P \circ (C \amalg E)$, we get:
\begin{align*}
  Z(P) \circ (Z(E) \otimes \id_{A})
    &= Z(P) \circ (Z(E) \otimes Z(C))\\
    &= Z(P \circ (E \amalg C))\\
    &= Z(C)\\
    &= Z(P \circ (C \amalg E))\\
    &= Z(P) \circ (Z(C) \otimes Z(E))\\
    &= Z(P) \circ (\id_{A} \otimes Z(E))
\end{align*}
so that $Z(E)$ acts as the identity when $Z(P)$ is seen as a bilinear
multiplication. Furhtermore, let $\sgm_{X, Y}$ be the commutator diffeomorphism
$X \amalg Y \to Y \amalg X$ in $\textbf{Diff}_{1}$. Then, $R(\sgm_{X, Y})$ is
the commutator cobordism in $\Cob_2$. We then observe that
$R(\sgm_{A, A}) \circ P \eqcob P$ yielding
\[
  Z(R(\sgm_{A, A})) \circ Z(P) = Z(P)
\]
Since $R(\sgm_{A, A})$ is a commutator in $\Cob_2$ by construction and $Z$ is a
symmetric monoidal functor, $Z(R(\sgm_{A, A}))$ must be the commutator of
$\Vect_{\mathbf{k}}$. Together, these establish:

\begin{thm}
$Z(P)$ induces an associative and commutative bilinear multiplication $m$ on $A$
with unit $Z(C)$, making $A$ a commutative $\mathbf{k}$--algebra.
\end{thm}

An additional structure on $A$ gives it the form which allows us to bridge TQFTs
with quantum information. For this, we define:
\begin{defn}
For a $\mathbf{k}$--algebra $A$ with multipliction $m$, a linear map
$\fn{\beta}{A \otimes A}{\mathbf{k}}$ satisfying $\beta \circ (m \otimes \id_A)
= \beta \circ (\id_A \otimes m)$ is called an associative pairing of $A$ with
itself.
\end{defn}

\begin{defn}[Frobenius Algebra]
A $\mathbf{k}$--algebra $A$ equipped with an associative pairing which is also
non-degenerate in the sense of \ref{vect:nondegen:def} is called a Frobenius
algebra.
\end{defn}

We note that for an associative pairing to be non-degenerate, it must have an
associated copairing, by definition. Setting $B = D \circ P$ and $\beta = Z(B)$,
we observe that $B \circ (C \amalg P) \eqcob B \circ (P \amalg C)$ -- the
argument is similar to that for the associativity of $P$ -- yielding
\[
  \beta \circ (Z(P) \amalg Z(C)) = \beta \circ (Z(C) \amalg Z(P))
\]
We can further set $B' := (D \circ P)^* \eqcob P^* \circ D^*$ and
$\alpha := Z(B')$, and show that the following diagrams commute using an
argument identical to the proof of \ref{tqft:nondegen}:
\begin{eqnarray}\label{frob:s1nondegen}
\begin{tikzcd}[column sep=large]
  S^1 \amalg \varnothing
    \arrow[rd, "\sgm" above right]
    \arrow[d, "\id_{S^1} \amalg B" left] & \\
  S^1 \amalg S^1 \amalg S^1
    \arrow[r, "B' \amalg \id_{S^1}" below] &
  \varnothing \amalg S^1
\end{tikzcd} &
\begin{tikzcd}[column sep=large]
  \varnothing \amalg S^1
    \arrow[r, "B' \amalg \id_{S^1}" above]
    \arrow[rd, "\sgm" below left] &
  S^1 \amalg S^1 \amalg S^1
    \arrow[d, "\id_{S^1} \amalg B"]\\
  & S^1 \amalg \varnothing
\end{tikzcd}
\end{eqnarray}
Applying $Z$ to these diagrams shows that $\beta$ is an associative,
non-degenerate pairing of $A = Z(S^1)$ with itself with copairing $\alpha$,
establishing $A$ as a commutative Frobenius algebra with multiplication $m$
induced by $Z(P)$, unit $e := Z(C)$ and pairing $\beta$. We stress that the
commutativity is due to the commutator in $\Cob_2$.

Conversely, if we have a commutative Frobenius algebra $(A, m, e, \beta)$, we
can set $Z$ accordingly, giving rise to a $(2, \mathbf{k})$--TQFT by (finite)
glueing of disjoint unions and orientation reversals of the generating objects
and morphisms in $\Cob_2$. Finally, we can further show that monoidal natural
transformations between $2$--dimensional TQFTs give rise to homomorphisms of
commutative Frobenius algebras and vice versa. This correspondence can be
further shown to be an equivalence of categories, the details of which can be
found in \cite{Jorge}.



\subsection{Planar TQFTs}

A generalization of TQFTs to a form that captures all Frobenius algebras is in
order. This begins with the observation that a $\mathbf{k}$--algebra is exactly
a monoid object with respect to the tensor product of $\Vect_{\mathbf{k}}$ and
in defining a Frobenius algebra we did not refer to any property specific to
vector spaces. This allows us to generalize Frobenius algebras to arbitrary
monoidal categories. If a monoid object in any monoidal category is equipped a
pairing and a copairing, as in the case of vector spaces, which satisfy
non-degeneracy identities analogous to diagram \eqref{frob:s1nondegen}, we call
it a Frobenius monoid. Thus, $S^1$ is precisely a Frobenius monoid in $\Cob_2$.

\begin{defn}
Objects $O_1, \dots, O_k$, morphisms $f_1, \dots, f_m$ and propositions
$P_1, \dots, P_n$ involving the given objects and morphisms in a category
$\s{C}$, is called a structure in $\s{C}$. The smallest subcategory $\s{D}$ of
$\s{C}$ containing a structure is called the category generated by that
structure. Finally, $\s{D}$ is said to be freely generated if functors (possibly
with additional structure) $F : \s{D} \to \s{E}$ are in bijection with the
possible choices for images under $F$ of the objects defining the structure.
\end{defn}

The last paragraph in the prvious section argues precisely that $\Cob_2$ is the
symmetric monoidal category freely generated by the commutative Frobenius monoid
$S^1$. The fact that the image $A$ of $S^1$ under a TQFT is a commutative
Frobenius algebra is precisely because the commutator cobordism
$S^1 \amalg S^1 \to S^1 \amalg S^1$ under the TQFT enforces commutativity on the
multiplication of $A$. We note that this commutator can be embedded in three
real dimensions but not two for this commutator is made of two cylinders, one
going ``over'' the other, which, when embedded in a two dimensional surface,
must overlap. As we will see shortly, this is precisely the way to remove
commutators from the category of cobordisms.

As noted before, the compact $1$--manifolds, up to diffeomorphism, are precisely
disjoint unions of $S^1$ or disjoint unions of the closed interval. We now
define a thick tangle or planar cobordism to be an oriented smooth $2$--manifold
embedded in the plane and whose boundaries are disjoint unions of the unit
interval $I$.  Analogous arguments as in the development of $\Cob_d$ then show
that planar cobordisms form a monoidal category with composition given by gluing
at common boundaries and monoidality given by disjoint union and the empty set.

\begin{defn}[Thick Tangles]
We denote the category of planar cobordisms (or thick tangles) as $\Thick$.
Monoidal functors out of $\Thick$ are said to be planar TQFTs or simply planar
field theories or PFTs.
\end{defn}

It can be shown, by very similar arguments as for $S^1$ in $\Cob_2$, that $I$ is
a Frobenius monoid in $\Thick$. In \cite{NonCommTQFT}, Lauda shows that $\Thick$
is equivalent to the monoidal category freely generated by a Frobenius monoid
and notes that a restatement of this is the following:

\begin{thm}
Given a PFT, $\fn{Z}{\Thick}{\Vect_{\mathbf{k}}}$, there is a unique Frobenius
algebra $A = Z(I)$. Conversely, given a Frobenius algebra $A$, we have a unique
PFT $Z$ determined by $Z(I)$.
\end{thm}

This theorem establishes that all Frobeniusm algebras, commutative or not, are
accounted for by PFTs. A corollary of this is that $\Thick$ is monoidal but
cannot have a commutator because if it did, the same argument as for TQFTs would
show that PFTs are equivalent to commutative Frobenius algebras, a
contradiction.



