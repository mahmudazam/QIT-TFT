
\pagebreak

\section{Introduction}

A qubit in quantum information and computing is modelled as a variable taking
values in the complex vector space $\C^2$.\footnote{This simple description will
be sufficient for us.} A register of length $n$, then, is a variable taking
values in the $n$--fold tensor product $Q_n := \bigotimes_{i = 1}^n \C^2$. State
transformations in quantum mechanics and hence gates in quantum computing are
modelled by unitary maps $Q_n \to Q_m$. Quantum circuits are sequences of
unitary maps, the composites of which are again unitary maps. A standard result
in linear algebra is that the space $\s{L}(V) := \Hom_{\Vect_{\C}}(V, V)$ of
operators on a finite dimensional Hilbert space $V$ form a $C^*$ algebra with
the product given by composition and the involution given by the conjugate
transpose, since the operators are necessarily bounded. The unitary maps on $V$
then form a $*$--subalgebra of $\s{L}(V)$, as the conjugate transpose of a
unitary map is again unitary.

The basic scenario in quantum information is where Bob sends a message to Alice
in the form of a quantum mechanical system. Bob prepares a quantum system in
some specific state and transmits the state to Alice through a medium. The
medium for sending the message, referred to as a channel, may introduce some
noise. To capture this communication scenario mathematically, we assume that the
message sent by Bob is a probability distribution of possible quantum states.
More explicitly, a message from Bob is a probability density function
$\fn{p}{\mathcal{X}}{\sbr{0, 1}}$ along with a set of quantum states
$\set{\psi_x}_{x \in \mathcal{X}}$ in some Hilbert space, typically finite
dimensional, satisfying that $p(x)$ is the probability that the state prepared
and sent by Bob is $\psi_x$ \cite{cqshannon}. We note that in the case that
there is no channel noise, we can simply take $p(x) = 1$ if $\psi_x$ is the
state prepared by Bob and $p(x) = 0$, otherwise.

Suppose we have a collection of projective measurement operators $\Pi_j$ such
that $\sum_{j} \Pi_j$ is the identity. It can be shown that the probability that
we obtain measurement outcome $j$ upon measuring a message is
\[
  \text{trace}\br{\Pi_j \sum_{x \in \mathcal{X}} p(x) \br{\psi_x \m \psi_x}}
\]
where $\rho := \sum_{x \in \mathcal{X}} p(x) \br{\psi_x \m \psi_x}$ is called
the density operator of the message. Furthermore, the density operator contains
all the information necessary to compute the probabilities for any measurement
on the given message and hence we can take a message to be its density operator.
The evolution of messages is also captured in this formalism. It is natural to
define the action of a unitary transformation $U$ on a message
$(p, \set{\psi_x}_{x \in \mathcal{X}})$ with density operator $\rho$ as:
\[
  U \cdot (p, \set{\psi_x}_{x \in \mathcal{X}})
    \mapsto (p, \set{U\psi_x}_{x \in \mathcal{X}})
\]
We can then show that the density operator of this new message is
$U \rho U^{\dagger}$, where $(-)^{\dagger}$ is the conjugate transpose. Finally,
the density operators form a $C^*$ algebra with multiplication given by
composition and involution, by the conjugate transpose, noting that density
operators are invariant under the conjugate transpose. The proofs of the claims
above can be found in \cite[{$\S 4$}]{cqshannon}.

We observe that all of the constructs above are captured by the formalism of
$C^*$ algebras making them fundamental objects of study in quantum information.
Furthermore, the theory of $C^*$ algebras can be treated completely in the
language of symmetric monoidal categories. This has been shown in
\cite{channels} by proving an equivalence of categories between the category of
finite dimensional $C^*$ algebras under completely positive maps and a category
constructed from the so called normalizable dagger Frobenius algebra objects in
$\FHilb$\footnote{the category of finite dimensional Hilbert spaces and linear
maps} under morphisms satisfying a condition equivalent to complete positivity.
Furthermore, it is well known that two dimensional topological quantum field
theories (TQFTs) are equivalent to commutative Frobenius algebras. In this
thesis, we outline a modification of TQFTs that allows us to connect them to
$C^*$ algebras and, by extension, quantum information and computing.

